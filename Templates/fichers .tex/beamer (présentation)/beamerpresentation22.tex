\documentclass[mathserif,10pt]{beamer}
\usetheme{Madrid}
\usepackage[utf8]{inputenc}
\usepackage{amsmath,amssymb,amsthm}
\usepackage{amsmath}
\usepackage{amsfonts}
%\usepackage{fancybox}
\usepackage{fancybox}
\usepackage{xr-hyper}
\externaldocument{doc}
\usepackage[T1]{fontenc}
\usepackage[frenchb]{babel}
%\usetheme{Antibes}
\uselanguage{French}
\usepackage{bbding}
\languagepath{French}
\setlength\parindent{0pt}
%\setbeamertemplate{navigation symbols}{}
\newtheorem{theo}{Théorème}
\newtheorem{proposition}{Proposition}
\setbeamertemplate{footline}[frame number]
\renewcommand{\r}{\mathbb{R}}
\setbeamertemplate{theorems}[numbered]
\usepackage{subfigure} 
\usepackage{bbding}
\hypersetup{colorlinks,linkcolor=blue,urlcolor=links}
\author{\textcolor{magenta} {
				Soutenance de Magister en Mathématiques}\\
				\textcolor{magenta} {Par : **********}\\
				\textcolor{black} {Directeur de mémoire : Boubaker-Khaled SADALLAH\\}}

\title{\textbf{ Comportement asymptotique des valeurs propres liées à l'équation d'Allen-Cahn} }
\setbeamercovered{transparent} 
%\setbeamertemplate{navigation symbols}{1111111111} 
%\logo{\includegraphics[scale=0.1]{ens.png}  } 
\institute{\'Ecole Normale Supérieure\\
Kouba, Alger} 
\date{\today} 
%\subject{1111111} 
\begin{document}

\begin{frame}
\titlepage
\end{frame}

\begin{frame}\transglitter[duration=2]
\frametitle{Plan de l'exposé} 
\tableofcontents \pause
\end{frame}
%\chapter{kkk}
\section{Introduction générale}
%\subsection{Résultats fondamentaux sur les intégrales elliptiques complètes }
\begin{frame}{}\transdissolve[duration = 1]
\begin{center}
\shadowbox{
\Huge \bf \begin{color}{blue}{Introduction}\end{color}
}
\end{center}
\end{frame}
\begin{frame}{Introduction générale}\transglitter[duration=1]
\textbf{Introduction}\\ \ \\
L'équation d'Allen-Cahn  est un modèle mathématique simple pour certains procédés de séparation de phases. Elle sert également comme  exemple type pour les équation semi-linéaires paraboliques. La présence d'un petit paramètre  $ \varepsilon$ \\
$$u_{t}= \varepsilon^{2}u_{xx}+ f(u), $$ 
qui définit l'épaisseur des interfaces séparant les différentes phases rend difficile l'analyse des problèmes posés.
 L'équation d'Allen-Cahn a été initialement présentée par Samuel  Allen et John  Cahn  en 1979  dans [\ref{réf22}] pour décrire le mouvement d'opposition de phases dans les solides cristallins.
%\begin{alertblock}
%\end{alertblock}
\end{frame}
\section{Préliminaires}
\begin{frame}{}\transglitter[duration=1]
\begin{center}
\shadowbox{\parbox{10cm}{
\Huge \bf \begin{color}{blue} \begin{center} \textcolor{red}{ Chapitre 1:} Préliminaires  \end{center}  \end{color}
}}
\end{center}
\end{frame}
\begin{frame}{\textbf{Chapitre 1: Préliminaires}  }\transglitter[duration=1]
Pour résoudre ce problème  on a besoin des intégrales elliptiques.

\begin{block}{Définition:[Intégrale elliptique complète]}
Soient $k \in [0,1)$ et $\nu \in \mathbb{C}$. Les trois  intégrales elliptiques complètes sont définies respectivement par:
\begin{equation}
K(k):=\displaystyle\int_{0}^{1}\dfrac{1}{\sqrt{(1-s^{2})(1-k^{2}s^{2})}} \mathrm{d}s,
\end{equation}
\begin{equation}
E(k):= \displaystyle\int_{0}^{1}\sqrt{\dfrac{1-k^{2}s^{2}}{1-s^{2}}}\mathrm{d}s,
\end{equation}
et
\begin{equation}
\Pi(\nu ,k):=\int_{0}^{1}\dfrac{1}{(1+\nu s^{2})\sqrt{(1-s^{2})(1-k^{2}s^{2})}}\mathrm{d}s.
\end{equation}
\end{block}
\end{frame}
\begin{frame}{\textbf{Chapitre 1: Préliminaires}}\transglitter[duration=1]
\begin{block}{Définition: [Fonction de Jacobi $\mathrm{sn}$]}
 La fonction elliptique de Jacobi : $\mathbf{sn} t$  s'obtient par l'inversion  de intégrale :
 $$
 t =\displaystyle\int_{0}^{s}\dfrac{\mathrm{d}r}{\sqrt{(1-r^{2})(1-k^{2}r^{2})}}, \quad 0\leq k <1.
 $$
 D'o\`u  $ s = \mathbf{sn} t = \mathbf{sn}(t,k)$.
\end{block}
\end{frame}
\section{Linéarisation de l'équation d'Allen-Cahn }
\begin{frame}{Chapitre 2: Linéarisation de l'équation d'Allen-Cahn  }\transglitter[duration=1]
%\textcolor{red} {\HandPencilLeft} 
Considérons le problème aux limites non linéaire stationnaire d'Allen-Cahn  suivant :
\begin{equation}\label{cha3.1}
\begin{cases}
\varepsilon^{2}u_{xx}(x)+f(u(x))=0 \quad x \in  \ (0,1), \\[12pt]
u_{x}(0)=u_{x}(1)=0, 
\end{cases}
\end{equation}
où $\varepsilon $ est un (petit) paramètre strictement positif et $f$ est assez régulier. \pause 
On note 
$$
F(u):= \int_{0}^{u}f(s)ds,
 $$
et on suppose que $f$ est impaire et possède une non-linéarité bistable équilibrée. \pause
i.e. ayant uniquement trois zéros $u_{-} < 0 <u_{+}$ tels que 
 \begin{equation}\label{cha3.3}
 f_{u}(0)> 0, \hspace*{1cm} f_{u}(u_{\pm})< 0, \hspace*{1cm}  F(u_{+})=F(u_{-}).
\end{equation}
\end{frame}
\begin{frame}{Chapitre 2: Linéarisation de l'équation d'Allen-Cahn  }\transglitter[duration=1]
En particulier, pour $n \in \mathbb{N} $ arbitraire, \eqref{cha3.1} admet des n-mode solutions  $\pm u_{n,\varepsilon} (x)$ lorsque $\varepsilon $  est petit.\\
% Ici $u_{n,\varepsilon }$ est caractérisés comme suit: $ u_{n,\varepsilon }$ a exactement $n$ zéros $z_{l}=z_{l}^{n}, (l=1,...,n)$ et $ u_{n,\varepsilon }(0) > 0.$\\
 Le problème aux valeurs propres linéarisé de \eqref{cha3.1} associé à $ u_{n,\varepsilon }$  est \pause
\begin{equation}\label{cha3.5}
\begin{cases}
\varepsilon^{2}\varphi_{xx}(x)+f_{u}(u_{n,\varepsilon }(x))\varphi (x) +\lambda \varphi (x)=0 \quad  \mathrm{dans} \ \ (0,1), \\
\varphi_{x}(0)=\varphi_{x}(1)=0.
\end{cases}
\end{equation}
Pour $ j \in \mathbb{N} \cup \{0\}$, nous désignons par $\lambda_{j} = \lambda_{j}^{n,\varepsilon }$ et $\varphi_{j} =\varphi_{j}^{n,\varepsilon }$ respectivement la $(j+1)$ ème valeur propre et fonction propre.\pause

Wakasa  et Yotsutani en 2015 ramènent le problème \eqref{cha3.5} (grâce à un changement d'échelle et à des considérations physiques) à l'étude du problème 
\begin{equation}\label{cha3.7}
\left\{ \begin{array}{cc}
\Phi_{zz}(z)+f_{u}(U(z))\Phi (z) + \Lambda \Phi (z) =0 & \quad \mathrm{ dans} \ \mathbb{R}, \\
\Phi \in L^{2} ( \mathbb{R} ). \hspace*{3.8cm}& \
\end{array} \right.
\end{equation}
\end{frame}
\begin{frame}{Chapitre 2: Linéarisation de l'équation d'Allen-Cahn  }\transglitter[duration=1]
et ils ont remarqué qu'il y a un nombre fini de fonctions propres spéciales $\varphi_{j}^{n,\varepsilon}$  qui 
contrôlent les autres fonctions propres. Elles coïncident avec $\Phi_{j}$  de \eqref{cha3.7}. Par cette observation, T. Wakasa, S. Yotsutani en 2015 [\ref{réf17}] proposent la conjecture suivante  relativement aux éléments propres de \eqref{cha3.5}:\\ \pause
\begin{center}
\begin{tabular}{|c|}
\hline 
 Toutes les valeurs propres et fonctions propres de \eqref{cha3.5} sont classées \\
 en plusieurs groupes en fonction
 de la structure de \eqref{cha3.7} et il  y a une \\
 propriété asymptotique universelle dans chacun des groupes.\hspace*{1.25cm}
\\ 
\hline 
\end{tabular} 
\end{center}
Cette conjecture a été résolue dans le cas $f(u)=u-u^{3}$ en 2015-2016.\\
Les principaux résultats de ces travaux sont: \pause
\begin{theorem}\label{theorem1}
 Supposons que $f(u)= u- u^{3}$. Soit $n \in \mathbb{N}$. quand $\varepsilon \longrightarrow 0,$ dans \eqref{cha3.5} on obtient
 \begin{enumerate}
 \item $ \lambda_{0}^{n,\varepsilon } = -96 \exp (-\frac{\sqrt{2}}{n\varepsilon}) + o(\exp (-\frac{\sqrt{2}}{n\varepsilon})), $
  \item $\lambda_{n}^{n,\varepsilon } = \frac{3}{2} -12 \exp (-\frac{1}{\sqrt{2}n\varepsilon}) + o(\exp (-\frac{1}{\sqrt{2}n\varepsilon})),$
  \item $ \lambda_{2n}^{n,\varepsilon } = 2 + 96 \exp (-\frac{\sqrt{2}}{n\varepsilon}) + o(\exp (-\frac{\sqrt{2}}{n\varepsilon})).$
 \end{enumerate}
 \end{theorem}
 \end{frame} 
 \begin{frame}\transglitter[duration=1]
\begin{theorem}\label{theorem2}
%(voir [\ref{réf17}],Theorem 1.2, p.3967, [\ref{réf17}], Proposition 1.2, p. 5471) 
Supposons que $f(u)= u- u^{3}$. Soient $n \in \mathbb{N}$ fixé et $j \in \mathbb{N}\cup \{0\}$ . Quand $\varepsilon \longrightarrow 0$, dans \eqref{cha3.5} on obtient
\begin{enumerate}
\item  Pour $0 < j < n,$ 
$$\lambda_{j}^{n,\varepsilon} = -96 \cos^{2}\frac{j\pi}{2n}. \exp (-\frac{\sqrt{2}}{n\varepsilon}) + o (\exp (-\frac{\sqrt{2}}{n\varepsilon})),$$
\item   Pour $n < j < 2n,$
$$ \lambda_{j}^{n,\varepsilon} = \frac{3}{2} - 12 \cos\frac{(j - n)\pi}{n}\exp (-\frac{1}{\sqrt{2}n\varepsilon}) + o(\exp (-\frac{1}{\sqrt{2}n\varepsilon})),$$
\item  Pour $j > 2n,$
$$  \lambda_{j}^{n,\varepsilon} = 2 + (j -2n)^{2}\pi^{2}\varepsilon^{2} + o(\varepsilon^{2}).$$
\end{enumerate} 
\end{theorem}
\end{frame}
\section{ L'équation de représentation}
\begin{frame}{}\transglitter[duration=1]
\begin{center}
\shadowbox{\parbox{10cm}{
\Huge \bf \begin{color}{blue} \begin{center} \textcolor{red}{ Chapitre :2} Linéarisation de l'équation d'Allen-Cahn \end{center}  \end{color}
}}
\end{center}
\end{frame}
\begin{frame}{Chapitre 3:L'équation de représentation}\transglitter[duration=1]
Le concept pour obtenir l'équation de représentation commence par la recherche d'une solution de l'équation \eqref{cha3.5} 
de la forme $\varphi (x)=\mathcal{P}(u(x)),$ où $\mathcal{P} \in C^{2}([u_{-},u_{+}])$ et $u(x)=u_{n,\varepsilon }(x)$ pour un certain $n\in \mathbb{N}.$ 
On pose $\alpha=\alpha_{n,\varepsilon }=u_{n,\varepsilon}(0).$

On déduit de l'équation \eqref{cha3.1} et du fait $-\alpha < u(x) <\alpha$ que
\begin{equation}\label{eq29}
2(F(\alpha )-F(u))\mathcal{P}_{uu}(u)-f(u)\mathcal{P}_{u}(u)+(f_{u}(u)+\lambda )\mathcal{P}(u)=0,\ u\in (-\alpha ,\alpha ).
\end{equation}
On dit que  \eqref{eq29} est l'équation de représentation 
de deuxième ordre de \eqref{cha3.5}.\\ \pause
Afin de donner une classe plus large de fonctions propres liées à l'équation \eqref{cha3.5}, on utilise l'équation de troisième ordre 
\pause
\begin{equation}\label{eq30}
2(F(\alpha)-F(u))\mathcal{Q}_{uuu}(u)-3f(u ).\mathcal{Q}_{uu}(u)+(3f_{u}(u)+4\lambda )\mathcal{Q}_{u}(u)+2f_{uu}(u)\mathcal{Q} (u)=0,
\end{equation}
et on pose 
\begin{equation}\label{eq32}
\rho (\lambda;\alpha ):=\frac{\mathcal{Q}(\alpha ,\lambda ; \alpha )}{2}\left[-f(\alpha )\mathcal{Q}_{u}(\alpha ,\lambda ; \alpha )+2(f_{u}(\alpha )+\lambda )\mathcal{Q}(\alpha ,\lambda ; \alpha )\right].
\end{equation}
\end{frame}
\begin{frame}{Chapitre 3:L'équation de représentation}\transglitter[duration=1]
%\setbeamercolor{block title}{fg=black,bg=green!97!green}
%\setbeamercolor{block title}{fg=black,bg=pink!96!pink}
\setbeamercolor{block title}{fg=black,bg=cyan!96!cyan}
\begin{exampleblock}{}
\begin{proposition}\label{prop2.8}
Soit $f$ une fonction impaire de classe $C^{2}$ qui satisfait \eqref{cha3.3}. 
%Soient $\alpha_{n,\varepsilon }$ et $u_{n,\varepsilon }$ définies par la Proposition \ref{prop2}.
Supposons que \eqref{eq30} admet une solution particulière $\mathcal{Q}(u,\lambda ;\alpha_{n,\varepsilon} )$ 
avec $\rho (\lambda ,\alpha_{n,\varepsilon})> 0.$ Alors
$$
\varphi (x;\lambda )= \sqrt{\mid \mathcal{Q}(u_{n,\varepsilon }(x),\lambda ; \alpha_{n,\varepsilon })\mid  }\cos \left( \dfrac{1}{\varepsilon }\int\limits_{0}^{x}\dfrac{\sqrt{\rho (\lambda , \alpha_{n,\varepsilon })}}{\mid  \mathcal{Q}(u_{n,\varepsilon }(\xi ),\lambda ; \alpha_{n,\varepsilon })\mid }d\xi \right),
$$
et
$$
\tilde{\varphi}(x;\lambda )= \sqrt{\mid \mathcal{Q}(u_{n,\varepsilon }(x),\lambda ; \alpha_{n,\varepsilon })\mid  }\sin \left( \dfrac{1}{\varepsilon }\int\limits_{0}^{x}\dfrac{\sqrt{\rho (\lambda , \alpha_{n,\varepsilon })}}{\mid  \mathcal{Q}(u_{n,\varepsilon }(\xi ),\lambda ; \alpha_{n,\varepsilon })\mid }d\xi \right)
$$
sont les deux solutions linéairement indépendantes de
\begin{equation}\label{eq2.14}
\varepsilon^{2}\varphi_{xx}(x)+ f_{u}(u_{n,\varepsilon }(x))\varphi (x) + \lambda \varphi (x)= 0.
\end{equation}
\end{proposition}
\end{exampleblock}
\end{frame}
\section{Expressions des fonctions propres }
\begin{frame}{}\transglitter[duration=1]
\begin{center}
\shadowbox{\parbox{10cm}{
\Huge \bf \begin{color}{blue} \begin{center} \textcolor{red}{ Chapitre :4}  \ \ Expressions des fonctions propres  \end{center}  \end{color}
}}
\end{center}
\end{frame}
\begin{frame}{Chapitre 4: Expressions des fonctions propres }\transglitter[duration=1]
Soit $\mathcal{A}_{0}(k):=\sqrt{1+k^{2}}K(k) \ \ \mathrm{pour \ tout }  \  k\in (0,1),
$ on a la \pause
\begin{exampleblock}{}
%\setbeamercolor{block title}{fg=black,bg=green!97!green}
\setbeamercolor{block title}{fg=black,bg=cyan!96!cyan}
\begin{proposition}\label{pro12}
%\begin{proposition}
%(voir [\ref{réf17}], Proposition 2.1, p.3968)
Soit $n$ fixé dans $\mathbb{N}$. On pose $f(u)=u-u^{3}.$ 
L'équation  \eqref{cha3.1} admet deux $n$-mode solutions $\pm u_{n,\varepsilon}(x)$ si et seulement si $\varepsilon \in (0,1/(n\pi )).$ En outre, $ u_{n,\varepsilon }$ est donnée par
\begin{align}\label{eq4.1}
\nonumber
 u_{n,\varepsilon } (x)&= \sqrt{\frac{2k_{n,\varepsilon }^{2}}{1+k_{n,\varepsilon }^{2}}}\mathrm{sn}  \left(\frac{1}{\sqrt{1+k_{n,\varepsilon }^{2}}}\left( x+\frac{1}{2n}\right) ,k_{n,\varepsilon } \right)\\
&=  \sqrt{\frac{2k_{n,\varepsilon }^{2}}{1+k_{n,\varepsilon }^{2}}}\mathrm{sn}   \left(2nK(k_{n,\varepsilon }) \left( x+\frac{1}{2n}\right) ,k_{n,\varepsilon } \right), 
\end{align}
où $k_{n,\varepsilon }$ est la solution unique de 
$
\mathcal{A}_{0}(k)=\frac{1}{2n\varepsilon }, \  k \in (0,1).
$
\end{proposition}
\end{exampleblock}
\end{frame}

 \begin{frame}{Chapitre 4: Expressions des fonctions propres } \transglitter[duration=1]
\begin{theorem}\label{th1}
Soient $n \in \mathbb{N}$ et $\varepsilon \in (0,1/(n\pi ))$ fixés. 
On pose $f(u)=u-u^{3}$.
Soient $k_{n,\varepsilon }$ et $u_{n,\varepsilon }$ donnés par la Proposition \ref{pro12}. 
Alors \eqref{cha3.5} admet les valeurs propres et fonctions propres suivantes: 

%\begin{large}
$\mathbf{(i)}\left\{ \begin{array}{c}
\lambda_{0}^{n,\varepsilon } = \frac{1+k_{n,\varepsilon }^{2}-
2\sqrt{1-k_{n,\varepsilon }^{2}+k_{n,\varepsilon }^{4}}}{1+k_{n,\varepsilon }^{2}},
\hspace*{4.5cm} \\
\hspace*{-1cm}\varphi_{0}^{n,\varepsilon }(x) =1-\frac{(1+k_{n,\varepsilon }^{2} )(1+k_{n,\varepsilon }^{2}-\sqrt{1-k_{n,\varepsilon }^{2}+k_{n,\varepsilon }^{4} })}{2k_{n,\varepsilon }^{2} }u_{n,\varepsilon}(x)^{2}.
\end{array}\right.
$
$\mathbf{(ii)}\left\{ \begin{array}{c}
\lambda_{n}^{n,\varepsilon } =\frac{3k_{n,\varepsilon }^{2}}{1+k_{n,\varepsilon }^{2} },
\hspace*{5cm} \\
\hspace*{-1cm}\varphi_{n}^{n,\varepsilon }(x)=2u_{n,\varepsilon }(x)\sqrt{\frac{2}{1+k_{n,\varepsilon }^{2}}-u_{n,\varepsilon }(x)^{2}}.
\end{array}\right.
$
%\begin{small}
$\mathbf{(iii)}\left\{ \begin{array}{c}
\lambda_{2n}^{n,\varepsilon } = \frac{1+k_{n,\varepsilon }^{2}+
2\sqrt{1-k_{n,\varepsilon }^{2}+k_{n,\varepsilon }^{4}}}{1+k_{n,\varepsilon }^{2}},
\hspace*{5.5cm} \\
\hspace*{-1cm}\varphi_{2n}^{n,\varepsilon }(x)=\frac{1}{2}[-1+\frac{(1+k_{n,\varepsilon }^{2} )(1+k_{n,\varepsilon }^{2}-\sqrt{1-k_{n,\varepsilon }^{2}+k_{n,\varepsilon }^{4} })}{2k_{n,\varepsilon }^{2} }u_{n,\varepsilon}(x)^{2}].
\end{array}\right.
$
%\end{small}
\end{theorem} 
\end{frame}
%\begin{frame}{Chapitre 4: Expressions des fonctions propres }
\begin{frame}{Chapitre 4: Expressions des fonctions propres }\transglitter[duration=1]
 On va donner les autres expressions des fonctions propres de \eqref{cha3.5}.

\pause
On introduit maintenant la fonction caractéristique $\mathcal{A}_{1}$ pour l'équation \eqref{cha3.5} avec $f(u)=u-u^{3}$ et $ k \in (0,1)$:
%\begin{equation}\label{eq4.4}
\begin{align*}
\mathcal{A}_{1}(k,\mu )=
\begin{cases}
\frac{\sqrt{\mathcal{R}(k,\mu)}  }{3\sqrt{3}k^{2} } (\nu_{+}\Pi (\nu_{+},k)-\nu_{-}\Pi (\nu_{-},k)) \quad &\mathrm{si }\  \mu \in (\mu_{-}(k),0),\\
\frac{\sqrt{\mathcal{R}(k,\mu)}  }{3\sqrt{3}k^{2} } (\nu_{-}\Pi (\nu_{-},k)-\nu_{+}\Pi (\nu_{+},k)) \quad &\mathrm{si } \ \mu \in (3k^{2},3),\\
\frac{\sqrt{-\mathcal{R}(k,\mu)}  }{3\sqrt{-3}k^{2} } (\nu_{+}\Pi (\nu_{+},k)-\nu_{-}\Pi (\nu_{-},k)) \quad & \mathrm{si } \ \mu \in (\mu_{+}(k),+\infty ),
\end{cases}
\end{align*}
%\end{equation}
où 
\begin{align}
\mathcal{R}(k,\mu )& := -\mu (\mu -3)(\mu -3k^{2}),\\
 \mu_{\pm }(k)& := 1+k^{2}\pm 2\sqrt{1-k^{2}+k^{4}},\\
  \nu_{\pm }(k,\mu )& :=\dfrac{ 3k^{2}[\mu -3(1+k^{2})\pm\sqrt{-3(\mu -\mu_{+}(k))(\mu -\mu_{+}(k)) }] }{ 2(\mu -3)(\mu -3k^{2})}.
\end{align}

\end{frame}

\begin{frame}{Chapitre 4: Expressions des fonctions propres}\transglitter[duration=1]
\begin{theorem}\label{th2}
Soient $n \in \mathbb{N}$ et $\varepsilon \in (0,1/n\pi ))$ fixés. 
On pose $f(u)=u-u^{3}.$ 
Soient $k_{n,\varepsilon } $ et  $u_{n,\varepsilon }$
donnés dans la Proposition \ref{pro12}.


 Supposons que $j\neq 0,n,2n,$ et soit $\mu_{j}^{n}(k)$ la solution unique de 
$
\mathcal{A}_{1}(k,\mu )=\dfrac{j\pi}{2n}.
$  Alors \eqref{cha3.5} admet les valeurs propres et fonctions propres suivantes:
 \vspace*{-0.25cm}
 $$
 \lambda_{j}^{n,\varepsilon }=\dfrac{1}{1+k^{2}_{n,\varepsilon }}\mu_{j}^{n}(k_{n,\varepsilon })\quad (j\neq 0,n,2n),
 $$
 \vspace*{-0.5cm}
et
%\vspace*{-0.5cm}
$$
\varphi^{n,\varepsilon }_{j}(x)=\sqrt{\mid \mathcal{Q}_{k_{n,\varepsilon }}(u_{n,\varepsilon }(x),\mu_{j}^{n}(k_{n,\varepsilon }))\mid }\cos\left( \dfrac{1}{\varepsilon }\displaystyle\int\limits_{0}^{x}\dfrac{\sqrt{\rho_{k_{n,\varepsilon }}(\mu_{j}^{n}(k_{n,\varepsilon })) }}{\mid \mathcal{Q}_{k_{n,\varepsilon }}(u_{n,\varepsilon }(\xi ),\mu_{j}^{n}(k_{n,\varepsilon }))\mid   }d\xi \right),
$$
\vspace*{-0.25cm}
où
\vspace*{-0.5cm}
\begin{equation}\label{eq4.7}
\mathcal{Q}_{k}(u,\mu ):= \mathcal{Q}(u,\dfrac{\mu }{1+k^{2}};\dfrac{2k^{2}}{1+k^{2}} ) \quad \mathrm{et}\quad \rho_{k}(\mu ):= \rho (\dfrac{\mu }{1+k^{2}};\dfrac{2k^{2}}{1+k^{2}} ).
\end{equation}
\end{theorem}
\end{frame} 
\section{Démonstration des résultats principaux }
\begin{frame}{}\transglitter[duration=1]
\begin{center}
\shadowbox{\parbox{10cm}{
\Huge \bf \begin{color}{blue} \begin{center} \textcolor{red}{ Chapitre :5}  \ \ Démonstration des résultats principaux  \end{center}  \end{color}
}}
\end{center}
\end{frame}
\begin{frame}{Chapitre 5: Démonstration des résultats principaux }\transglitter[duration=1]
On va donner une idée  de démonstration pour le
Théorème \ref{theorem1} et le Théorème \ref{theorem2}. Pour cela nous utiliserons on particulier les théorèmes  \ref{th1} et \ref{th2}, en plus de la Proposition  suivante 

%{\color{red}{Mettez les bons numéros de ces théorèmes}}

\pause
%\setbeamercolor{block title}{fg=black,bg=green!97!green}
\setbeamercolor{block title}{fg=black,bg=cyan!96!cyan}
\begin{exampleblock}{}
%\setbeamercolor{block title}{fg=black,bg=green!97!green}
\setbeamercolor{block title}{fg=black,bg=cyan!96!cyan}
\begin{proposition}\label{prop5.1}
Soient $n \in \mathbb{N}$ arbitraire et  $k_{n,\varepsilon }$ donné par la Proposition \ref{pro12}. 
On a
$$ 
1-k_{n,\varepsilon }^{2}=16\exp \left( -\dfrac{1}{\sqrt{2}n\varepsilon }\right)+o\left(-\dfrac{1}{\sqrt{2}n\varepsilon }\right)\quad \mathrm{ quand }\
\varepsilon \to 0.
$$
\end{proposition}
\end{exampleblock}
\pause
\textbf{Idée de la démonstration du Théorème \ref{theorem1}:}\pause
\begin{itemize}
\item On utilise le Théorème \ref{th1}.\pause
\item On pose $ h(x)= \lambda_{j}^{n,\varepsilon } (j= 0,n,2n)$ et on fait le changement de variable $ x = 1-k_{n,\varepsilon}^{2} $.\pause
\item On utilise le développement de $h$ (au voisinage de 0) d'ordre 2.\pause
\item En appliquant la Proposition \ref{prop5.1} on obtient alors les résultats du théorème \ref{theorem1}.
\end{itemize}
\end{frame}
\begin{frame}{chapitre 5: Démonstration des résultats principaux }\transglitter[duration=1]
\setbeamercolor{block title}{fg=black,bg=cyan!96!cyan}
\begin{exampleblock}{}
\begin{proposition}\label{pro15}
Soit $n\in \mathbb{N}$ arbitraire et soit $j \in \mathbb{N}$ tel que $j\neq 0,n,2n.$  Soit $\mu_{j}^{n}(k)$ donné par le Théorème \ref{th2}. Quand $ k\rightarrow 1$, on a 
\begin{enumerate}
\item Si $0<j<n,$ alors 
$
\mu_{j}^{n}(k)=-\frac{3}{4}\cos^{2}\dfrac{j\pi}{2n}.(1-k^{2})^{2}+o\left((1-k^{2})^{2}\right).
$
\item Si $n<j<2n,$ alors 
$
\mu_{j}^{n}(k)=3-3\cos^{2}\dfrac{(j-n)\pi}{2n}.(1-k^{2})+o(1-k^{2}).
$
\item Si $j>2n,$ alors 
$
\mu_{j}^{n}(k)=4+\left(\dfrac{j-2n}{2n}\right)^{2}\pi^{2}K(k)^{-2}+o\left(K(k)^{-2}\right).
$
\end{enumerate}
\end{proposition}
\end{exampleblock}\pause
%\end{frame}
%\begin{frame}{chapitre 5: Démonstration des résultats principaux }
\textbf{Idée de la démonstration du Théorème \ref{theorem2}: }\pause
\begin{enumerate}
\item Pour $j \neq 0,n,2n,$  le Théorème \ref{th2} implique
$
\lambda_{j}^{n,\varepsilon }=\dfrac{1}{1+k^{2}{_n,\varepsilon }}\mu \left(k_{n,\varepsilon };\dfrac{j\pi }{2n}\right).
$\pause
\item On a
$
\dfrac{1}{1+k^{2}}=\dfrac{1}{2}\left(1+\dfrac{1}{2}(1-k^{2})\right)+o(1-k^{2})\quad \mathrm{quand}\quad k \rightarrow 1.
$\pause
\item En utilisant les Propositions \ref{prop5.1} et \ref{pro15} on obtient les résultats du théorème \ref{theorem2}.
\end{enumerate}
\end{frame}  
\section{Perspectives}
\begin{frame}{}\transglitter[duration=1]
\begin{center}
\shadowbox{\parbox{10cm}{
\Huge \bf \begin{color}{blue} \begin{center} Perspectives 
\\ et\\ Bibliographique \end{center}  \end{color}
}}
\end{center}
\end{frame}

\begin{frame}{Perspectives}\transglitter[duration=1]

La réalisation de ce travail nous a permis de traiter l'équation d'Allen-Cahn. Les études effectuées jusqu'à maintenant ont touché divers aspects relatifs aux propriétés
 des solutions  de l'équation d'Allen-Cahn. On peut voir de nombreux développements à ces études, par exemple :\pause
 
 \begin{itemize}
% \item[1.]  \'Etude de ce phénomène en considérant d'autres terme  de non-linéaires, quasi-linéaires,
 \item[1.] Choix d'autres termes non linéaires dans l'équation d'Allen-Cahn.\pause
 
 \item[2.] \'Etude du cas où les conditions aux limites sont non linéaires.\pause
 
 \item[3.] \'Etude de ce problème avec un autre terme stochastique. \pause
 
\item[4.] \'Etude   des approximations numériques de l'équation d'Allen-Cahn.\pause

\item[5.]  \'Etude  de ce type de problèmes en utilisant d'autres méthodes.\pause

\item[6.] \'Etude de ce type d'équations dans un espace à plusieurs dimensions.\pause

\item[7.] \'Etude du problème non stationnaire \pause
$$
\begin{cases}
u_{t}= \varepsilon^{2}u_{xx}(x)+f(u(x)) \quad x \in  \ (0,1), \\[12pt]
u_{x}(0)=u_{x}(1)=0, 
\end{cases}
$$
 \end{itemize}

\end{frame}
\section{Bibliographie}
\begin{frame}[allowframebreaks]\transglitter[duration=1]
  \frametitle<presentation>{Bibliographie}

\begin{thebibliography}{99}
 \beamertemplatebookbibitems
\bibitem{} 
 \textbf{Allen S. M. and Cahn J. w.:} A microscopic theory for antiphase boundary
motion and its application to antiphase domain coarsening, Acta Metall, 27, 1979,pp. 1085-1095. \label{réf22}

\bibitem{} 
 \textbf{Brunovský P.,  Fiedler B.:} Connecting orbits in scalar reaction diffusion equations. II. The complete solution, J. Differential Equations 81 (1),1989,pp. 106–135 \label{réf1}

\bibitem{} 
\textbf{Carr  J., Pego R.L.:} Metastable patterns in solutions of $u_{t}=\varepsilon^{2}u_{xx}-f(u)$, Comm. Pure Appl. Math. 42 (5),1989,pp. 523–576.\label{réf3}
\bibitem{}
 \textbf{Chafee N., Infante E.F.:} A bifurcation problem for a nonlinear partial differential equation of parabolic type, Appl. Anal. 4, 1974/75, pp.17–37. \label{réf4}
\bibitem{17}   \textbf{Wakasa T.,Yotsutani S.:} Limiting classification on linearized eigenvalue problem for 1-dimensional Allen–Cahn
equation I — asymptotic formulas of eigenvalues, J. Differential Equations 258, 2015, pp. 3960–4006. \label{réf17}
%%%%%%%%%%%%%%%%%%%%%%%%%%%%%%%%%%%%%%%%%%%%%%%%%%%%%%%%%%%%%%%
\bibitem{15}\textbf{ Wakasa T., Yotsutani S.:} Limiting classifications on linearized eigenvalue problem for 1-dimensional Allen–Cahn equation II—asymptotic profiles of eigenfunctions,J. Differential Equations 261, 2016 pp. 5465–5498 . \label{réf15}
%%%%%%%%%%%%%%%%%%%%%%%%%%%%%%%%%%%%%%%%%%%%%%%%%%%%%%%%%%%%%%%
\bibitem{13} \textbf{Wakasa T.,Yotsutani S.:} Representation formulas for some 1-dimensional linearized eigenvalue problems,Commun. Pure Appl. Anal. 7 (4), 2008, pp. 745–763. \label{réf13}

\bibitem{16}  \textbf{Whittaker E.T., Watson G.N.:} A course of modern analysis, Cambridge Mathematical Library, Cambridge University Press, Cambridge, 1996, an introduction to the general theory of infinite processes and of analytic functions;
with an account of the principal transcendental functions, reprint of the fourth (1927) edition. \label{réf16}
%%%%%%%%%%%%%%%%%%%%%%%%%%%%%%%%%%%%%%%%%%%%
\bibitem{10}  \textbf{Yuan L., Wei-Ming Ni, Shoji Y.:} On a limiting system in the Lotka–Volterra competition with crossdiffusion, Discrete Contin. Dyn. Syst. 10 (1–2), 2004, pp. 435–458. \label{réf10}
\end{thebibliography}
\end{frame}
\begin{frame}{\centering{Comportement asymptotique des valeurs propres liées à l'équation d'Allen-Cahn
\\}}\transglitter[duration=1]
\begin{center}
\Huge \bf \begin{color}{blue}Merci pour votre \\
attention !\end{color}
\end{center}
\end{frame}
\end{document}
