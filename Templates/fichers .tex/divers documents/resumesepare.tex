\documentclass[12pt,a4paper]{report}
%\usepackage[utf8]{inputenc}
\usepackage[francais]{babel}
%\usepackage[T1]{fontenc}
\usepackage{fontspec}
\usepackage{amsmath}
\usepackage{amsfonts}
\usepackage{amssymb}
\usepackage{amsthm}
\usepackage{makeidx}
\usepackage{gensymb}
\usepackage{graphicx}
\usepackage{lmodern}
\usepackage{hyperref}
\usepackage{cancel}
\usepackage{siunitx}
\usepackage{blindtext}
\usepackage{xcolor}
\usepackage{cite}
\usepackage[explicit]{titlesec}
%\usepackage[numbers]{natbib}
%\usepackage[round]{natbib}   % omit 'round' option if you prefer square brackets
\usepackage[square,sort,comma,super,authoryear]{natbib}
\everymath{\displaystyle}
\usepackage[final]{pdfpages}
%\usepackage{kpfonts}
%\usepackage{fourier}
\usepackage{fancyhdr}
\pagestyle{fancy}
%\fancyhf{}
\fancyhead[R]{\footnotesize\rmfamily\nouppercase\leftmark}
\fancyhead[L]{\footnotesize\rmfamily\nouppercase\rightmark}

%----------------------------arabic setting--------------------------------------------
%\iffalse
%   \setmainfont[Ligatures=TeX]{Roman}
   \newfontfamily\arabicfont
      [Script=Arabic,     % to get correct arabic shaping
        Scale=1.2]          % make the arabic font bigger, a matter of taste
        {Arial}     % whatever Arabic font you like

   \newcommand{\textarabic}[1]     % Arabic inside LTR
       {\bgroup\textdir TRT\arabicfont #1\egroup}
   \newcommand{\n}         [1]     % for digits inside Arabic text
       {\bgroup\textdir TLT #1\egroup}
   \newcommand{\afootnote} [1]     % Arabic footnotes
       {\footnote{\textarabic{#1}}}
   \newenvironment{Arabic}     % Arabic paragraph        {\textdir TRT\pardir TRT\arabicfont}{}
   {\textdir TRT\pardir TRT\arabicfont}{}
%\fi
%--------------------------------------------------------------------------------


%_______________________dont allow breaking math mode_______________________________________________
\relpenalty=10000
\binoppenalty=10000
%________________________________________________________________________________________









\theoremstyle{plain}
\newtheorem{theo}{\textit{Théorème}}[chapter]
\newtheorem{pro}{\textit{Proposition}}[chapter]
\newtheorem{corl}{\textit{Corollaire}}[chapter]
\newtheorem{cons}{\textit{Conséquence}}[chapter]
\newtheorem{lem}{\textit{Lemme}}[chapter]
\newtheorem{exo}{Exercice}

\theoremstyle{definition}
 \newtheorem{defn}[theo]{\textit{Définition}}
 \theoremstyle{remark}
\newtheorem*{rem}{\textit{Remarque}}
\newtheorem*{exm}{\textit{Exemple}}
\newtheorem*{nott}{\textit{Notation}}






















%------------------------------raccourcis pour les notations---------------------------


%-------------------citations pour les chapitres---------------------------------------
\makeatletter
%\renewcommand{\@chapapp}{}% Not necessary...
\newenvironment{chapquote}[2][2em]
  {\setlength{\@tempdima}{#1}%
   \def\chapquote@author{#2}%
   \parshape 1 \@tempdima \dimexpr\textwidth-2\@tempdima\relax%
   \itshape}
  {\par\normalfont\hfill--\ \chapquote@author\hspace*{\@tempdima}\par\bigskip}
\makeatother
%---------------------------------------------------------------------------------------

\newlength\chapnumb
\setlength\chapnumb{4cm}

\titleformat{\chapter}[block]
{\normalfont\sffamily}{}{0pt}
{\parbox[b]{\chapnumb}{%
   \fontsize{120}{110}\selectfont\thechapter}%
  \parbox[b]{\dimexpr\textwidth-\chapnumb\relax}{%
    \raggedleft%
    \hfill{\LARGE#1}\\
    \rule{\dimexpr\textwidth-\chapnumb\relax}{0.4pt}}}
\titleformat{name=\chapter,numberless}[block]
{\normalfont\sffamily}{}{0pt}
{\parbox[b]{\chapnumb}{%
   \mbox{}}%
  \parbox[b]{\dimexpr\textwidth-\chapnumb\relax}{%
    \raggedleft%
    \hfill{\LARGE#1}\\
    \rule{\dimexpr\textwidth-\chapnumb\relax}{0.4pt}}}






%______________________l'espace entre lignes _______________________________________________________
%\setlength{\parindent}{2em}
%\setlength{\parskip}{1em}
%\renewcommand{\baselinestretch}{1.3}
%--------------------------------page de garde---------------------------------------
\author{Abdelhakim Dahmani}
\title{Le Problème de Goursat dans un espace de Carleman }
\date{\today}

   
   
\begin{document}
   
   
   
   
   
   
%\addcontentsline{toc}{chapter}{Résumé}
\chapter*{Résumé,Abstract,\textarabic{ملخص}}
\pagenumbering{roman}

\setcounter{page}{3}


\begin{center}
\textbf{\textarabic{ملخص}}
\end{center}

\begin{Arabic}
في هذه المذكرة نهتم ببرهنة وجود و وحدانية الحل لمسألة غورسا الغير الخطية في فضاء كارلمان، حيث نقوم بتحويل المعادلة التكاملية-التفاضلية إلى مسألة نقطة صامدة في جبر بناخ معرف بإستعمال سلاسل لا منتهية كوِِنت بإستعمال متتالية محدبة لوغاريتميا ذات تزايد قابل  للمراقبة  
 \hfill
 \end{Arabic}
 

\textarabic{  \textbf{\textarabic{الكلمات المفتاحية:}}
 مشكل غورسا، النقطة الصامدة، جبر بناخ، سلسلة غير منتهية،  فضاء كارلمان. \hfill }




\begin{center}
\textbf{\textit{Résumé}}
\end{center}

Dans ce mémoire on s'intéresse à un résultat d'éxistence et d'unicité pour le problème de Goursat non-linéaire dans un espace de Carleman, on transforme le problème integro-différentiel à un problème de point fixe dans une algèbre de Banach définie par le formalisme de certaine série formelle construite à partir d'une suite logarithmiquement convexe à croissance contrôlée. 

\textbf{\textit{Mots clés:}} Problème de Goursat, Point fixe, Algèbre de Banach, série formelle, espace de Carleman.
\begin{center}
\textbf{\textit{Abstract}}
\end{center}
 In this dissertation, we are interested in an existence and uniqueness result of the non-linear Goursat problem in a Carleman space, whereby we have transformed the integrodifferential equation  into a fixed point problem in a Banach Algebra defined by using the formalism of some formal power serie constructed by a logarithmically convex sequence with a controllable increase.  

\textbf{\textit{Keywords:}} Goursat problem, fixed point, Banach algebra, formal power serie, Carleman space.


\thispagestyle{plain}

   \end{document}