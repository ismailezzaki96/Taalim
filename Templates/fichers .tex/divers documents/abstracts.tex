


%----------------------------résume en français et anglais ----------------------------

\addcontentsline{toc}{chapter}{Résumé}
\chapter*{Résumé,Abstract,molakhas}
\begin{center}
\textbf{\textit{Résumé}}
\end{center}

Dans ce mémoire on s'intéresse à un résultat d'éxistence et d'unicité pour le problème de Goursat non-linéaire dans un espace de Carleman, on transforme le problème integro-différentiel à un problème de point fixe dans une algèbre de Banach définie par le formalisme de certain série formelle construit à partir une suite logarithmiquement convexe à croissance contrôlé 

\textbf{\textit{Mots clés:}} Problème de Goursat, Point fixe, Algèbre de Banach, série formelle, classe de Carleman.
\begin{center}
\textbf{\textit{Abstract}}
\end{center}
In this thesis one is interested in a result of existence and uniqueness for the non-linear Goursat problem in a Carleman space, one transforms the integrodifferential problem into a problem of a fixed point in a Banach algebra defined by the formalism of certain formal power series constricted  by a logarithmic convex sequence with controlled increasing

\textbf{\textit{Keywords:}} Goursat problem, fixed point,Banach algebra,formal power series,Carleman space

\begin{center}
\textbf{\textit{Résumé!!}}
\end{center}

Dans ce mémoire on s'intéresse à un résultat d'éxistence et d'unicité pour le problème de Goursat non-linéaire dans une classe de Carleman, on transforme le problème integro-différentiel en un problème de point fixe dans une algèbre de Banach définie par le formalisme de certain série formelle construit à partir une suite logarithmiquement convexe à croissance contrôlé 

\textbf{\textit{Mots clés:}} Problème de Goursat, Point fixe, Algèbre de Banach, série formelle, classe de Carleman.



%\thispagestyle{plain}

% pour une page sans en-tête ni pieds de page
%\chapter*{Résumés}
%\addcontentsline{toc}{chapter}{Résumé}%Pour l'ajout dans la table des matières au même rang que chapitre
%\begin{abstract}
%Dans ce mémoire on s'intéresse à un résultat d'éxistence et d'unicité pour le problème de Goursat non-linéaire dans un espace de Carleman, on transforme le problème integro-différentiel en un problème de point fixe dans une algèbre de Banach définie par le formalisme de certain série formelle construit avec une suite logarithmiquement convexe à croissance contrôlé 
%\end{abstract}
%\thispagestyle{plain}
%\thispagestyle{empty}%idem pour la page blanche qui suit
%\selectlanguage{english}% pour un typographie anglaise
%\renewcommand{\abstractname}{Abstract}%pour changer le titre
%\begin{abstract}
%My abstract: vous pouvez notez ici que l'espacement entre le mot abstract et les : n'est pas le même qu'en français, comme le veut la typographie anglaise.
%\end{abstract}
%\thispagestyle{plain}

%\thispagestyle{empty}%
%\selectlanguage{french}% on n'oublie pas de repasser en langue française.
%---------------------------------------------------------------------------------
