 \documentclass[dvipsnames,mathserif]{beamer}
\usepackage{pgf,tikz}
\usepackage{pgfplots}
\usepackage{mathrsfs}
%\usepackage{movie}
\usepackage{tikzpeople}
\usepackage{amsmath}
\usepackage{amsfonts}
\usepackage{amssymb}
\usetikzlibrary{shapes.callouts}
\usepackage[charter]{mathdesign}
%\usepackage{fontspec}
\usepackage{euler}
%\defaultfontfeatures{Mapping=tex-text}
\usetikzlibrary{arrows}
\usetikzlibrary{calc}
\usepackage{varwidth}
\usepackage{listings}
\usepackage{graphicx}
\usepackage{float}
\usepackage{hyperref}
\usetikzlibrary{shapes.geometric}
\usepackage{pifont}
\usepackage{colortbl}
\hypersetup{pdfpagemode=FullScreen}
\usepackage{ifthen}
\usepackage{animate}
\usepackage{empheq}
\usepackage{lipsum,multicol}
\usepackage{lipsum}
\usepackage{tikzpeople}
\usepackage{polyglossia}
\setdefaultlanguage[locale=algeria]{arabic}
\setotherlanguage{english}
\newfontfamily\arabicfont[Script=Arabic,Scale=1.3]{Andalus}
\newfontfamily\arabicfontsf[Script=Arabic]{Amiri}
\newfontfamily\arabicfonttt[Script=Arabic]{Times New Roman}
\newfontfamily\fonttiltle[Script=Arabic]{arabswell_1}%arabswell_1
%\newfontfamily\aa[Script=Arabic,Scale=2.5]{Aldhabi}
\newfontfamily\arabicfonttt[Script=Arabic,Scale=1]{ae_Granada}
\usetheme{Frankfurt}%Frankfurt,CambridgeUS,Madrid
\setbeamertemplate{footline}[frame number]
\usefonttheme[]{serif}
%\useoutertheme{miniframes}
%\usecolortheme{crane}
%\usecolortheme{whale}
\setbeamercovered{transparent} 
\usecolortheme{rose}%,wolverine,albatross,whale,seahorse,rose
\usetikzlibrary{decorations.pathreplacing}
\usetikzlibrary{arrows,shapes}

\usepackage[most]{tcolorbox}

\usepackage{color}
\usepackage{xcolor}
%%%%%%%%%%%%%%%%%%%%%%%%%%

%%%%%%%%%%%%%%%%%%%%%%%%%%
\usetikzlibrary{decorations.pathmorphing}
\tcbuselibrary{skins}
\newtcolorbox{bekhadda}[1][]{
tile,
colback=green!7,coltitle=blue!50!black,colbacktitle=blue!5,
center title,
toprule=1.25mm,bottomrule=1.25mm,
extras unbroken and first={
borderline north={0.25mm}{0.5mm}{blue,decoration={zigzag,amplitude=0.5mm},decorate}},
extras unbroken and last={
borderline south={0.25mm}{0.5mm}{blue,decoration={zigzag,amplitude=0.5mm},decorate}},
#1
} 
%%%%%%%%%%%%%%%%%%%%%%%%%%%%%%%%%%%%%%%%%%%
\newtcolorbox{box4}[2][]{colback=green!5!white,
colframe=green!75!black,fonttitle=\bfseries,
colbacktitle=green!85!green,enhanced,
attach boxed title to top center={yshift=-2mm},
title=#2,#1}
%%%%%%%%%%%%%%%%%%%%%%%%%%%%%%%%%%
\usepackage{pspicture}
\definecolor{monstertandark}{HTML}{F0DBB5}	
\newtcolorbox{paperbox}[2][]{
	frame hidden,
	boxrule=0pt,
	breakable,
	enhanced,
	before skip=11pt plus 1pt,
	toptitle=2mm,
	boxsep=-0.06cm,
	left=8pt,
	right=8pt,
	fonttitle= \Large\sffamily \selectfont\scshape\bfseries\color{blue!50!black},
	fontupper=\sffamily \selectfont,
	title=#2,
	arc=0mm,
	parbox = false,
	borderline north={1pt}{-0.7pt}{black},
	borderline south={1pt}{-0.7pt}{black},
	interior style image=goldshade.png,
	colframe=monstertandark,
	title style image=blueshade.png,
	fuzzy shadow={0mm}{-3.5pt}{-0.5pt}{0.4mm}{black!60!white},
	overlay={
		\fill [fill=black] (frame.south west) -- ++(7pt,0) -- ++(0,-5pt) -- cycle;
		\fill [fill=black] (frame.north west) -- ++(7pt,0) -- ++(0,5pt) -- cycle;
		\fill [fill=black] (frame.north east) -- ++(-7pt,0) -- ++(0,5pt) -- cycle;
		\fill [fill=black] (frame.south east) -- ++(-7pt,0) -- ++(0,-5pt) -- cycle;
		},
	after={\vspace{10pt plus 1pt}\noindent},
	#1
}  
%%%%%%%%%%%%%%
 
\newtcbox{\otherbox}[1][]{nobeforeafter,math upper,tcbox raise base,
enhanced,frame hidden,boxrule=0pt,interior style={top color=green!10!white,
bottom color=green!10!white,middle color=green!50!yellow},
fuzzy halo=1pt with green,#1}
\usepackage{empheq}
%%%%%%%%%%%%%%%%%%%%%%%%%%%
\newtcolorbox{box1}[2]{breakable,
  enhanced,
  leftrule=0pt,
  toprule=0pt,
  outer arc=0pt,
  arc=0pt,
  colframe=#2,
  colback=#2!3,title=#1,coltitle=white,
  attach boxed title to top right,
  boxed title style={
    colback=#2,
    outer arc=0pt,
    arc=0pt,
    top=3pt,
    bottom=3pt,
    },
  fonttitle=\sffamily \bfseries}
\newtcolorbox{box2}[2]{enhanced,breakable,pad at break*=1mm,skin=enhancedlast jigsaw,
attach boxed title to top right={xshift=4mm,yshift=-0.5mm},
interior style={top color=#2!3!white,bottom color=white},
boxed title style={empty,arc=0pt,outer arc=0pt,boxrule=0pt},
underlay boxed title={
\fill[#2] (title.north east) -- (title.north west)
-- +(\tcboxedtitleheight-11.5mm,-\tcboxedtitleheight+1mm)
-- ([xshift=-4mm,yshift=0.5mm]frame.north west) -- +(0mm,-1mm)
-- (title.south east) -- cycle;
\fill[#2!30!white!70!black] ([yshift=-0.5mm]frame.north west)
-- +(-0.3,0) -- +(0,-0.2) -- cycle;
\fill[#2!30!white!70!black] ([yshift=-0.5mm]frame.north east)
-- +(0,-0.2) -- +(0.3,0) -- cycle; },
colframe=#2,
 ,title=#1,fonttitle=\sffamily ,rightrule=1mm} 
 \newtcolorbox{box3}[2]{enhanced,
attach boxed title to top right={xshift=-0.3cm,yshift=-3mm},
fonttitle=\sffamily ,arc=10pt,sharp corners=uphill,
colbacktitle=#2!45!white,coltitle=#2!10!black,colframe=#2!50!black,drop lifted shadow,
interior style={top color=yellow!10!white,bottom color=#2!10!white},
boxed title style={boxrule=0.75mm,colframe=#2!80!white,
interior style={top color=#2!10!white,bottom color=#2!10!white,
middle color=#2!50!white},
drop fuzzy shadow},
title=#1} 
%%%%%%%%%%%%%%%%%%%

 \newtcolorbox{ambox}[2]{enhanced,
attach boxed title to top center={xshift=-0.3cm,yshift=-3mm},
fonttitle=\sffamily ,arc=10pt,sharp corners=uphill,
colbacktitle=#2!45!white,coltitle=#2!10!black,colframe=#2!50!black,drop lifted shadow,
interior style={top color=yellow!10!white,bottom color=#2!10!white},
boxed title style={boxrule=0.75mm,colframe=#2!80!white,
interior style={top color=#2!10!white,bottom color=#2!10!white,
middle color=#2!50!white},
drop fuzzy shadow},
title=#1} 
%%%%%%%%%%%%%%%%%%%%%%%%%
\tcbset{colback=red!5!white,colframe=red!75!black,
arc=3mm}
%%%%%%%%%%%%%%%%%%
% \usepackage{varwidth}
\newtcolorbox{mybox}[2][]
{enhanced,
before skip=2mm,after skip=2mm,
colback=yellow!20!white,colframe=black!50,boxrule=0.2mm,
attach boxed title to top right =
    {xshift=-0.6cm,yshift*=1mm-\tcboxedtitleheight},
    varwidth boxed title*=-3cm,
    boxed title style={frame code={
                        \path[fill=tcbcol@back!30!black]
                            ([yshift=-1mm,xshift=-1mm]frame.north west)  
                            arc[start angle=0,end angle=180,radius=1mm]
                            ([yshift=-1mm,xshift=1mm]frame.north east)
                            arc[start angle=180,end angle=0,radius=1mm];
                        \path[left color=tcbcol@back!60!black,right color = tcbcol@back!60!black,
                            middle color = tcbcol@back!80!black]
                            ([xshift=-2mm]frame.north west) -- ([xshift=2mm]frame.north east)
                            [rounded corners=1mm]-- ([xshift=1mm,yshift=-1mm]frame.north east) 
                            -- (frame.south east) -- (frame.south west)
                            -- ([xshift=-1mm,yshift=-1mm]frame.north west)
                            [sharp corners]-- cycle;
                            },interior engine=empty,
                    },
fonttitle=\bfseries\sffamily,
title={#2},#1}
%%%%%%%%%%%%%%%%%%%%
% for RTL liste
\makeatletter
\newcommand{\leftm}{\@totalleftmargin}
\makeatother
\newcommand{\RTListe}{\raggedleft\rightskip\leftm}
%\input{chichamou}
% RTL frame title
\setbeamertemplate{frametitle}
{\vspace*{-1mm}
  \nointerlineskip
    \begin{beamercolorbox}[sep=0.3cm,ht=2.2em,wd=\paperwidth]{frametitle}
        \vbox{}\vskip-2ex%
        \strut\hskip1ex\insertframetitle\strut
       \vskip-0.8ex%
    \end{beamercolorbox}
}


% align subsection in toc
\makeatletter
\setbeamertemplate{subsection in toc}
{\leavevmode\rightskip=5ex%
  \llap{\raise0.1ex\beamer@usesphere{subsection number projected}{bigsphere}\kern1ex}%
  \inserttocsubsection\par%
}
\makeatother

% RTL triangle for itemize
\setbeamertemplate{itemize item}{\scriptsize\raise1.25pt\hbox{\donotcoloroutermaths$\blacktriangleleft$}} 

%\setbeamertemplate{itemize item}{\rule{4pt}{4pt}}

\defbeamertemplate{enumerate item}{square2}
{\LR{
    %
    \hbox{%
    \usebeamerfont*{item projected}%
    \usebeamercolor[bg]{item projected}%
    \vrule width2.25ex height1.85ex depth.4ex%
    \hskip-2.25ex%
    \hbox to2.25ex{%
      \hfil%
      {\color{fg}\insertenumlabel}%
      \hfil}%
  }%
}}

\setbeamertemplate{enumerate item}[square2]

\setbeamertemplate{navigation symbols}{}
\usetikzlibrary{shapes.gates.logic.US,trees,positioning,arrows}
\definecolor{BrickRed}{RGB}{132,31,39}
\tcbuselibrary{skins,listings,breakable,xparse}
%%%%%%%%%%%%%%%%%%%%%%%%%%%%
%==================================
\lstdefinestyle{json}{
   language=[LaTeX]TeX,
,escapeinside=``,
	keywordstyle=\color{blue},%
	upquote=true,
columns=flexible,
texcsstyle=*\color{blue},
	 basicstyle=\scriptsize,
	stringstyle=\color{blue},
	commentstyle=\color{red},	moretexcs={xymatrix,setbeamertemplate,usetheme,usefonttheme,,rotatebox,diagbox,newcolumntype,arraybackslash,underbracket,mathclap,closedcycle,setbeamercolor,url,rowcolors,appendix,filldraw,foreach,shade,path,definecolor,color,underset,textcolor,tikzstyle,tikzset,part,RL,LR,setdefaultlanguage,setotherlanguage,newfontfamily,arabicfont,text,iint,iiint,dfrac,tfrac,multirow,href,rhead,cfoot,thesection,lhead,fancyplain,leftmark,chapter,tableofcontents,savebox,maketitle,mathbb,includegraphics, systeme,setlength,contentsname,sysequivsign,sysaddeqsign,labelenumii,labelenumi,alph,textLR,
arabicfontsf,arabicfonttt,draw,node, addcontentsline,addplot,numberwithin,bibname,sysautonum,syscodeextracol,sysextracolsign,syslineskipcoeff,efootnote,subsection,subsubsection,abstractname,textsubscript,mathcal}}
\tcbset{enhanced,boxrule=1pt,sharp corners,rounded corners=all,bicolor,colback=green!10!white,colbacklower=white,colframe=BrickRed}
\usepackage{newverbs}
\newverbcommand{\cverb}{\color{blue}}{}
%\AtBeginSection[]
%{
 % \begin{frame}{}
   % \tableofcontents[currentsection]
  %\end{frame}
%}
\let\origtheequation\theequation
\makeatletter
\def\tagform@#1{\maketag@@@{\ignorespaces#1\unskip\@@italiccorr}}
\makeatother
\renewcommand{\theequation}{(\origtheequation)}
%\usepackage[colorlinks=true]{hyerref}
\def\equationautorefname{Eq.}

%===================================================================
\begin{document}

\setbeamertemplate{background canvas}{\includegraphics[width=\paperwidth,height=\paperheight]{YA1}}
\begin{frame}[plain]
\begin{figure}[H]
\centering
\includegraphics[scale=0.5]{bas.png} 
\end{figure}
\begin{center}
\vspace{-0.2cm}
\begin{tikzpicture}
\node[graduate,stripes=red,minimum size=2.5cm] at (0,0) {};
\end{tikzpicture}
\quad
\begin{tikzpicture}
{Large
\node [draw,cloud callout,brown,callout pointer start size=.1] { بعد و طيبة تحية };
}
\end{tikzpicture}
\quad
\begin{tikzpicture}
\node[graduate,undershirt=red,minimum size=2.5cm] at (0,0){} ;
\end{tikzpicture}
\end{center}
%\end{tikzpicture}
%\begin{flushright}
%\includegraphics[scale=0.50]{14.JPG} 
%\end{flushright}
\end{frame}
%=======================================
\setbeamertemplate{background canvas}{\includegraphics[width=\paperwidth,height=\paperheight]{w4}}
\begin{frame}[plain]\transsplitverticalout
$$\includegraphics[width=8cm, height=4cm]{0142.png}$$ 
\begin{tcolorbox}[colback=white,drop large lifted shadow,top=.5cm,bottom=.5cm]
{\begin{center}
	{\Huge\emph{{نُـرَحِبُ بـِضِيــُوفِنَــا الـكِرَام}}}
	\end{center}}
\end{tcolorbox}
\end{frame}
\rightskip\rightmargin
\setbeamertemplate{background canvas}{\includegraphics[width=\paperwidth,height=\paperheight]{web}}
\begin{frame}[plain]{
\large 
\textbf{مُذكـّرةَ تخَـرج لنيَـل شهَـادة أُستَـاذ التَّـعلِيم الثَّانَـــوي
 بِعُنــْــوَان :}
}\transglitter[duration=0.5]

\begin{tcolorbox}[colback=white,drop large lifted shadow,top=.5cm,bottom=.5cm]
\begin{center}
\begin{LARGE}
%\pause
\end{LARGE}\hfill\\
 \begin{Large}
المعادلات التفاضلية ذات الرتب الكسرية في فضاء بناخ
 \end{Large}
\end{center}
\end{tcolorbox}
\begin{small}
\begin{columns}
\begin{column}{0.5\linewidth}
{\textbf{
تحت إشراف الأستاذ :
}}
\begin{itemize}
\item[$\textcolor{red!90!black}{\bigstar}$]
بوسنة جلال
\end{itemize}
\end{column}
\begin{column}{0.3\linewidth}
{{\textbf{من إعداد الطالبين :}}}
\begin{itemize}
\item[$\textcolor{red!90!black}{\bigstar}$]
بخدة أمـــــن
~\item[$\textcolor{red!90!black}{\bigstar}$]
يفرح عابــــد
\end{itemize}
\end{column}
\end{columns}
\end{small}
\textbf{{{\emph{تناقش يوم 2018/07/02 ، من طرف لجنة المناقشة :}}}
}
\begin{small}
\begin{itemize}\RTListe
\item[$\textcolor{red}{\surd}$] 
 بوسنة جلال
 ..........
 أستاذ بالمدرسة العليا للأساتذة
 ..........
  مشرفا.
 
\item[$\textcolor{red}{\surd}$]
منصوري بوزيد
 ..........
 أستاذ بالمدرسة العليا للأساتذة
 ..........
 رئيسا.
\item[$\textcolor{red}{\surd}$]
كناف أسماء
 ..........
 أستاذة بالمدرسة العليا للأساتذة
 ..........
مناقشة.
\end{itemize}

\end{small}
\end{frame}
%===========================
\setbeamertemplate{background canvas}{\includegraphics[width=\paperwidth,height=\paperheight]{walid}}

\begin{frame}{ مخطط العرض}
\tableofcontents
\end{frame}
%[pausesubsections]
%=============================

\setbeamertemplate{background canvas}{\includegraphics[width=\paperwidth,height=\paperheight]{w4}}
\section{\textcolor{red}{مقدمـــة}}

\begin{frame}[plain]{{\Large{ \textbf{{مُقَـــدّمـــــــة}}}}}\transsplitverticalout
\begin{paperbox}{مقدمة}
 تظهر المعادلات التفاضلية ذات الرتب الكسرية بشكل تلقائي في مختلف الميادين العلمية مثل الكيمياء الكهربائية ،الطب ،الفيزياء ... 
 
 الهدف من هذه المذكرة، هو دراسة بعض أنواع المعادلات التفاضلية ذات الرتب الكسرية في فضاء بناخ ، إذ قمنا بإعطاء نتائج وجود و وحدانية الحلول لبعض المسائل بمفهوم كابوتو في فضاء بناخ ، و ذلك إعتمادا على تقنية نظرية النقطة الثابتة .

وأوجزنا هذا العمل في ثلاثة فصول كالتالــــــــي
\end{paperbox}
\end{frame}
\begin{frame}
%\begin{box4}[colback=yellow]{\textcolor{black}{الفصل الأول }}
\begin{ambox}{الفصل الأول}{magenta}
 تقديم الدالتين  Gamma و Béta و بعض التعاريف الأساسية،بالإضافة إلى نظريات النقطة الثابتة (بناخ ، شايفر ، مو نش) ثم التطرق إلى الإشتقاق و التكامل الكسريين و بعض الخواص .
\end{ambox}
\end{frame}
\begin{frame}
%\begin{box4}[colback=yellow]{\textcolor{black}{الفصل الثاني }}
\begin{ambox}{الفصل الثاني}{magenta}
 دراسة وجود و وحدانية حلول مسائل لمعادلات تفاضلية ذات رتب  كسرية  في فضاء بناخ ،
إذ تطرقنا في هذه الدراسة إلى محورين ،حيث تناولنا في:
\begin{tcolorbox}[enhanced,title=\,,
frame style tile={width=1cm}{pink_marble.png}]
{\LARGE\color{blue}{\ding{118}}}
\underline{المحور الأول :}
دراسة وجود و وحدانية الحلول للمسألة التالية:
\begin{align*}
{}^{c}D{}^{\alpha }y(t)=f\big(t,y(t)\big)\,,\,\forall t\in J=\big[0,T\big]\,,\,0<\alpha \le 1 , \\
ay(0)+by(T)=c. \hspace{3cm} \,
\end{align*}
\tcblower
{\LARGE\color{blue}{\ding{118}}}
\underline{المحور الثاني :}
دراسة وجود الحلول للمسألة التالية :
\begin{align*}
{}^c{D^\alpha }y(t) = f\big(t,y(t)\big)\,,\,t \in J = \big[ {0,T} \big],1 < \alpha  \le 2 ,\\
y(0) = y_{0},\,\,\,\,\,\,\,\,\,\,\,\,\,\,y(T) = y_{T}.\hspace{2cm} \,
\end{align*}
\end{tcolorbox}
\end{ambox}
\end{frame}
\begin{frame}
%\begin{box4}[colback=yellow]{\textcolor{black}{الفصل الثالث }}
\begin{ambox}{الفصل الثالث}{magenta}
 دراسة الإستقرار التقريبي للحل البديهي للمسألة التالية:
\begin{tcolorbox}[enhanced,title=\,,
frame style tile={width=1cm}{pink_marble.png}]
\begin{align*}
{}^CD_{{0^ + }}^\alpha x(t)& = kx(t) + f\big(t,x(t),x\big(t - \tau (t)\big)\big) \\
&+ {}^CD_{{0^ + }}^\alpha g\big(t,x\big(t - \tau (t)\big)\big)\,\,,\,t \ge 0,\\
x'(0) &= 0\,,x(t) = \phi (t)\,\,,\,t \in \big[ {{m_0},0} \big].
\end{align*}
\end{tcolorbox}
\end{ambox}
\end{frame}
\setbeamertemplate{background canvas}{\includegraphics[width=\paperwidth,height=\paperheight]{YA1}}
\section{\textcolor{black}{مفاهيــــم عامة وتعاريـــــف}}
\begin{frame}[containsverbatim]{الفصل الأول}\transsplithorizontalout[duration=0.5]

\begin{tcolorbox}[colback=white,drop large lifted shadow,top=.5cm,bottom=.5cm]
{\begin{center}
	{\Huge\ttfamily\emph{{ مفاهيــــم عامة وتعاريـــــف}}}
	\end{center}}
\end{tcolorbox}
\end{frame}
%\section{الدالة Gamma و الدالة Béta}
\setbeamertemplate{background canvas}{\includegraphics[width=\paperwidth,height=\paperheight]{im3}}
\subsection{التوابع الخاصة}
 \begin{frame}{الدالة Gamma}
\begin{box3}{ تعريف}{magenta}
{\LARGE{\ding{45}}}
من أجل كل$z\in \mathbb{C}$ حيث 
$\operatorname{Re}(z)>0$ 
 ،الدالة Gamma تعطى بالعلاقة التالية: 
\begin{center}
 \otherbox{\Gamma \left(z \right)=\int\limits_{0}^{+\infty }{{{t}^{z-1}}{{e}^{-t}}dt}}
\end{center}
  \end{box3}
  \end{frame}
  \begin{frame}{الدالة Béta}
\begin{box3}{ تعريف}{magenta}
{\LARGE{\ding{45}}}
من أجل كل $(u,v)\in {{\mathbb{C}}^{2}}$ حيث $\operatorname{Re}(u)>0$ و $\operatorname{Re}(v)>0$ ،نعرف الدالة Béta كمايلي:
\begin{center}
 \otherbox{B\big(u,v\big)=\int\limits_{0}^{1}{{{t}^{u-1}}{{(1-t)}^{v-1}}\,dt}.}
\end{center}
  \end{box3}
  \end{frame}
  \begin{frame}{ العلاقة بين الدالة Gamma و الدالة Béta}
\begin{box2}{ خاصية}{red}
{\LARGE{\ding{45}}}
من أجل كل $(u,v)\in {{\mathbb{C}}^{2}}$ حيث   $\operatorname{Re}(v)>0$  و $\operatorname{Re}(u)>0$   
،الدالتان  $\Gamma $  و $B$   تحققان العلاقة التالية:
\begin{center}
 \otherbox{B\big(u,v\big)=\dfrac{\Gamma (u)\Gamma (v)}{\Gamma (u+v)}.}
\end{center}
  \end{box2}
  \end{frame}
  \subsection{نظريات النقطة الثابة }
    \begin{frame}{ نظريات النقطة الثابتة }
\begin{mybox}[colbacktitle = green]{ نظرية بناخ}
{\LARGE{\ding{45}}}
ليكن $E$ فضاء بناخ و $A:\,E\to E$  تطبيق،
إذا كان  $A$  مقلص على  $E$ فإن  $A$ يقبل نقطة ثابتة وحيدة على  
$E$
 أي  
\begin{center}
 \otherbox{\exists !\,x\in E\,\,\,,Ax=x \,.}
\end{center}
  \end{mybox}
  \end{frame}
  \begin{frame}{ نظريات النقطة الثابتة }
   \begin{mybox}[colbacktitle = green]{ نظرية شايفر}
{\LARGE{\ding{45}}}
ليكن  $E$ فضاء بناخ و $A:\,E\to E$ تطبيق مستمر كليا ،
إذا كانت المجموعة
\begin{center}
 \otherbox{ \xi=\left. \big\{ y\in E\,,\,y=\lambda Ay\,,\,\lambda \in \big] 0,1 \big[ \big. \right\}} 
 \end{center}
 محدودة
فإن  $A$  يقبل على الأقل نقطة ثابتة في  $E$  أي 

 \tcbhighmath{\exists x\in E\,\,\,,Ax=x}
  \end{mybox}
  \end{frame}
  \begin{frame}{ نظريات النقطة الثابتة }
   \vspace{-0.3cm}
     \begin{mybox}[colbacktitle = green]{ نظرية مونش}
{\LARGE{\ding{45}}}
لتكن $D$ مجموعة جزئية محدودة ، مغلقة و محدبة من فضاء بناخ ، حيث $0\in D$ و $N$ تطبيق مستمر من $D$ نحو $D$ ، إذا كانت العلاقة
 \begin{center}
 \otherbox{ V=N\left( V \right)\cup \big\{ 0 \big\}\Rightarrow \alpha \left( V \right)=0}
\end{center}
أو 
~~~~~~~~~~~~~~~~~~~~
 \otherbox{V=\overline{conv}N\left( V \right)}   
 
   محققة من أجل كل مجموعة جزئية $V$ من $D$ ، إذن$N$ يقبل نقطة ثابتة 
حيث
$\alpha $ هو القياس غير المتراص بمفهوم kuratowski
.
  \end{mybox}
  \end{frame}
 % \section{التكامل الكرسري و الإشتقاق الكسري }
  \subsection{ التكامل و الإشتقاق الكسري }
  \begin{frame}{التكامل الكسري على مجال  $\left[ a,b \right]$ }
  \begin{box3}{ تعريف}{magenta}
 {\LARGE{\ding{45}}}
لتكن $f\in {{L}^{1}}\big( \left[ a,b \right] \big)$   و $\alpha \in {{\mathbb{R}}_{+}}$ ،
تكامل الدالة $f$  من رتبة كسرية $\alpha$ بمفهوم
 \LR { Riemman liouvile }
 يرمز له بـ ${I}_{a}^{\alpha }$  و يعطى بالعبارة التالية:

 \otherbox{{I}_{a}^{\alpha }f(x)=\dfrac{1}{\Gamma \left( \alpha  \right)}\int\limits_{a}^{x}{{{(x-t)}^{\alpha -1}}f(t)dt\,,\,x\in \big[ a,b \big]}.}
 
إذا كان $\alpha =n\,\in \mathbb{N}$  العلاقة تصبح 
 
 \otherbox{I_{a}^{n}f(x)= \dfrac{1}{(n-1)!}\int\limits_{a}^{x}{{{(x-t)}^{n-1}}f(t)dt\,,\,x>a}.} 
  \end{box3}
  \end{frame}
  %\subsection{الإشتقاق الكسري }
  \begin{frame}{{الإشتقاق الكسري من رتبة $\alpha >0$ } }
  \begin{box3}{ تعريف}{magenta}
 {\LARGE{\ding{45}}}
لتكن $f\in {{L}^{1}}\big(\left[ a,b \right] \big)$ ، 
يعرف المشتق الكسري ذو رتبة $\alpha $ ،حيث $n-1\le \alpha \le n$ بمفهوم  Riemman-liouvile  بالعبارة التالية 
:
\begin{empheq}[box=\tcbhighmath]{align}
\nonumber  {}^{R}D_{a}^{\alpha }f(x)&= \dfrac{1}{\Gamma \left( n-\alpha  \right)}{{\left( \dfrac{d}{dx} \right)}^{n}}\int\limits_{a}^{x}{{(x-t)}^{n-\alpha -1}}f(t)\,dt \\ 
 \nonumber &={{\left( \dfrac{d}{dx} \right)}^{n}}\left({I}_{a}^{n-\alpha }f(x) \right).
\end{empheq}
% \otherbox{I_{a}^{n}f(x)=\frac{1}{(n-1)!}\int\limits_{a}^{x}{{{(x-t)}^{n-1}}f(t)dt\,,\,x>a}.} 
 حيث $n=\left[ \alpha  \right]+1$  ، $\left[ \alpha  \right]$ الجزء الصحيح للعدد $\alpha $ 
  \end{box3}
  \end{frame}
  \begin{frame}{{الإشتقاق الكسري من رتبة $\alpha >0$ }}
    \begin{box3}{ تعريف}{magenta}
  {\LARGE{\ding{45}}}
لتكن $f$ دالة  حيث  $f\in {{AC}^{n}}\bigg( \big[ a,b \big],\mathbb{R} \bigg)$،
المشتق الكسري ذو الرتبة $\alpha >0$  للدلة $f$ بمفهوم كابوتو معرف بـ 

\vspace{0.5cm}
 \otherbox{{}^{C}D_{a}^{\alpha }\,f(x)=\dfrac{1}{\Gamma \left( n-\alpha  \right)}\int\limits_{a}^{x}{{{\left( x-t \right)}^{n-\alpha -1}}\,{{f}^{\left( n \right)}}\left( t \right)}\,dt .} 

\vspace{0.5cm}
حيث   $n=\left[ \alpha  \right]+1$  ، $\left[ \alpha  \right]$ الجزء الصحيح للعدد $\alpha $  
.
 \end{box3}
  \end{frame}
    \begin{frame}{العلاقة بين مشتق ريمان-ليوفيل و مشتق كابوتو}
      \begin{mybox}[colbacktitle = green]{ نظرية }
    {\LARGE{\ding{45}}}
لتكن
 $f\in {{C}^{n}}\big( \left[ a,b \right] \big)$
 و
 $\alpha > 0 $
  حيث 
 $n-1\le \alpha <n\,\,\,\,\,\,\,, n\in {{\mathbb{N}}^{*}}$ 
،
\vspace{0.5cm}
العلاقة  بين  مؤثر كابوتو و مؤثر ريمان – ليوفيل تعطى بـ
 
 \otherbox{{}^{C}D_{a}^{\alpha }\,f(t)={}^{R}D_{a}^{\alpha }\,f(t)-\sum\limits_{k=0}^{n-1}{\frac{{{\left( t-k \right)}^{k-\alpha }}}{\Gamma \left( k+1-\alpha  \right)}{{f}^{\left( k \right)}}\left( a \right)} .} 
    \end{mybox}
    \end{frame}
    \begin{frame}
\begin{box2}{\textcolor{white}{توطئة 1.2}}{blue}
     {\LARGE{\ding{45}}}
 لتكن الدالة~
$h:\big[ 0,+\infty  \big[ \rightarrow \mathbb{R}$~،
من أجل $\alpha >0$  لدينا 
\begin{empheq}[box=\tcbhighmath]{align}
\nonumber  & {{I}^{\alpha }}\left( ^{C}D_{a}^{\alpha }h\left( t \right) \right)={{c}_{0}}+{{c}_{1}}t+{{c}_{2}}{{t}^{2}}+...+{{c}_{n-1}}{{t}^{n-1}}+h\left( t \right) ,\\ 
\nonumber & {{c}_{k}}\in \mathbb{R}\,\,,\,k=0,1,...,n-1\,\,\,\,,\,\,n=\left[ \alpha  \right]+1
 . 
\end{empheq}
\end{box2}
    \end{frame}
\setbeamertemplate{background canvas}{\includegraphics[width=\paperwidth,height=\paperheight]{YA1}}
\section{\textcolor{black}{المعادلات التفاضلية ذات الرتب الكسرية في فضاء بناخ}}
\begin{frame}[containsverbatim]{الفصل الثاني }\transsplithorizontalout[duration=0.5]

\begin{tcolorbox}[colback=white,drop large lifted shadow,top=.5cm,bottom=.5cm]
{\begin{center}
	{\Huge\ttfamily\emph{{ المعادلات التفاضلية ذات الرتب الكسرية في فضاء بناخ}}}
	\end{center}}
\end{tcolorbox}
\end{frame}
\setbeamertemplate{background canvas}{\includegraphics[width=\paperwidth,height=\paperheight]{im4}}
\subsection{ المعادلات التفاضلية ذات الرتب الكسرية $0<\alpha \le 1$ }
\begin{frame}{ المعادلات التفاضلية ذات الرتب الكسرية $0<\alpha \le 1$ }
\vspace{-1.3cm}
%\begin{box4}[colback=yellow]{\textcolor{black}{نضع المســــــــــــــــــــألة }}
\begin{ambox}{نضع المســــــــــــــــــــألة }{magenta}
\begin{align}
\begin{cases}
 {}^{c}D{}^{\alpha }y(t)=f\big(t,y(t)\big)\,,\,\forall t\in J=\big[0,T\big]\,,\,0<\alpha \le 1 , \\
 ay(0)+by(T)=c. 
\end{cases}
\end{align}
\end{ambox}
\end{frame}
\begin{frame}
\vspace{-1cm}
\begin{box2}{\textcolor{white}{توطئة 1.2}}{blue}
%\begin{block}{توطئة}
{\LARGE{\ding{45}}}
ليكن
$0<\alpha\le 1$
و لتكن الدالة 
$h:\big[0,T\big]\to E$
مستمرة .
 نقول أن الدالة $y$ هي حل للمعادلة التكاملية التالية 
\begin{align*}
 \nonumber y(t)&=\dfrac{1}{\Gamma (\alpha )}\int\limits_{0}^{t}{{{(t-s)}^{\alpha -1}}}h(s)\,ds\\
\nonumber &-\frac{1}{a+b}\left[ \dfrac{b}{\Gamma (\alpha )}\int\limits_{0}^{T}{(T-s}{{)}^{\alpha -1}}h(s)\,ds-c \right]. 
 \end{align*}
%\end{block}
إذا و فقط إذا كان $y$ حل للمسألة 
\begin{align*}
 & {}^{C}{{D}^{\alpha }}y(t)=h(t)\,,t\in J \\ 
&ay(0)+by(t)=c.\hspace{3cm}
\end{align*}
\end{box2}
\end{frame}
\begin{frame}{نظرية}
\begin{mybox}[colbacktitle = green]{نظرية 1.2}
{\LARGE{\ding{45}}}
نعتبر أن الفرضية التالية محققة :
\\
$\big(H_{1}$
يوجد عدد ثابت $k>0$
بحيث
$$\forall t\in J\,,\,\forall u,\tilde{u} \,\in E\,,{{\big\| f\big(t,u\big)-f\big(t,\tilde{u} \big) \big\|}_{E}}\le \,k{{\big\| u-\tilde{u}\big\|}_{E}}.$$
إذا كان 
\begin{align*} 
\dfrac{k{{T}^{\alpha }}\left( 1+\dfrac{\left| a \right|}{\left| a+b \right|} \right)}{\Gamma (\alpha +1)}\,\,<1,
\end{align*} 
فإن المسألة (1)
تقبل حل وحيد على المجال
$\big[0,T\big]$
.
\end{mybox}
\end{frame}
\begin{frame}{الإثبات}
\begin{tcolorbox}[enhanced,arc=3mm,boxrule=1.5mm,boxsep=1.5mm,
colback=yellow!20!white,
colframe=blue,
borderline={1mm}{1mm}{yellow},
borderline={1mm}{2mm}{green} ]
نعتبر المؤثر 
$F:C\big(J,E\big) \to \,\,C\big(J,E\big)$
المعرف بـ 
\begin{align*}
F\big(y\big)(t) &= \dfrac{1}{{\Gamma (\alpha )}}\int\limits_0^t {\big(t - s\big)^{\alpha -1}} f\big(s,y(s)\big)\,ds \\
&- \dfrac{1}{{\big(a + b\big)}}\left[\dfrac{b}{{\Gamma (\alpha )}}\int\limits_0^T {\big(T - s\big)}^{\alpha  - 1}f\big(s,y(s)\big)ds - c\right].
\end{align*}

نبين أن $F$ هو تقلص 

و بماأن $F$ هو تقلص إذن حسب نظرية النقطة الثابتة لـ Banach 
فإن
 $F$
 يقبل نقطة ثابتة وحيدة ، و هي حل للمسألة(1)  
\end{tcolorbox}
\end{frame}
\begin{frame}{نظرية}
\begin{mybox}[colbacktitle = green]{نظرية 2.2}
{\LARGE{\ding{45}}}
نعتبر أن الفرضيات التالية محققة :

$\big(H_{2}$
الدالة 
$f:\big[0,T\big]\times E\to \,E$ 
مستمرة ،

$\big(H_{3}$
يوجد عدد ثابت 
$ M>0 $
بحيث 
$${{\big\| f(t,u) \big\|}_{E}}\le M\,\,,\,\forall t\in J\,,\,\forall u\in E,$$ 
إذن المسألة (1)
تقبل حل واحد على الأقل في $J$.
\end{mybox}
\end{frame}
\begin{frame}{الإثبات}
\begin{tcolorbox}[enhanced,arc=3mm,boxrule=1.5mm,boxsep=1.5mm,
colback=yellow!20!white,
colframe=blue,
borderline={1mm}{1mm}{yellow},
borderline={1mm}{2mm}{green} ]
نستخدم نظرية النقطة الثابتة لـ 
Schaefer
لإثبات أن المؤثر $F$ المعرف سابقا  يقبل  على الأقل نقطة ثابتة في $J$.

يكفي إثبات
أن المؤثر 
$F$
مستمر كليا

و المجموعة 
$$\xi  = \bigg\{y \in C(J,E)\,;\,y = \lambda F(y)\,;\,\forall \lambda  \in \big]0,1\big[\,\, \bigg\}$$
محدودة .

و بالتالي حسب نظرية النقطة الثابتة لـ
Schaefer
نستنتج أن 
$F$
يقبل نقطة ثابتة في $J$و هي حل للمسألة(1) 
.
\end{tcolorbox}
\end{frame}
\subsection{ المعادلات التفاضلية ذات الرتب الكسرية $1<\alpha \le 2$ }
\begin{frame}
%\begin{box4}[colback=yellow]{\textcolor{black}{نضع المســــــــــــــــــــألة }}
\begin{ambox}{نضع المســــــــــــــــــــألة }{magenta}
\begin{align}
\begin{cases}
{}^C{D^\alpha }y(t) = f\big(t,y(t)\big)\,,\,t \in J = \big[ {0,T} \big],1 < \alpha  \le 2 ,\\
y(0) = y_{0}\;,\;y(T) = y_{T}.
\end{cases}
\end{align}
\end{ambox}
\end{frame}
\begin{frame}{توطئة}
\begin{box2}{\textcolor{white}{توطئة 3.2}}{blue}
{\LARGE{\ding{45}}}
ليكن
$1 < \alpha  \le 2$
و
$h:J \to E$
دالة مستمرة.
المسألة الخطية
\begin{align*}
{}^CD^{\alpha }y(t) = h(t)\,,\,t \in J ,\\
y(0) = y_{0}\,\,,\,\,\,y(T) = y_{T}.
\end{align*}
تقبل حل وحيد يعطى بالعلاقة 
\begin{align*}
y(t) = g(t) + \int\limits_0^T {G\big(T,s\big)} h(s)ds,
\end{align*}
\end{box2}
\end{frame}
\begin{frame}
\begin{box2}{}{blue}
حيث
$$G(t,s) = \begin{cases}\dfrac{1}{{\Gamma (\alpha )}} 
\bigg(\big(t - s\big)^{\alpha  - 1} - \dfrac{t}{T}{\big(T - s\big)^{\alpha  - 1}}\bigg)\,\,\;\;si\,\,0 \le s \le t ,\\
 \dfrac{1}{{\Gamma \big(\alpha \big)}}\bigg(- \dfrac{t}{T}{\big(T - s\big)^{\alpha  - 1}}\bigg)\,\,\,\,\,\,\,\,\,\,\,\,\,\,\,\,\,\,\,\;\;\;si\,\,\,t \le s \le T.
\end{cases}$$
و~~~~~~~~~~~~~~~~~~~~~~~~~~
$g(t) = \left( {1 - \dfrac{t}{T}} \right){y_0} + \dfrac{t}{T}{y_T}.$
\end{box2}
\end{frame}

\begin{frame}{توطئة}
\vspace{-0.5cm}
\begin{box2}{\textcolor{white}{توطئة 4.2}}{blue}
{\LARGE{\ding{45}}}  
المسألة(2) تقبل حل
$y$ 
إذا وفقط إذا كان$y$ 
يحقق المعادلة التكاملية
 
~~~~~~~~~~~~~~~~~~~~~~~~~
$y(t) = g(t) + {\displaystyle \int\limits_0^T} {G(T,s)} f\big(s,y(s)\big)ds,$

حيث ~~~~~~~~
$g(t) = \left( {1 - \dfrac{t}{T}} \right){y_0} + \dfrac{t}{T}{y_T},$
و
$$G(t,s) = \begin{cases}\dfrac{1}{{\Gamma (\alpha )}}\left( 
{\big(t - s\big)^{\alpha  - 1}} - \dfrac{t}{T}{\big(T - s\big)^{\alpha  - 1}}\right)\,\,\,si\,\,0 \le s \le t,\\
\dfrac{1}{{\Gamma (\alpha )}} \left(- \dfrac{t}{T}{\big(T - s\big)^{\alpha  - 1}}\right)\,\,\,\,\,\,\,\,\,\,\,\,\,\,\,\,\,\,\,\,\,\,\,\,\,\,\,\,\,si\,\,t \le s \le T.
\end{cases}$$
\end{box2}
\end{frame}
\begin{frame}{نظرية}
\begin{mybox}[colbacktitle = green]{نظرية 2.3}
{\LARGE{\ding{45}}}
نعتبر أن الفرضيات
$\big({H_1}\big)$
-
$\big({H_3\big)}$
محققة.
إذا كان 
\begin{align}
{G^*}{\displaystyle \int\limits_0^T} {p(s)ds < 1}ة
\end{align}
إذن المسألة (2) تقبل على الأقل حل في $J$.
\end{mybox}
\end{frame}
\begin{frame}{الإثبات}
\begin{tcolorbox}[enhanced,arc=3mm,boxrule=1.5mm,boxsep=1.5mm,
colback=yellow!20!white,
colframe=blue,
borderline={1mm}{1mm}{yellow},
borderline={1mm}{2mm}{green} ]
نحول المسألة (2)إلى مسألة نقطة ثابتة.
نعتبر المؤثر

~~~~~~~~~~~~~
$N:C\big(J,E\big) \to C\big(J,E\big)$
المعرف بـ 

~~~~~~~~~~~~
$N\big(y\big)(t) = g(t) + {\displaystyle \int\limits_0^T} {G(t,s)} f\big(s,y(s)\big)ds,$

من 
التوطئة 3.2
النقاط الثابتة للمؤثر
$N$
هي عبارة عن حل للمسألة (2)
.ليكن
\begin{align*}
R > \dfrac{{{g^*}}}{{1 - {G^*}{  \int\limits_0^T }{p(s)ds} }},
\end{align*}
\end{tcolorbox}
\end{frame}
\begin{frame}
\begin{tcolorbox}[enhanced,arc=3mm,boxrule=1.5mm,boxsep=1.5mm,
colback=yellow!20!white,
colframe=blue,
borderline={1mm}{1mm}{yellow},
borderline={1mm}{2mm}{green} ]
و المجموعة ~~~~~~~~
$D_{R}=\bigg\{ y \in C\big( J,E\big):{\big\|y\big\|}_{\infty} \le R\bigg\} .$
عبارة عن مجموعة جزئية مغلقة ،محدودة و محدبة .نبين أن المؤثر
$N$
يحقق شروط نظرية النقطة الثابتة لـ 
Mönch 
.
\end{tcolorbox}
\end{frame}
%%%%%%%%%%%%%%%%%%%%%%%%%%%%%%%%%%%%%%%%%%%%%%%%%%%%%%
\setbeamertemplate{background canvas}{\includegraphics[width=\paperwidth,height=\paperheight]{im4}}
\section{\textcolor{black}{ الإستقرار التقريبي بوجود تأخير للمعادلات التفاضلية ذات الرتب الكسرية لا خطية }}
\begin{frame}[containsverbatim]{الفصل الثالث  }\transsplithorizontalout[duration=0.5]

\begin{tcolorbox}[colback=white,drop large lifted shadow,top=.5cm,bottom=.5cm]
{\begin{center}
	{\Huge\ttfamily\emph{{ الإستقرار التقريبي بوجود تأخير للمعادلات التفاضلية ذات الرتب الكسرية لا خطية}}}
	\end{center}}
\end{tcolorbox}
\end{frame}
\begin{frame}
%\begin{box4}[colback=yellow]{\textcolor{black}{نضع المســــــــــــــــــــألة }}
\begin{ambox}{نضع المســــــــــــــــــــألة }{magenta}
\begin{align}
\begin{cases}
{}^CD_{{0^ + }}^\alpha x(t) = kx(t) + f\big(t,x(t),x\big(t - \tau (t)\big)\big)\\
 + {}^CD_{{0^ + }}^\alpha g\big(t,x\big(t - \tau (t)\big)\big)\,\,,\,t \ge 0,\\
x'(0) = 0\,,x(t) = \phi (t)\,\,,\,t \in \big[ {{m_0},0} \big],
\end{cases}
\end{align}
حيث 
$1 < \alpha  < 2$
و $k$ ثابت حقيقي معطى ،
 نرمز لحل  المسألة 

~~~~~~~~~~~~~~~~~~~~~~
 بـ
 $x\big(t,\phi,0\big)$
 .
 \end{ambox}
\end{frame}
\subsection{درسة إستقرار الحل البديهي}
\begin{frame}
\begin{tcolorbox}[enhanced,colback=yellow!10!white,boxrule=0pt,frame hidden,
borderline north={1mm}{-2mm}{red},
borderline south={1mm}{-2mm}{blue},
borderline west={1mm}{-2mm}{green},
borderline east={1mm}{-2mm}{yellow}]
{\LARGE\ding{45}}
  لتبيان الإستقرار التقريبي للحل البديهي نقوم بتحويل المسألة 
  إلى معادلة تكاملية ثم نستخدم نظرية النقطة الثابتة لبناخ لتبيان وجود وحدانية الحل .

نعتبر  الفضاء التالي 
$$E = \left\{ {x \in C\bigg( {\big[{m_0}, + \infty \big[} \bigg):\mathop {\sup }\limits_{t \ge {m_0}} \big\{ h(t)\left| {x(t)} \right|\big\}  < +\infty } \right\}.$$
و المجموعة 
$$\xi (\varepsilon ) = \bigg\{ {x \in E:\big\| x \big\| \le \varepsilon ,x(t) = \phi (t),t \in \big[{m_0},0\big]}\;,\varepsilon  > 0 \bigg\}$$
\end{tcolorbox}
\end{frame}
\begin{frame}
\begin{mybox}[colbacktitle = green]{ تعريف}
 {\LARGE{\ding{45}}}
نقول عن حل بديهي 
$x=0$
للمسألة
(7)
أنه

{\large{\ding{182}}}
 مستقر في فضاء بناخ 
$E$
إذا تحقق مايلي 
$$\forall \varepsilon  > 0\,,\,\exists \,\,\delta  = \delta (\varepsilon ) > 0\,,\,\left| {\phi (t)} \right| \le \delta ,$$
فإن
$x(t)=x\big(t,\phi,0\big)$
موجود من أجل كل 
$t \ge {m_0}$
و يحقق 
$\big\| x \big\| \le \varepsilon$
.

{\large{\ding{183}}}
مستقر تقريبيا، إذا كان مستقر في فضاء بناخ 
$E_{\star}$
ويوجد 
$\delta  > 0$
، إذا كان
$\left| {\phi (t)} \right| \le \delta $
فإن 
$\mathop {\lim }\limits_{t \to  + \infty } x(t) = 0$
.
\end{mybox}
\end{frame}
\begin{frame}
\begin{box3}{توطئة 1.3}{magenta}
\label{tw3}
 {\LARGE{\ding{45}}}
ليكن 
$r \in C\big( \,{[{m_0}, + \infty [}\,\big)$
.نقول عن الدالة 
$x \in C\big( \,{[{m_0}, + \infty [}\, \big)$
حل للمسألة 
\begin{align}
\begin{cases}
{}^{C}D_{{0^{+}}}^\alpha x(t) = r(t)\,,\,\,t \in {\mathbb{R}^{+}}\,\,\,\,, \,1 < \alpha  < 2,\\
x'(0) = 0\,,\,x(t) = \phi (t)\,\,\,,\,t \in \big[{m_0},0\big],
\end{cases}
\end{align}
إذا و فقط إذاكان
$x$ 
حل للمسألة 
\begin{align}
\begin{cases}
x'(t) = I_{{0^ + }}^{\alpha  - 1}r(t)\,\,,\,t \in {\mathbb{R}^{+}},\\
\,\,\,x(t) = \phi (t)\,\,,\,\,t \in \big[{m_0},0\big].
\end{cases}
\end{align}
\end{box3}
\end{frame}
\begin{frame}
\vspace{-0.6cm}
\begin{box3}{ توطئة 2.3}{magenta}
 {\LARGE{\ding{45}}}
إذا كان 
$R \in C\big(\, {\big[0, + \infty \big[} \,\big)$
يحقق المعادلة التكاملية 
\begin{align}
\label{4.3}
R(t) = 1 + k\,I_{{0^ + }}^\alpha R(t),
\end{align}
إذن
$x \in C\big( {\big[m_0, + \infty \big[} \,\big)$
حل للمسالة (4)
إذا وفقط إذا كان 
\begin{align*}
 &x(t) = R(t)\phi (0) - g\big(0,\phi \big( - \tau (0)\big)\big){\displaystyle \int\limits_0^t }R(s)ds\\
 & +{\displaystyle \int\limits_0^t }{R(t - s)g\big(s,x\big(s - \tau (s)\big)\big)ds}  \\
&+ \dfrac{1}{{\Gamma (\alpha  - 1)}}\,{\displaystyle\int\limits_0^t} {R(t - s){\displaystyle \int\limits_0^s }{{{(s - u)}^{\alpha  - 2}}\,f\big(u,x(u),x} } \big(u - \tau (u)\big)\big)\,du\,ds.\,~~~~~~~~
\end{align*}
\end{box3}
\end{frame}
\begin{frame}
\begin{tcolorbox}[enhanced,colback=yellow!10!white,boxrule=0pt,frame hidden,
borderline north={1mm}{-2mm}{red},
borderline south={1mm}{-2mm}{blue},
borderline west={1mm}{-2mm}{green},
borderline east={1mm}{-2mm}{yellow}]
نعرف المؤثر 
$F:E \to E$
كالتالي
\begin{align*}
\hspace{-0.5cm}
\big(Fx\big)(t)=
\begin{cases}
\phi(t) \hspace{2cm} ,\,\,t\in \big[m_0 ,0\big],\\
{\displaystyle\int\limits_0^t }\bigg[ R(t - s)g\big(s,x\big(s - \tau (s)\big)\big) + k(t - s)f\big(s,x(s)\\
,x\big(s - \tau (s)\big)\big) \bigg]  \,ds\\
 + R(t)\phi (0) - g\big(0,\phi \big( - \tau (0)\big)\big){\displaystyle\int\limits_0^t} {R(s)ds} 
\,\,\,\,\,\,\,, t>0,
\end{cases}
\end{align*}
حيث  
$$k(t)=\dfrac{1}{\Gamma(\alpha-1)}\int\limits_{0}^{t} {\big(t-u\big)}^{\alpha-2} R(u)du.$$
\end{tcolorbox}
\end{frame}
\begin{frame}
\begin{tcolorbox}[enhanced,colback=yellow!10!white,boxrule=0pt,frame hidden,
borderline north={1mm}{-2mm}{red},
borderline south={1mm}{-2mm}{blue},
borderline west={1mm}{-2mm}{green},
borderline east={1mm}{-2mm}{yellow}]
{\LARGE\ding{45}}
نعتبر أن الفرضيات التالية محققة :

{\large{$\big(H_1$}}
يوجد ثابت 
$M_1 >0$
بحيث
$$\sup\limits_{t\ge 0}{\displaystyle\int\limits_{0}^{t}} h(t) \big|R(s)\big|ds \le M_1.$$
{\large{$\big(H_2$}}
يوجد ثابت 
$v\in \big]0,1\big[$
بحيث 
$$\mathop {\sup }\limits_{t \ge 0} \dfrac{{\big| k \big|}}{{\Gamma (\alpha )}}{\displaystyle\int\limits_0^t} {{{(t - s)}^{\alpha  - 2}}} h(t)\,{h^{ - 1}}(s)ds = v.$$
\end{tcolorbox}
\end{frame}
\begin{frame}
\begin{tcolorbox}[enhanced,colback=yellow!10!white,boxrule=0pt,frame hidden,
borderline north={1mm}{-2mm}{red},
borderline south={1mm}{-2mm}{blue},
borderline west={1mm}{-2mm}{green},
borderline east={1mm}{-2mm}{yellow}]
{\large{$(H_3$}}
$f$
و
$g$
دالتين مستمرتين و 
$f\big(t,0,0\big)=g\big(t,0\big)=0$
.ويوجد كذلك ثابتين
$\beta\in \big]0,1\big[$
و 
$ \ell >0$
ودوال مستمرة 
${L_{{f_1}}}\,,\,{L_{{f_2}}}\,,\,{L_{g\,}}:\mathbb{R}^{+}\to\mathbb{R}^{+}$

بحيث
\begin{align*}
\hspace{-0.3cm}
\mathop {\sup }\limits_{t \ge 0} \,{\displaystyle\int\limits_0^t} {h(t)\bigg( {\big| {R(t - s)} \big|{L_g}(s) + \big| {k(t - s)} \big|\bigg[{L_{{f_1}}}(s) + {L_{{f_2}}}(s)\bigg]} \bigg)} \,ds\, \le \beta , \\
\big| {g(t,x) - g(t,y)} \big| \le h(t){L_g}(t)\big| {x - y} \big|,~~~~~~~~~~~~\\
\begin{array}{l}
\bigg| {f\big(t,x,y\big) - f\big(t,z,w\big)} \bigg| \le h(t)\bigg( {{L_{{f_1}}}(t)\left| {x - z} \right| + {L_{{f_2}}}(t)\left| {y - w} \right|} \bigg)\,,\\
\,\,\,\,\,\,\,\,\,\,\,\,~~~~~~~~~~~~~~~~\forall \,t \ge 0\,,\,\left| x \right|,\left| y \right|,\left| z \right|,\left| w \right| \le l.
\end{array}
\end{align*}
\end{tcolorbox}
\end{frame}
\begin{frame}
\begin{mybox}[colbacktitle = green]{نظرية 1.3  }
{\LARGE{\ding{45}}}
نعتبر أن الفرضيات
$~\big(H_1\big)$-$\big(H_3\big)$
محققة،إذن الحل البديهي 
$x=0$
للمسألة 
(4)
مستقر في فضاء بناخ 
$E$
.
\end{mybox}
\end{frame}
\begin{frame}{الإثبات}
\begin{tcolorbox}[enhanced,arc=3mm,boxrule=1.5mm,boxsep=1.5mm,
colback=yellow!20!white,
colframe=blue,
borderline={1mm}{1mm}{yellow},
borderline={1mm}{2mm}{green} ]
لتبيان أن الحل البديهي $x=0$ للمسألة مستقر في فضاء بناخ $E$ يكفي تبيان أن:
المسألة تقبل حل وحيد في 
$\xi\left( \varepsilon\right)$

واضح أن 
$\xi\left( \varepsilon\right) \subseteq E$
مجموعة جزئية مغلقة و محدبة  و المؤثر $F$ مستمر 

نبين أن
$F:\xi\left( \varepsilon\right) \rightarrow \xi\left( \varepsilon\right)$
و بعد ذلك نبين أن $F$ هو تقلص و بالتالي حسب نظريةنقطة ثابتة لبناخ نستنتج أن $F$ يقبل نقطة ثابتة وحيدة 
$x\in \xi\left( \varepsilon\right)$
 وهي حل للمسألة (4) 
 \end{tcolorbox}
 \end{frame}
 \begin{frame}{الإثبات ...}
 \begin{tcolorbox}[enhanced,arc=3mm,boxrule=1.5mm,boxsep=1.5mm,
colback=yellow!20!white,
colframe=blue,
borderline={1mm}{1mm}{yellow},
borderline={1mm}{2mm}{green} ]
 ثانيا من أجل أي حل للمسألة (4) و من أجل كل 
 $\varepsilon>0$
يوجد 
$$\delta  = \dfrac{{1 - \beta }}{{{M_2} + {M_1}{L_g}(0)}}\varepsilon ,$$
يحقق
$\big| {\phi (t)} \big| \le \delta $
.

فإن الحل الوحيد $x$ للمسألة  يحقق 
$\| x\| \le \varepsilon$

و بالتالي الحل البديهي  للمسألة مستقر  في الفضاء $E$
\end{tcolorbox}
\end{frame}

\subsection{دراسة الإستقرارالتقريبي للحل البديهي}
\begin{frame}{دراسة الإستقرارالتقريبي للحل البديهي}
\begin{tcolorbox}[boxrule=2mm]
{\LARGE\ding{45}}
نعرف فضاءا بناخ 
$E_{\star}$
و
$E_{\beta}$
$$E_{\star} = \bigg\{ {x \in C\big( {\big[{m_0}, + \infty \big[}\, \big)\,\,:\,\mathop {\lim }\limits_{t \to  + \infty } x(t) = 0} \bigg\},$$
و
\begin{align*}
{E_\beta } = \bigg\{ x \in C\big( {\big[{m_0}, + \infty \big[} \, \big)&:\,\mathop {\lim }\limits_{t \to  + \infty } {t^{\alpha  - \beta }}x(t) = 0\\
&,0 < \beta  < 1 < \alpha  < 2 \bigg\},
\end{align*}
\end{tcolorbox}
\end{frame}
\begin{frame}
\begin{tcolorbox}[boxrule=2mm]
المزودين بالناظمين 
${\big\| x \big\|_{E_{\star}}}$
و
${\big\| x \big\|_{{E_\beta }}}$
التاليين:

${\big\| x \big\|}_{E_ {\star}}  = \mathop {\sup }\limits_{t \ge {m_0}} \big\{{\big| {x(t)} \big|} \big\},$,
${\big\| x \big\|_{{E_\beta }}} = \mathop {\sup\limits_{t \ge {m_0}} \big\{ {{t^{\alpha  - \beta }}\left| {x(t)} \right|} \big\}} $

~~~~~~~~~~~~~~~~~~~
واضح أن
${E_\beta } \subseteq {E_\star } \subseteq E$
.

نضع
${\xi _\star } = \bigg\{ {x \in {E_\star }:\,x(t) = \phi (t)\,,\,t \in \,\, \big[ {{m_0},0} \big]} \bigg\}.$
\end{tcolorbox}
\end{frame}
\begin{frame}
\begin{mybox}[colbacktitle = green]{نظرية 2.3  }
 {\LARGE{\ding{45}}}
نفرض أن الشرط 
$\big(H_3\big)$
من 
(\textbf{نظرية 1.3})
محقق مع أخذ

~~~~~~~~~~~~
$h(t)=1$
،
$1 <\alpha < 1 + \beta $

$\mathop {\lim }\limits_{t \to  + \infty }{\displaystyle \int\limits_0^t} {\bigg( {\left| {R(t - s)} \big|{L_g}(s) + \big| {k(t - s)} \right| \bigg[{L_{{f_1}}}(s) + {L_{{f_2}}}(s)\bigg]} \bigg)}ds $
\centerline{$= 0$}

~~~~~~~~~~~~~
$\mathop {\lim }\limits_{t \to  + \infty } \,{t^{\alpha  - \beta }}R(t) = 0$
و
$\mathop {\lim }\limits_{t \to  + \infty } \,{\displaystyle \int\limits_0^t}{R(s)} ds = 0$

إذن الحل البديهي للمسألة 
مستقر تقريبيا
 في الفضاء 
$E_\star$
.
\end{mybox}
\end{frame}
\begin{frame}{الإثبات }
\begin{tcolorbox}[enhanced,arc=3mm,boxrule=1.5mm,boxsep=1.5mm,
colback=yellow!20!white,
colframe=blue,
borderline={1mm}{1mm}{yellow},
borderline={1mm}{2mm}{green} ]
بكل سهولة يمكن إثبات أنه يوجد حل على الأقل للمسألة
في الفضاء
$E_\beta$ 
و نبين أن الحل البديهي للمسألة (4) مستقر في الفضاء  $E_\star$ 

إضافة إلى ذلك ،نبين أن 
$F:{\xi _\star } \to {\xi _\star }$

و بالتالي الحل البديهي للمسألة (4) مستقر تقريبيا في فضاء 
$E_{\star}$ 
\end{tcolorbox}
\end{frame}
\title{اكتب موضوع}
\setbeamertemplate{background canvas}{\includegraphics[width=\paperwidth,height=\paperheight]{im3}}
\section{\textcolor{red}{الخاتمة}} 
 \begin{frame}{الخاتمة}
\begin{bekhadda}[title=\textcolor{red}{الخاتمة}]
  وفي الأخير كانت مذكرتنا عبارة عن رحلة جاهدة للإرتقاء بدرجات العقل
ومعارج الأفكار ولا ندعي فيه الكمال ولكن عذرنا أننا بذلنا فيه قصارى
جهدنا.

فإن أصبنا فمن الله تعالى وإن أخطأنا فذلك من أنفسنا و الشيطان فلنا شرف المحاولة و التعلم. 
\end{bekhadda}
\end{frame}

\setbeamertemplate{background canvas}{\includegraphics[width=\paperwidth,height=\paperheight]{072}}
\begin{frame}[plain]
\begin{center}
\begin{tikzpicture}
\node[graduate,minimum size=2.5cm] at (0,0) {};
\end{tikzpicture}
\quad
\begin{tikzpicture}
\path[fill=yellow,draw=yellow!75!red] (0,0) circle (1cm);
\fill[red] (45:5mm) circle (1mm);
\fill[red] (135:5mm) circle (1mm);
\draw[line width=1mm,red] (215:5mm) arc (215:325:5mm);
\end{tikzpicture}
\quad
\begin{tikzpicture}
\node[graduate,minimum size=2.5cm] at (0,0){} ;
\end{tikzpicture}
\end{center}
\begin{tcolorbox}[colback=white,drop large lifted shadow,top=.2cm,bottom=.5cm]
{\begin{center}
	{\Huge\emph{{{شُكرًا علَى حُسْن الإِصْغَاء وَالمُتَابَعَة}}}}
	\end{center}}
\end{tcolorbox}
\end{frame}
\end{document}









