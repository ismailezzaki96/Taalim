% pour genree un pdf: faire
% pdflatex exemple.tex
\documentclass{article}

%% Paquets LateX utiles

\usepackage[utf8]{inputenc} 		% encodage des caracteres utilise (pour les caracteres accentues) -- non utilise ici.
%\usepackage[latin1]{inputenc} 		% autre encodage
\usepackage[french]{babel}		% pour une mise en forme "francaise"
\usepackage{amsmath,amssymb,amsthm}	% pour les maths
\usepackage{graphicx}			% pour inclure des graphiques
\usepackage{hyperref}			% si vous souhaitez que les references soient des hyperliens
\usepackage{color}			% pour ajouter des couleurs dans vos textes


\def \R {\mathbb R}					% definit un nouveau mot cle LateX. Ici, \R designera l'ensemble des reels
\newcommand \fonctionsContinues[2] {C^0(#1,#2)}		% Une nouvelle commande avec un argument


\title{Exemple LateX}
\author{Olivier Pantz\footnote{Universit\'e de Nice}}	% pour les accents, on peut soit preciser l'encodage et utiliser des caracteres accentues, soit utilise \'e pour un e accent aigu, \`e pour un e accent grave, etc...

\begin{document}
\maketitle						% Genere le titre

\tableofcontents					% si on veut une table des matieres
\listoffigures						% si on veut la liste des figures
%\listoftables						% si on veut la liste des tableaux


\begin{center}
\textbf{R\'esum\'e}
\end{center}
Un exemple de fichier {\LaTeX} de quelques lignes.

\section{Introduction}
{\LaTeX} permet d'\'ecrire des articles scientifiques (ou non). On pr\'esente ici quelques exemples.
Ce n'est pas un WYSIWYG (What You See Is What You Get).  {\LaTeX} interpr\`ete le code
pour g\'en\'erer un fichier (pdf ou autre), un peu \`a l'image d'un navigateur qui interpr\`ete le code HTML.
Cependant, la syntaxe {\LaTeX}  n'est pas de type XML, m\^eme si l'esprit est comparable.

\section{Les formules}
\subsection{Int\'egrale}
\subsubsection{En ligne}
Une formule en ligne telle que $\int_0^1 x\,dx=1/2$.
\subsubsection{Num\'erot\'ee}
\begin{equation}\label{Formule1}
\int_0^1 x\,dx=\frac 1 2.
\end{equation}
\paragraph{R\'ef\'erence} On peut faire r\'ef\'erence \`a la formule (\ref{Formule1}) [il faut compiler deux fois].
\subsection{Les nouvelles commandes}
\paragraph{Def} On note $\R$ l'ensemble des r\'eels.
\paragraph{Newcommand} On d\'esign par $\fonctionsContinues{[0,1]}{\R}$ l'enesmble des fonctions continues de $[0,1]$ vers $\R$.

\section{Figures}
On peut aussi inclure des figures sous diff\'erents formats. La figure \ref{cercle} est de format \textbf{pdf}  et a \'et\'e g\'en\'er\'ee avec \href{xwww.xfig.org}{Xfig}.
\begin{figure}
\begin{center}
\includegraphics[width=0.3\textwidth]{cercle.pdf}
\includegraphics[width=1cm]{cercle.pdf}
\caption{Des cercles}\label{cercle}
\end{center}
\end{figure}
{\LaTeX} optimise le placement des figures. C'est en g\'en\'eral une mauvaise id\'ee d'essayer de le contrarier.
\section{Conclusion}
Un bon manuel
est \cite{NotSoShort} qu'on peut trouver en ligne. Il faut toujours citer ses sources.
On peut \'egalement r\'ealiser des pr\'esentations de types PowerPoint avec {\LaTeX} en utilisant beamer.
Une alternative est de faire une présentation web (avec \href{http://lab.hakim.se/reveal-js/}{reveal.js} et \href{https://www.mathjax.org/}{MathJax}).

\bibliographystyle{abbrv}	% style utiise pour genere la page des references
\begin{thebibliography}{99}
\bibitem{NotSoShort} Tobias Oetiker, Hubert Partl, Irene Hyna and Elisabeth Schlegl
\href{https://tobi.oetiker.ch/lshort/lshort.pdf}{\emph{The Not So Short
Introduction to \LaTeX $2\varepsilon$}},
 (2015)
\end{thebibliography}
\end{document}


