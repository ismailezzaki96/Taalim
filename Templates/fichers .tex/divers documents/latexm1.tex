\documentclass{article}

\textwidth 15cm
\oddsidemargin=0mm
\topmargin=-7mm
\textheight 22 cm

\usepackage[utf8]{inputenc}
\usepackage[T1]{fontenc}
\usepackage{times}
\usepackage[francais]{babel}
\usepackage{amssymb}
\usepackage{ifpdf}
\ifpdf
 \usepackage[colorlinks,pdftex]{hyperref}
\fi
\newcommand{\bs}{\symbol{92} }  
\newtheorem{thm}{Théorème}
\newtheorem{prop}[thm]{Proposition}
\newtheorem{defn}{D\'efinition}

\begin{document}

\noindent UGA \hfill Introduction LaTeX  \hfill 2017/8

\section{Introduction}
\LaTeX\ est le format standard utilis\'e dans le monde de 
l'\'edition math\'ematique. Il existe des distributions libres de
\LaTeX\ pour toutes les plateformes (voir les r\'ef\'erences).

Un document au format \LaTeX\ est un texte (au format ASCII) contenant des
commandes de formatage. Ces commandes servent \`a
structurer le texte (chapitres, sections, etc.) en laissant au
compilateur le soin de rendre cette structure au mieux en fonction
du format de sortie (texte imprim\'e, fichier PDF, sortie HTML pour
mettre sur un site Web). Elles g\`erent aussi l'affichage des
symboles math\'ematiques, la num\'erotation des chapitres
(chapter), sections, sous-sections (section, subsection),
les r\'ef\`erences (on place un rep\`ere {\tt nom} avec {\tt label} puis on 
se r\'ef\`ere \`a {\tt nom} avec {\tt ref} ou {\tt pageref})
\begin{verbatim} 
\label{toto} puis \ref{toto} ou \pageref{toto}
\end{verbatim}
ou permettent de cr\'eer automatiquement la table des mati\`eres 
({\tt \bs \hskip-.05truein tableofcontents}).
Une fois saisi, le texte source \LaTeX\ doit \^etre compil\'e 
(parfois deux fois de suite pour mettre \`a jour les r\'ef\'erences et 
la table des mati\`eres). Le compilateur ne tient pas compte de la
mise en page du texte source, le nombre d'espace entre deux mots
est ignor\'e de m\^eme que le passage \`a la ligne. Seuls les
sauts de lignes sont interpr\'et\'es comme signalant un d\'ebut
de paragraphe.

La syntaxe d'une commande de formatage \LaTeX \  est :
\begin{verbatim}
\command[option]{argument} 
\end{verbatim}
Il existe dix caract\`eres r\'eserv\'es qui ne sont donc pas imprim\'es 
tels quels :\\
\$ \  \& \  \% \ \# \ \_ \ \{ \ \} \  \circonflexe \ \tild \ \symbol{92}

Pour les imprimer, il faut taper~:
\begin{verbatim}
\$ \& \% \# \_ \{ \} \circonflexe \tild \symbol{92}
\end{verbatim}

Le passage \`a la ligne (changement de paragraphe) se fait 
en ins\'erant une ligne vide, la ligne suivante est alors
indentée\footnote{On peut forcer un passage \`a la ligne sans 
indentation en tapant
{\tt \bs \hskip-.07truein\bs} mais ceci n'est pas 
recommand\'e pour la lisibilit\'e du texte.}

Un espace est cr\'ee avec {\tt \bs \hskip-.07truein  $_{\sqcup}$}.

\section{\'Edition, compilation, prévisualisation et impression.}

\subsection{Choix de l'éditeur, saisie d'un premier document.}
Pour \'editer votre texte en \LaTeX, vous devez utiliser un \'editeur
comme pour taper le code source d'un programme. 
Vous pouvez utiliser n'importe quel \'editeur si vous en connaissez
d\'ej\`a un, comme par exemple emacs
(un éditeur de fichiers sources C/C++, Java, Python, LaTeX, ... très puissant
mais qui nécessite un apprentissage...).
Sinon, vous pouvez apprendre TexMaker qui est un environnement 
facilitant beaucoup l'apprentissage de LaTeX avec
des raccourcis clavier compatibles Windows et des assistants et barres d'icones pour
saisir les symboles mathématiques, 

\subsubsection{TexMaker}
Recherchez Texmaker dans les programmes (menu Bureau,
pour l'installer sur votre
ordinateur, voir les liens en fin de document)
ou ouvrez un terminal 
(menu Accessoires) et tapez la commande
{\tt texmaker \&}\\
(Si vous avez oubli\'e le {\tt \&}, tapez Ctrl-Z puis {\tt bg}).
Cliquez sur Nouveau puis sur Assistant dans la barre d'icones, sélectionnez utf8x au lieu de latin1
comme encodage. Ajouter ensuite la partie du texte ci-dessous entre 
\verb|\begin{document}| et \verb|\end{document}|.

\subsubsection{Editeur classique.}
T\'el\'echargez le document \\
\verb|www-fourier.ujf-grenoble.fr/~parisse/info/essai.tex|\\
Ouvrez un terminal, puis \'editez le document dans le terminal, par exemple avec {\tt Emacs }~:\\
{\tt emacs essai.tex \&}\\
(Si vous avez oubli\'e le {\tt \&}, tapez Ctrl-Z puis {\tt bg}).
Vous pouvez aussi créer un nouveau document à
partir d'un fichier vide et taper les lignes suivantes
(sans les commentaires qui commencent par \%).
\begin{verbatim}
\documentclass[a4paper,11pt]{article}  % 11 ou 12pt, article ou report ou book 
\usepackage[utf8]{inputenc}            % caractères accentués en UTF8
\usepackage[T1]{fontenc}               % idem
\usepackage[francais]{babel}           % français (chapter -> chapitre...)
\usepackage{graphicx}                  % graphiques
\usepackage{amsmath,amsfonts,amssymb}  % symboles AMS
\newcommand{\N}{{\mathbb{N}}}          % définit la commande \N
\title{Un essai\\Stage latex M1} % définit le titre (ici sur 2 lignes)
\author{Mon Nom}                       % indiquez votre nom

\begin{document}                       % début du document

\maketitle                             % écrit le titre (cf. \title et \author)

\section{Calcul de $ A^P \pmod{N}$}    % un paragraphe
Soit $A \in \N$ un entier, ...         % on utilise la commande \N

\subsection{Traduction Algorithmique}  % un sous paragraphe
\label{sec:tradu}                      % définit un label
L'algorithme de la puissance rapide se compose de plusieurs parties
\begin{enumerate}
\item On commence par ...
\item Ensuite ...
\end{enumerate}

\subsection{Le programme en $C^{++}$}  % un autre sous paragraphe
On a vu (section \ref{sec:tradu}) ...  % une référence au label

\newpage                               % nouvelle page 
\tableofcontents                       % table des matières

\end{document}                         % fin du document 
\end{verbatim}

\subsection{La compilation et la prévisualisation}
Pour traduire les diff\'erentes commandes de votre texte, il faut le compiler.

\subsubsection{Texmaker}
Vous devez d'abord sauver votre texte, \`a la souris, icone disquette
ou menu {\tt Fichier -> Enregistrer} ou au clavier Ctrl-S.
Cliquez ensuite sur l'icone à gauche de Compilation rapide dans la barre
d'icones (vous pouvez sélectionner un autre format de rendu: par exemple 
le format Latex et le rendu DVI, ce dernier se met à jour automatiquement).
S'il y a des erreurs, elles apparaissent numérotées en-dessous, en cliquant sur le
numéro, on positionne le curseur dans le texte source à l'endroit de l'erreur.

\subsubsection{\'Editeur classique.}
Vous devez d'abord sauver votre texte, par exemple
avec emacs \`a la souris, icone disquette
ou menu {\tt Files -> Save current buffer} ou au clavier en tapant 
({\tt Ctrl-X Ctrl-S} sous emacs). 
Puis compilez en tapant
dans la fen\^etre de commandes (Konsole ou xterm)~:\\
{\tt latex essai}\\
Dans emacs, vous pouvez aussi utiliser le menu {\tt Tex-> Tex file}.

La compilation se fait avec traduction en un fichier {\tt essai.dvi} ou,
avec un arr\^et \`a la premi\`ere erreur rencontr\'ee.
Lorsque une erreur est d\'etect\'ee depuis emacs, un message apparait indiquant, la
nature de l'erreur et la ligne o\`u elle se situe. 

Tapez sur la touche
\verb|Entree| pour continuer ou tapez \verb|x| puis \verb|Entree| 
pour interrompre la compilation. Corrigez votre erreur dans
l'\'editeur et recompilez. Si vous avez compil\'e avec le menu
{\tt Tex->Tex file} d'emacs, vous pouvez consulter le fichier {\tt essai.log} pour
d\'eterminer les erreurs \`a corriger.

Pour visualiser votre texte avant l'impression, utilisez
le menu {\tt Tex->Tex view} dans emacs ou tapez dans la fen\^etre
de commandes~:\\
{\tt xdvi essai \&}

Si la page de visualisation n'est pas mise \`a jour lorsque
vous compilez \`a nouveau, vous devez quitter \verb|xdvi| en tapant
sur la touche \verb|q| et le relancer comme ci-dessus.

\subsection{L'impression}
Pour imprimer (attention ne le faites pas maintenant!), 
vous cliquerez dans TexMaker sur l'icone d'imprimante
ou vous taperez dans le fen\^etre de commandes~:\\
\verb|dvips essai|

\subsection{Cr\'eer des fichiers PDF et HTML \`a partir d'un source \LaTeX}
Si vous utilisez la commande {\tt pdflatex} \`a la place de la
commande {\tt latex}, le compilateur g\'en\'ere un fichier
{\tt .pdf} au format PDF (que l'on peut lire avec Acrobrat Reader
ou sous Unix avec {\tt gv} ou \verb|evince|). On peut aussi convertir un fichier
DVI en fichier PDF par la commande {\tt dvipdf}.

Pour obtenir une sortie HTML, utilisez la commande {\tt hevea} ou
{\tt latex2html} (disponible sur certains syst\`emes seulement).

\section{Les environnements \LaTeX}
Dans TexMaker, les commandes correspondantes se trouvent dans le menu {\tt LaTeX}.

C'est une partie du document délimitée par:\\
\verb|\begin{type d'environnement}...\end{type d'environnement}|\\
Voici quelques environnements souvent utilis\'es~:
\begin{itemize}
\item \verb|\begin{verbatim}| ... \verb|\end{verbatim}| : 
pas d'interprétation des commandes, 
le texte est mis en style {\tt \bs \hskip-.05truein texttt} 
(contrairement à {\tt \{\bs \hskip-.05truein tt...\}} qui met en style 
{\tt \bs \hskip-.05truein texttt} mais interpréte...) 
\item \verb|\begin{itemize}| ... \verb|\end{itemize}| ou 
\verb|\begin{enumerate}| ... \verb|\end{enumerate}| : 
permet d'énumerer une liste ; chaque élement de la liste doit commencer par 
{\tt \bs \hskip-.05truein item}\\
La diff\'erence est que {\tt enumerate} numérote les items
\item \verb|\begin{center}| ... \verb|\end{center}| permet de centrer
un texte
\item \verb_\begin{tabular}{|l|c|r|r|}_ ... \verb|\end{tabular}|~: 
cr\'ee un tableau. Le nombre d'arguments (ici 4) indique le nombre de 
colonnes. Ces arguments d\'efinissent l'alignement
{\tt l} (left),{\tt c} (center), {\tt r}  (right). On tape les lignes du 
tableau en séparant les colonnes  par \&. Chaque ligne est terminée par 
la commande {\tt \bs \hskip-.07truein\bs}. Si on \'ecrit la commande
{\tt \bs \hskip-.05truein hline} apr\`es une fin de ligne, cela affichera un 
trait de s\'eparation horizontal, Pour les traits de s\'eparation
verticaux, utiliser {\tt |} dans l'argument.
\item ``Exception'': pour mettre une partie de texte en italique, on \'ecrit
\verb|{\em ... }|, en gras \verb|{\bf ...}|.
\end{itemize}


\section{L'environnement mathématique}
Dans TexMaker, les commandes correspondantes se trouvent dans le menu {\tt Maths}.

\subsection{Le mode mathématique  }
Dans le corps d'un texte, les formules mathématiques sont délimit\'ees par
 un dollar, alors que les formules devant apparaître sur une ligne séparée 
sont délimit\'ees par deux  dollars. On tape par exemple :
\begin{verbatim}
 Considérons les équations $x+y=0$ et $x-y=2$.
\end{verbatim}
et on obtient~: 

Considérons les \'equations $x+y=0$ et $x-y=2$

alors que si on tape :
\begin{verbatim}
 Considérons les équations \[x+y=0 \ \mbox{et} \ x-y=2\]
\end{verbatim}
on obtient (la commande {\tt \bs \hskip-.05truein mbox} permet d'écrire 
du texte dans une formule) : 

Considérons les \'equations \[x+y=0 \ \mbox{et} \ x-y=2\]
%La commande {\tt \bs \hskip-.05truein mbox} permet d'écrire du texte dans une formule.

On peut aussi obtenir une \'equation num\'erot\'ee avec l'environnement
\verb|equation|~:
\begin{verbatim}
\begin{equation} \label{eq:def_x}
x = \sqrt{y+z}
\end{equation}
\end{verbatim}
ce qui donne~:\\
\begin{equation} \label{eq:def_x}
x = \sqrt{y+z}
\end{equation}

\subsection{Les fractions}
Une fraction s'obtient  avec la commande 
{\tt \bs \hskip-.05truein frac} ({\tt \bs \hskip-.05truein overline} surligne)
\begin{verbatim}
\[\frac{x}{2y}=0.4\overline{230769}\]
\end{verbatim}
donne :\[\frac{x}{2y}=0.4\overline{230769}\]
%\[x \ et\ il\ fait \ y\]
%\[x \ \mbox{et il fait }\ \quad y\]
%\[\framebox{RENEE DE GRAEVE}\]

\subsection{Les indices, les exposants et les flèches de vecteurs}
Les indices s'obtiennent avec le caract\`ere \_ , les exposants avec
 le caract\`ere \circonflexe \ et les flèches de vecteurs avec la commande
{\tt \bs \hskip-.05truein overrightarrow} 

Exemple :
\begin{verbatim}
\[x_1={(a^2+b^2)}^\frac{ 1 }{2}\]
\[\overrightarrow{OA_{1,i}}=x^{2t}\cdot \overrightarrow{OB_i}\]
\end{verbatim}
donne :\[x_1={(a^2+b^2)}^{\frac{ 1 }{2}}\]
\[\overrightarrow{OA_{1,i}}=x^{2t}\cdot \overrightarrow{OB_i}\]

\subsubsection{Les racines}
Une racine s'obtient avec  la commande :
%{\tt \bs \hskip-.1truein sqrt}
\begin{verbatim}\[\sqrt{x^2+1}\]
\[\sqrt[3]{x^2+1}\]
\end{verbatim}
donne :\[\sqrt{x^2+1}\]
\[\sqrt[3]{x^2+1}\]
\subsubsection{Les limites}
Une limite s'obtient  avec la commande : 
{\tt \bs \hskip-.05truein lim \{ ...\}}

Pour \'ecrire les fonctions math\'ematiques on les fait pr\'ec\'eder de {\tt \bs }.
On tape :
\begin{verbatim} 
\[\lim_{x \rightarrow +\infty} \ln (x) = +\infty\]
\end{verbatim}
pour obtenir :
\[\lim_{x \rightarrow +\infty} \ln (x) = +\infty\]

\subsection{Les matrices}
\begin{verbatim}
\[\left(\begin{array}{ccc}
 2 & 3 & 4\\ x & x^2 & x^3\\ 5 & 6 & 7
\end{array}\right)\]
\end{verbatim}
donne 
\[\left(\begin{array}{ccc}
 2 & 3 & 4\\ x & x^2 & x^3\\ 5 & 6 & 7
\end{array}\right)
\]

\subsection{Les int\'egrales et les s\'eries}
\begin{verbatim}
\[\int_a^b f(t) \; dt\]
\end{verbatim}
\noindent donne
\[\int_a^b f(t)\; dt \]
\\ 
\begin{verbatim}
$\sum_{i=0}^{+\infty} \frac{1}{i^2}$ et \[\sum_{i=0}^{+\infty} \frac{1}{i^2}\]
\end{verbatim}
\noindent donne : $\sum_{i=0}^{+\infty} \frac{1}{i^2}$ et
\[\sum_{i=0}^{+\infty} \frac{1}{i^2} \]

\subsection{Les deriv\'ees}
On utilise la commande \circonflexe  {\tt \bs \hskip-.05truein  prime} ou {\tt '}
\begin{verbatim}
\[ f'(x)=(\exp(2x))^\prime=2\exp(2x)\]
\end{verbatim}
donne
\[ f'(x)=(\exp(2x))^\prime=2\exp(2x)\]
Pour la d\'eriv\'ee seconde, utiliser \verb|f'{'}|.
Pour les  deriv\'ees partielles on utilise {\tt \bs \hskip-.05truein  partial} :
\begin{verbatim}
\[ \frac{\partial f(x,y)}{\partial x}=2\exp(2x)\]
\end{verbatim}
donne
\[ \frac{\partial f(x,y)}{\partial x}=2\exp(2x)\]

\section{\'Enonc\'es}
Pour mettre en valeur th\'eor\`emes, propositions, lemmes et autres
corollaires, on cr\'ee des environnements avec la commande 
\verb|newtheorem|. Ces environnements peuvent partager le m\^eme
compteur ou avoir des compteurs s\'epar\'es. On saisit au d\'ebut du
document~:
\begin{verbatim}
\newtheorem{thm}{Théorème}
\newtheorem{prop}[thm]{Proposition}
\newtheorem{defn}{D\'efinition}
\end{verbatim}
Ici, \verb|thm| et \verb|prop| partagent le m\^eme compteur, mais pas
\verb|defn|. 

Puis pour cr\'eer un \'enonc\'e, on tape\\
\verb|\begin{thm}|\\
\verb|\'Enonc\'e du th\'eor\`eme|\\
\verb|\end{thm}|\\
et on obtient
\begin{thm}
\'Enonc\'e du th\'eor\`eme
\end{thm}

\section{Références et citations}
On crée explicitement une référence avec la commande \verb|\label{}| et
un nom de label entre les accolades. La référence correspond au numéro de
section, sauf si on se trouve dans une environnement d'équation numérotée. 
On y fait ensuite référence avec
la commande \verb|\ref{}| (et \verb|\pageref{}| pour indiquer la page).

Certaines commandes, comme \verb|\tableofcontents| utilisent des références
crées implicitement (numéro de section).

Les citations d'oeuvre sont en général gérées par un programme externe, \verb|bibtex|
à partir d'un fichier de base de données bibliographiques. On utilise
les commandes (menu \verb|LaTeX| dans Texmaker)
\begin{itemize}
\item \verb|\cite{}| pour citer une oeuvre, avec en paramètre la référence de l'oeuvre
\item \verb|\bibliiography{}| pour indiquer le nom de fichier de la base de données
bibliographique,
\item \verb|\bibliographystyle{abbrv}| pour spécifier un style d'affichage des citations
(ici \verb|abbrv| pour abrégé),
\end{itemize}
Exemple, on crée un fichier \verb|mabiblio.bib| contenant (dans
Texmaker, utilisez le menu \verb|Bibliographie| pour aider la saisie)~:
\begin{verbatim}
@ book {Leborgne,
  AUTHOR="D. Leborgne",
  TITLE="{Calcul diff\'erentiel et g\'eom\'etrie}",
  PUBLISHER="PUF",
  YEAR="1982" }
\end{verbatim}
puis on ajoute dans le source latex une fois \verb|\bibliiography{mabiblio.bib}|
et \verb|\bibliographystyle{abbrv}|, et \verb|\cite{Leborgne}| autant de fois que
nécessaire. Sauvegarder le fichier \verb|mabiblio.bib|.

Depuis texmaker, revenir au fichier tex, puis avant-dernier menu
de la barre d'outil, faire une compilation
latex ou pdflatex, puis s\'electionner bibtex comme compilateur,
compiler, puis s\'electionner \`a nouveau latex ou pdflatex et
compiler 2 fois. En ligne de commande, taper\\
\verb|latex essai|\\
\verb|bibtex essai|\\
\verb|latex essai|\\
\verb|latex essai|

\section{Espaces, ponctuation, c\'esure.}
La philosophie de \LaTeX\ est de laisser le compilateur g\'erer les
espaces, cependant il peut se produire qu'il soit n\'ecessaire
d'en ajouter. Les commandes \verb|\, \ \quad \qquad|
permettent d'ajouter un espacement horizontal de taille de plus en
plus grande. On peut aussi utiliser \verb|\hspace{0.3cm}| o\`u
l'argument est une longueur (avec une unit\'e) pour un espacement
horizontal, \verb|\vspace{0.2cm}| pour un espacement vertical.

L'espacement en d\'ebut de paragraphe peut \^etre omis par la
commande \verb|\noindent|.

La commande \verb|\\| force un saut de ligne,
la commande \verb|\pagebreak| force un saut de page.

Les r\`egles de ponctuation en fran\c{c}ais imposent de mettre
toujours un espace apr\`es le signe de ponctuation, et d'en mettre
un avant si le signe de ponctuation poss\`ede deux composantes
connexes. Dans ce cas on utilise un espace ins\'ecable \verb|~|
pour \'eviter que le signe de ponctuation se trouve tout seul sur une
ligne.

En principe, \LaTeX\ sait o\`u couper dans un mot pour passer
\`a la ligne, mais il peut \^etre n\'ecessaire de l'aider, en
particulier si le mot contient des accents, on ajoute alors des \verb|\-|
pour s\'eparer les syllabes du mot.

\section{Ins\'erer un graphique}
On peut ins\'erer une image au format EPS (encapsulated postscript) 
dans un source \LaTeX\ de la mani\`ere suivante~:
\begin{verbatim}
\includegraphics[width=\textwidth]{image}
\end{verbatim}
o\`u {\tt image} d\'esigne le nom du fichier \verb|image.eps|.
On peut aussi indiquer
une largeur en centimètres après \verb|width=|.
La commande Unix {\tt convert} permet de convertir une image
d'un autre format vers le format Encapsulated Postscript,
par exemple
\begin{verbatim}
convert image.png image.eps
\end{verbatim}
Il faut avoir d\'eclar\'e en t\^ete (avant \verb|\begin{document}|)
du fichier source~:
\begin{verbatim}
\usepackage{graphicx}
\end{verbatim}

\section{Créer des transparents.}
Pour créer des transparents, on utilise fréquemment la classe de document \verb|beamer|.
Cf. par exemple le tutoriel sur \\
\verb|http://www.tuteurs.ens.fr/logiciels/latex/beamer.html|.

\section{Interaction avec des logiciels de calcul.}
De nombreux logiciels de calcul scientifique permettent d'interagir
avec \LaTeX, on donne deux exemples dans cette section.

\subsection{\tt giac/xcas}
Depuis Xcas, vous pouvez copier dans le presse-papier 
la traduction \LaTeX\ d'une expression ou
sous-expression en la s\'electionnant et en utilisant le raccourci
Ctrl-T. On peut aussi g\'en\'erer facilement un graphique ins\'erable
dans un fichier \LaTeX (menu M \`a droite du graphique, puis
Exporter).

Vous pouvez compiler avec \verb|hevea| un fichier source \LaTeX\
contenant des commandes de calcul
en un document HTML5 interactif permettant au lecteur de modifier
et/ou ex\'ecuter les commandes de calcul
depuis le navigateur avec lequel il consulte le document, 
pour plus de d\'etails, cf.\\
\verb|http://www-fourier.ujf-grenoble.fr/~parisse/giac/test_fr.tex|\\
\verb|http://www-fourier.ujf-grenoble.fr/~parisse/giac/castex.html|

Sous linux, vous pouvez g\'en\'erer les deux formats de sortie 
PDF et HTML5 interactif en utilisant la commande {\tt icas} ou {\tt
giac} au lieu de {\tt pdflatex}. Il suffit de cr\'eer un document
latex normal, y ajouter (juste apr\`es \verb|\begin{document}|)
\verb|\begin{giacjsonline}| et (juste avant \verb|\end{document}|)
\verb|\end{giacjsonline}|, puis taper des commandes telles que
\verb|\giacinputbigmath{factor(x^10-1)}| ou
\verb|\giacinput{plot(sin(x))}|.
La compilation s'effectue alors depuis un terminal en tapant la
commande\\
\verb|giac nomfichier|

Enfin {\tt pgiac} est un programme qui permet de faire calculer 
automatiquement par Giac (le moteur de calcul formel de Xcas) 
certaines expressions d'un fichier source au format \LaTeX.
Voir le site de J.Michel Sarlat pour des exemples\\
\verb|http://melusine.eu.org/syracuse/giac/|


\subsection{\tt texmacs}
{\tt texmacs} est un programme permettant de saisir des documents
math\'ematiques avec une interface similaire \`a celle
des logiciels de traitement de texte usuels tout en conservant
une qualit\'e typographique comparable \`a \LaTeX.
Il permet d'importer et d'exporter au format \LaTeX. 
Il poss\`ede \'egalement une interface pour lancer certains 
logiciel de calcul (Menu Inserer, sous-menu session).
Pour lancer {\tt texmacs} sous Unix, tapez la commande~:\\
{\tt texmacs \&}

\subsection{Pour aller plus loin}
\begin{itemize}
\item Exemples de distribution LaTeX \\
Windows: miktex \verb|http://miktex.org/|\\
Mac: \verb|http://www.tug.org/mactex/|\\
Linux: rechercher latex sur votre gestionnaire de paquets et
s\'electionner par exemple texlive
\item Le site de Texmaker:
\verb|www.xm1math.net/texmaker/index_fr.html|
\item
\verb|http://fr.wikibooks.org/wiki/Programmation_LaTeX|
\item \verb|http://www.tuteurs.ens.fr/logiciels/latex/|
\item
Le \LaTeX\ navigator~: \verb|http://tex.loria.fr/index.html|
\item le groupe AmiTeX
\verb|http://fr.groups.yahoo.com/group/AmiTeX/|
\item hevea, traducteur vers HTML
\verb|hevea.inria.fr|
\item
Le site de {\tt texmacs}~: {\tt www.texmacs.org}
\end{itemize}

\end{document}

\subsection{R\'ef\'erence.}

% Math-mode symbol & verbatim
\def\W#1#2{$#1{#2}$ &\tt\string#1\string{#2\string}}
\def\X#1{$#1$ &\tt\string#1}
\def\Y#1{$\big#1$ &\tt\string#1}
\def\Z#1{\tt\string#1}

% A non-floating table environment.
\makeatletter
\renewenvironment{table}%
   {\vskip\intextsep\parskip\z@
    \vbox\bgroup\centering\def\@captype{table}}%
   {\egroup\vskip\intextsep}
\makeatother

% All the tables are \label'ed in case this document ever gets some
% explanatory text written, however there are no \refs as yet. To save
% LaTeX-ing the file twice we go:
\renewcommand{\label}[1]{}

% A4 page setup
%\topmargin -45pt
%\textwidth=532pt
%\oddsidemargin=-40pt \evensidemargin\oddsidemargin
%\textheight 682pt


\begin{table}
\begin{tabular}{*8l}
\X\alpha        &\X\theta       &\X o           &\X\tau         \\
\X\beta         &\X\vartheta    &\X\pi          &\X\upsilon     \\
\X\gamma        &\X\gamma       &\X\varpi       &\X\phi         \\
\X\delta        &\X\kappa       &\X\rho         &\X\varphi      \\
\X\epsilon      &\X\lambda      &\X\varrho      &\X\chi         \\
\X\varepsilon   &\X\mu          &\X\sigma       &\X\psi         \\
\X\zeta         &\X\nu          &\X\varsigma    &\X\omega       \\
\X\eta          &\X\xi                                          \\
                                                                \\
\X\Gamma        &\X\Lambda      &\X\Sigma       &\X\Psi         \\
\X\Delta        &\X\Xi          &\X\Upsilon     &\X\Omega       \\
\X\Theta        &\X\Pi          &\X\Phi
\end{tabular}
\caption{Greek Letters}\label{greek}
\end{table}

\begin{table}
\begin{tabular}{*8l}
\X\pm           &\X\cap         &\X\diamond             &\X\oplus     \\
\X\mp           &\X\cup         &\X\bigtriangleup       &\X\ominus    \\
\X\times        &\X\uplus       &\X\bigtriangledown     &\X\otimes    \\
\X\div          &\X\sqcap       &\X\triangleleft        &\X\oslash    \\
\X\ast          &\X\sqcup       &\X\triangleright       &\X\odot      \\
\X\star         &\X\vee         &\X\lhd$^b$             &\X\bigcirc   \\
\X\circ         &\X\wedge       &\X\rhd$^b$             &\X\dagger    \\
\X\bullet       &\X\setminus    &\X\unlhd$^b$           &\X\ddagger   \\
\X\cdot         &\X\wr          &\X\unrhd$^b$           &\X\amalg     \\
\X+             &\X-
\end{tabular}

$^b$ Not predefined in a format based on {\tt basefont.tex}.
     Use one of the style options\\
     {\tt oldlfont}, {\tt newlfont}, {\tt amsfonts} or {\tt amssymb}.

\caption{Binary Operation Symbols}\label{bin}
\end{table}


\begin{table}
\begin{tabular}{*8l}
\X\leq          &\X\geq         &\X\equiv       &\X\models      \\
\X\prec         &\X\succ        &\X\sim         &\X\perp        \\
\X\preceq       &\X\succeq      &\X\simeq       &\X\mid         \\
\X\ll           &\X\gg          &\X\asymp       &\X\parallel    \\
\X\subset       &\X\supset      &\X\approx      &\X\bowtie      \\
\X\subseteq     &\X\supseteq    &\X\cong        &\X\Join$^b$    \\
\X\sqsubset$^b$ &\X\sqsupset$^b$&\X\neq         &\X\smile       \\
\X\sqsubseteq   &\X\sqsupseteq  &\X\doteq       &\X\frown       \\
\X\in           &\X\ni          &\X\propto      &\X=            \\
\X\vdash        &\X\dashv       &\X<            &\X>            \\
\X:
\end{tabular}

$^b$ Not predefined in a format based on {\tt basefont.tex}.
     Use one of the style options\\
     {\tt oldlfont}, {\tt newlfont}, {\tt amsfonts} or {\tt amssymb}.

\caption{Relation Symbols}\label{rel}
\end{table}

\begin{table}
\begin{tabular}{*{5}{lp{3.2em}}}
\X,     &\X;    &\X\colon       &\X\ldotp       &\X\cdotp
\end{tabular}
\caption{Punctuation Symbols}\label{punct}
\end{table}

\begin{table}
\begin{tabular}{*6l}
\X\leftarrow            &\X\longleftarrow       &\X\uparrow     \\
\X\Leftarrow            &\X\Longleftarrow       &\X\Uparrow     \\
\X\rightarrow           &\X\longrightarrow      &\X\downarrow   \\
\X\Rightarrow           &\X\Longrightarrow      &\X\Downarrow   \\
\X\leftrightarrow       &\X\longleftrightarrow  &\X\updownarrow \\
\X\Leftrightarrow       &\X\Longleftrightarrow  &\X\Updownarrow \\
\X\mapsto               &\X\longmapsto          &\X\nearrow     \\
\X\hookleftarrow        &\X\hookrightarrow      &\X\searrow     \\
\X\leftharpoonup        &\X\rightharpoonup      &\X\swarrow     \\
\X\leftharpoondown      &\X\rightharpoondown    &\X\nwarrow     \\
\X\rightleftharpoons    &\X\leadsto$^b$
\end{tabular}

$^b$ Not predefined in a format based on {\tt basefont.tex}.
     Use one of the style options\\
     {\tt oldlfont}, {\tt newlfont}, {\tt amsfonts} or {\tt amssymb}.

\caption{Arrow Symbols}
\end{table}

\begin{table}
\begin{tabular}{*8l}
\X\ldots        &\X\cdots       &\X\vdots       &\X\ddots       \\
\X\aleph        &\X\prime       &\X\forall      &\X\infty       \\
\X\hbar         &\X\emptyset    &\X\exists      &\X\Box$^b$     \\
\X\imath        &\X\nabla       &\X\neg         &\X\Diamond$^b$ \\
\X\jmath        &\X\surd        &\X\flat        &\X\triangle    \\
\X\ell          &\X\top         &\X\natural     &\X\clubsuit    \\
\X\wp           &\X\bot         &\X\sharp       &\X\diamondsuit \\
\X\Re           &\X\|           &\X\backslash   &\X\heartsuit   \\
\X\Im           &\X\angle       &\X\partial     &\X\spadesuit   \\
\X\mho$^b$      &\X.            &\X|
\end{tabular}

$^b$ Not predefined in a format based on {\tt basefont.tex}.
     Use one of the style options\\
     {\tt oldlfont}, {\tt newlfont}, {\tt amsfonts} or {\tt amssymb}.

\caption{Miscellaneous Symbols}\label{ord}
\end{table}

\begin{table}
\begin{tabular}{*6l}
\X\sum          &\X\bigcap      &\X\bigodot     \\
\X\prod         &\X\bigcup      &\X\bigotimes   \\
\X\coprod       &\X\bigsqcup    &\X\bigoplus    \\
\X\int          &\X\bigvee      &\X\biguplus    \\
\X\oint         &\X\bigwedge
\end{tabular}
\caption{Variable-sized  Symbols}\label{op}
\end{table}


\begin{table}
\begin{tabular}{*8l}
\Z\arccos &\Z\cos  &\Z\csc &\Z\exp &
           \Z\ker    &\Z\limsup &\Z\min &\Z\sinh \\
\Z\arcsin &\Z\cosh &\Z\deg &\Z\gcd &
           \Z\lg     &\Z\ln     &\Z\Pr  &\Z\sup  \\
\Z\arctan &\Z\cot  &\Z\det &\Z\hom &
           \Z\lim    &\Z\log    &\Z\sec &\Z\tan  \\
\Z\arg    &\Z\coth &\Z\dim &\Z\inf &
           \Z\liminf &\Z\max    &\Z\sin &\Z\tanh
\end{tabular}
\caption{Log-like Symbols}\label{log}
\end{table}


\begin{table}
\begin{tabular}{*8l}
\X(             &\X)            &\X\uparrow     &\X\Uparrow     \\
\X[             &\X]            &\X\downarrow   &\X\Downarrow   \\
\X\{            &\X\}           &\X\updownarrow &\X\Updownarrow \\
\X\lfloor       &\X\rfloor      &\X\lceil       &\X\rceil       \\
\X\langle       &\X\rangle      &\X/            &\X\backslash   \\
\X|             &\X\|
\end{tabular}
\caption{Delimiters\label{dels}}
\end{table}

\begin{table}
\begin{tabular}{*8l}
\Y\rmoustache&  \Y\lmoustache&  \Y\rgroup&      \Y\lgroup\\[5pt]
\Y\arrowvert&   \Y\Arrowvert&   \Y\bracevert
\end{tabular}
\caption{Large Delimiters\label{ldels}}
\end{table}

\begin{table}
\begin{tabular}{*{10}l}
\W\hat{a}     &\W\acute{a}  &\W\bar{a}    &\W\dot{a}    &\W\breve{a}\\
\W\check{a}   &\W\grave{a}  &\W\vec{a}    &\W\ddot{a}   &\W\tilde{a}\\
\end{tabular}
\caption{Math mode accents}\label{accent}
\end{table}

\begin{table}
\begin{tabular}{*4l}
\W\widetilde{abc}       &\W\widehat{abc}                        \\
\W\overleftarrow{abc}   &\W\overrightarrow{abc}                 \\
\W\overline{abc}        &\W\underline{abc}                      \\
\W\overbrace{abc}       &\W\underbrace{abc}                     \\[5pt]
\W\sqrt{abc}            &$\sqrt[n]{abc}$&\verb|\sqrt[n]{abc}|   \\
$f'$&\verb|f'|          &$\frac{abc}{xyz}$&\verb|\frac{abc}{xyz}|
\end{tabular}
\caption{Some other constructions}\label{other}
\end{table}


\end{document}


