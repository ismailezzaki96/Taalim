\documentclass[12pt,a4paper,oneside]{book}
%%%%%%%%%%%%%%%%%%%%%%%%%%%%%%%%%%%%%%%%%%%%%%%%%
\usepackage{color}
\usepackage{amsmath}
\usepackage{graphicx}
\usepackage[T1]{fontenc}
\usepackage[utf8]{inputenc}
\usepackage[francais]{babel}
%%%%%%%%%%%%%%%%%%%%%%%%%%%%%%%%%%%%%%%%%%%%%%%%%
\usepackage[left=2cm,right=2cm,top=2cm,bottom=2cm]{geometry}
\pagestyle{plain}
\begin{document}
%%%%%%%%%%%%%%%%%%%%%%%%%%%%%%%%%%%%%%%%%%%%%%%%%
%---------------------------------- page de garde
\thispagestyle{empty}
\begin{center} 
\includegraphics{LOGOumifs.png} \\
\vspace{2\baselineskip} \\
\textcolor{blue}
{RAPPORT DE PROJET DE FIN D'ETUDE } \\
\vspace{2\baselineskip} \\
{\scriptsize EN} \\
\vspace{2\baselineskip}  \\
{\large \textcolor{blue}
{SCIENCES MATHEMATIQUES ET APPLICATIONS}}\\
\vspace{2\baselineskip} \\
{\scriptsize SOUS LE THEME} \\
\vspace{4\baselineskip}  \\
\Huge \textcolor{red}
{ETUDE DE LA SUITE DE KOLAKOSKI-(1,3)}\\
\vspace{2\baselineskip}   \normalsize 
\rule{0,55\textwidth}{2pt} \\
\vspace{2\baselineskip} 
\textcolor{green}{
1333111333131...
\left(\begin{array}{cccc}
1 & 0 & 2 & 0 \\ 
0 & 0 & 1 & 0 \\ 
0 & 0 & 0 & 1 \\ 
0 & 1 & 0 & 2
\end{array} \right) 
0,6027847152... \\ 
\vspace{2\baselineskip} \\
\rule{0,55\textwidth}{2pt} \\
\vspace{2\baselineskip}}\\
\end{center}
\hspace{2\baselineskip}
      \begin{tabular}{llll} 
\textbf{Présenté par} &:& \textcolor{blue}
              {Abdelhakim BOUCHIAR} & \\ 
                     & & \textcolor{blue}
              {Ismail ELMARJANNI} & \\
                     & & & \\
\textbf{Devant le jury} & : & \textcolor{blue}{Pr. Abdellah HAMMAM} & \textcolor{blue}{(Encadrant)}\\
& & \textcolor{blue}{Pr.} &  \\
& & \textcolor{blue}{Pr.} & \\
& & & \\
      \end{tabular}\\
\begin{flushright}
\textcolor{blue}{\emph{Année Universitaire: 2016/2017.}}
\end{flushright} 
\pagebreak           
%remerciements
\rule{1\textwidth}{0pt} \\ 
\rule{1\textwidth}{0pt} \\
 \rule{1\textwidth}{0pt} \\
 \rule{1\textwidth}{0pt} \\
\rule{1\textwidth}{0pt} \\
 \begin{center}
%---------------------------------- Remerciements
   {\Huge \textbf{Remerciements}}
 \end{center}
\vspace{8\baselineskip}  
\par Au terme de ce travail, nous profitons de l'occasion pour remercier tout d'abord, notre \\ \rule{1\textwidth}{0pt} \\ professeur Monsieur Abdellah HAMMAM, qui n'a pas cessé de nous encourager pendant la \\  
\rule{1\textwidth}{0pt}} \\ durée du travail sur ce projet, ainsi nous le remercions pour l'encadrement et l'aide. \\ \vspace{3\baselineskip} \\
\par Egalement, nous présentons nos remerciements à tous les membres du jury, qui nous ont  \\ \rule{1\textwidth}{0pt} \\fait l'honneur d'accepter de juger notre travail. \\  \vspace{3\baselineskip} \\
\par Nous tenons à remercier tous les professeurs des filières SMIA et SMA, pour les efforts qui \\ \rule{1\textwidth}{0pt} \\ nous assurent une formation efficace et complète tout au long des années d'étude.
\thispagestyle{empty}
\newpage
%%---------------------------------------- résumé                     
\rule{1\textwidth}{0pt} \\ \rule{1\textwidth}{0pt} \\ \rule{1\textwidth}{0pt} \\ \rule{1\textwidth}{0pt} \\
\rule{1\textwidth}{0pt} \\
     \begin{center}
           {\Huge \textbf{Résumé}}\\
     \end{center} \\ 
\vspace{10\baselineskip}
\thispagestyle{empty}
\par Parmi les problèmes rencontrés en mathématiques, on trouve la conjecture de \textbf{Keane}: Y 
\rule{1\textwidth}{0pt} \\
 a-t-il autant de 1 que de 2 dans le mot infini de \textbf{Kolakoski}?.  Jusqu'à maintenant personne ne le 
\rule{1\textwidth}{0pt} \\ 
sait. La résolution d'un problème de ce volume nécessite des études des cas simples afin d'avoir, 
\rule{1\textwidth}{0pt} \\ 
d'une part, des idées sur le comportement des mots de \textbf{Kolakoski} particuliers, et d'autre part, 
\rule{1\textwidth}{0pt} \\
 des résultats que l'on essaiera de généraliser. Par exemple, l'étude qu'on présentera dans ce 
 \rule{1\textwidth}{0pt} \\ 
 rapport consiste à traiter le cas de l'alphabet \{1,3\}, et on essaie de généraliser les méthodes \rule{1\textwidth}{0pt} \\ 
 employées sur les mots formés par deux lettres impaires. Notre objectif est la recherche de la 
\rule{1\textwidth}{0pt} \\
  fréquence de 3. On utilisera deux méthodes, l'une se base sur la notion de suites et sous 
\rule{1\textwidth}{0pt} \\ 
  suites, et l'autre sur la notion de valeurs propres. 
%------------------------------------table des matieres 
\tableofcontents
\thispagestyle{empty}
\pagebreak
%%%%%%%%%%%%%%%%%%%%%%%%%%%%%%%%%%%%%%%%%%%%%%%%%
\begin{titlepage}
\chapter*{}      
%----------------------------------- introduction
      \begin{center}
            {\Huge \textbf{Introduction}}
      \end{center}
\rule{1\textwidth}{0pt} \\\rule{1\textwidth}{0pt} \\\rule{1\textwidth}{0pt} \\\rule{1\textwidth}{0pt} \\
\addcontentsline{toc}{chapter}{Introduction}
\section*{}
\par Le travail présenté dans ce rapport entre dans le cadre de notre projet de fin d'étude pour \\ \rule{1\textwidth}{0pt} \\ l'obtention de la licence en sciences mathématiques et applications. Nous nous intéressons à \\ \rule{1\textwidth}{0pt} \\ l'étude du mot infini de Kolakoski en utilisant l'alphabet \{1,3\}.
\section*{}
\par Dans ce rapport, nous présentons notre investigation dans le but est de chercher la limite  \\ \rule{0,3\textwidth}{0pt} \\ de la densité de 3 dans le mot infini de \textbf{Kolakoski}.
\section*{}
\par Ce rapport comporte deux chapitres principaux. Le premier présentera le sujet étudié, les \\ \rule{0,3\textwidth}{0pt} \\ notations et les définitions de base, ainsi que des propriétés indispensables à l'étude. Le deuxième \\ \rule{0,3\textwidth}{0pt} \\constitue la démarche de l'étude, dans lequel on donnera deux méthodes différentes pour la  \\ \rule{0,3\textwidth}{0pt} \\ recherche de la densité. On finira ce rapport par une conclusion.
%%%%%%%%%%%%%%%%%%%%%%%%%%%%%%%%%%%%%%%%%%%%%%%%%
%------------------------------------- chapitre 1 
\begin{chapter}{Présentation du sujet}
\par Dans ce chapitre nous introduisons les concepts, les définitions de base et les notations qui seront utilisées tout au long de ce travail.
\section{Généralités}
\subsection{Définitions et Notations}
\subsubsection*{Définition}
On appelle alphabet tout ensemble fini d'entiers, qui sont appelés des lettres.
\subsubsection*{Remarque}
Dans notre étude on a besoin d'un alphabet \(\Sigma=\{a,b\}\), où a et b sont deux entiers non nuls positifs distincts. A partir de la section 1.4 , on prend a=1 et b=3.
\subsubsection*{Notation}
On désigne par  \(\Sigma^*\), l'ensemble des mots formés par les lettres de $\Sigma$. 
\subsubsection*{Définition (mots fini et infini)}
On définit un mot fini M sur $\Sigma$ comme une suite finie de lettres $(M_i)_{1\leq i\leq n}$, avec n fixé dans $ \mathbb{N}$, telle que $\forall i=1,...,n$ : $M_i$ $\in\Sigma$, et on écrit $M=M_1M_2...M_n$, où $M_n$ est la n-ième lettre de M. On dit alors que M est de longueur n, et on note long(M)=n.
\par De m\^eme on définit un mot infini K sur $\Sigma$ comme étant une suite infinie des éléments de $\Sigma$, on le note: $K=K_1K_2...$, où $K_i \in \Sigma$, $\forall i\geq 1.$\\
\par Soit $M \in \Sigma^*$, On adopte la convention qu'un mot vide est de longueur zéro, et on le not $\varepsilon$. Ensuite on définit la somme partielle par: $$S_n=\sum_{i=1}^nM_i.$$
\par Dans un mot M $\in \Sigma^*$ de longueur n, le nombre d'occurrences de $\alpha \in \Sigma$ est noté $\alpha_n$, s'il s'agit d'une occurrence d'indice impair on le not $\alpha_n^i$, et s'il est d'indice pair, on le note $\alpha_n^p $, on a donc:
$$\forall \alpha \in \Sigma : \alpha_n = \alpha_n^i + \alpha_n^p$$ 
et en particulier pour a et b
$$n= a_n + b_n = a_n^i + a_n^p + b_n^i + b_n^p$$ 
\par On définit aussi un bloc dans un mot M par toute sous partie de lettres identiques successives de longueur maximal.\\
\par On termine cette partie par des exemples qui illustrent ces notions.
 \subsection{Exemples}
\par Soit $\Sigma = \{1,3\}$, alors le mot M = 1 3 3 3 1 1 1 est bien dans $\Sigma^*$, car il ne comporte que de 1 et de 3, par contre le mot N = 1 3 2 1 1 1 n'est pas dans $\Sigma^*$ car tout simplement contient 2 et 2 $\notin \Sigma$. Le mot M est fini et de longueur long(M)=7 et on a $$M_1 = 1,\hspace{0,5\baselineskip}  M_2 = 3,\hspace{0,5\baselineskip} M_3 = 3,\hspace{0,5\baselineskip} ... $$ et on a aussi \\
 \[ 1_7^i=3 , \hspace{0,5\baselineskip}1_7^p=1 , \hspace{0,5\baselineskip} 3_7^i=1 ,\hspace{0,5\baselineskip}  3_7^p=2 \] 
 et donc
\[ 1_7 = 1_7^i + 1_7^p =3+1=4 , \hspace{0,5\baselineskip} 3_7=3_7^i+3_7^p=1+2=3,\hspace{0,5\baselineskip}   7=1_7+3_7 .\]
\par Ce mot M contient trois blocs: \hspace{0,5\baselineskip} 1 \hspace{0,5\baselineskip} , \hspace{0,5\baselineskip} 333 \hspace{0,5\baselineskip} et\hspace{0,5\baselineskip} 111.
\section{Densité}
\par Nous arrivons maintenant à l'objet de ce travail, il s'agit de la densité ou la fréquence.
\subsubsection{Définition}
\par Avec les notations énoncées au début de ce chapitre, on définit la densité d'une lettre $\alpha \in \Sigma$ dans un mot M fini de longueur n par \[ \rho_n = \frac{|\{ 1 \leq j \leq n : M_j =\alpha \} |}{n} ,\]   \\ 
ou d'une autre façon \[ \rho_n = \frac{\alpha_n^i+\alpha_n^p}{n} .\]   \\
\section{Fonction de codage par blocs et sa réciproque}
\subsection{Fonction de codage par blocs}
\subsubsection*{Définition}
\par La fonction de codage par blocs est définie par l'opérateur: 
 \[ \Delta: \Sigma^* \underset{M \longmapsto \Delta (M)}{\longrightarrow} \Sigma^*=\{a,b\}^*, \]
avec $\Delta(M) = M' = M'_1 M'_2 ... $ en prenant pour tout $ i \geq 1$, ${M'}_i$ comme étant la longueur maximale du i-ième bloc dans le mot M.
\subsubsection*{Exemple}
On considère le mot M=133311333.\\
on a
 \begin{center}
$M=B_1B_2B_3B_4$,\hspace{0,5\baselineskip} où $B_1=1$,\hspace{0,5\baselineskip} $B_2=333$,\hspace{0,5\baselineskip} $ B_3=11$ \hspace{0,5\baselineskip}et \hspace{0,5\baselineskip}$B_4=333$
\end{center}\\ 
avec 
\begin{center}
$long(B_1)=1$,\hspace{0,5\baselineskip} $long(B_2)=3$,\hspace{0,5\baselineskip} $long(B_3)=2$ \hspace{0,5\baselineskip} et \hspace{0,5\baselineskip} $long(B_4)=3$
\end{center}
  \\donc
 $$\Delta(M)=\Delta(133311333)=1323$$
\subsection{Inter\^et de \Delta}
\par La fonction de codage par blocs est utilisée  comme moyen de compression des données. Par exemple, la première étape qu'effectuent les photocopieuses afin de compresser les données consiste à l'application de $\Delta$ à chaque ligne de pixels.
\subsubsection*{Remarques}
\par $_*$ La fonction $\Delta$ est une contraction, c'est-à-dire que pour tout mot M dans $\Sigma^*$ on a 
$$long(\Delta(M))\leq long(M).$$
\par $_*$ Cette fonction n'est pas pratique pour notre étude, on utilise au lieu d'elle sa réciproque qui s'appelle la fonction pseudo-inverse donnée dans la section suivante. 
\subsection{Fonction pseudo-inverse}
\par Avant d'énoncer la définition de la fonction pseudo-inverse on a besoin de définir ce qu'on appelle un mot lisse.
\subsubsection*{Définition}
\par Un mot $M \in \Sigma^*$ est dit lisse si: $\forall k \geq 0:\Delta^k(M) \in \Sigma^*$.
\subsubsection*{Exemple}
\par Soit M=13331113331313331113331. On a:\\
$ \Delta^0(M)=M\\
\Delta^1(M)=13331113331\\
\Delta^2(M)=13331\\
\Delta^3(M)=131\\
\Delta^4(M)=111\\
\Delta^5(M)=3\\
\Delta^6(M)=1\\
\Delta^k(M)=1\hspace{0,5\baselineskip} \forall k \geq 7.$\\
Donc M est un mot lisse.
\subsubsection*{Définition (fonction pseudo-inverse)} 
%\addcontentsline{toc}{subsubsection}{Définition(fonction pseudo-inverse)}
\par C'est la fonction réciproque de la fonction de codage par blocs $\Delta$ , on la note $\Delta^{-1}$.
\par On a \\ \[ \Delta^{-1}: \Sigma^* \underset{M \longmapsto \Delta^{-1}(M)}{\longrightarrow} \Sigma^*=\{a,b\}^* \]\\
avec $\Delta^{-1}(M)= W_1 W_2 ...$ où $W_1=w_1^1=a$, et $\forall i \geq 2$ on a: \\ $ W_i=w_1^iw_2^i...w_{M_i}^i$ est un bloc de longueur $M_i$ tel que pour $j=1,...,M_{i}$ \\
\[
\left\{ \begin{array}{lll} 
w_j^i = a & si & w_1^{i-1} = b \\
w_j^i = b & si & w_1^{i-1} = a
\end{array} 
\]
 \subsubsection*{Exemple}
Soit M=1333111, donc $\Delta^{-1}(M)$ composé par 7 blocs $B_1,B_2,...,B_7$ \, tels que: \\

\begin{center}
\begin{tabular}{lcl}
$
M_1=1 & \Rightarrow & B_1=1\\
M_2=3 & \Rightarrow & B_2=333\\
M_3=3 & \Rightarrow & B_3=111\\
M_4=3 & \Rightarrow & B_4=333\\
M_5=1 & \Rightarrow & B_5=1\\
M_6=1 & \Rightarrow & B_6=3\\
M_7=1 & \Rightarrow & B_7=1$\end{tabular} \\
\end{center}
Donc $$\Delta^{-1}(M)=B_1B_2B_3B_4B_5B_6B_7=1333111333131$$
\subsubsection*{Remarque}
\par La fonction pseudo-inverse $\Delta^{-1}$ est une anti-contraction, c'est-à-dire que pour tout mot M dans $\Sigma^*$ on a $$long(M) \leq long(\Delta^{-1}(M)).$$ 
\section{Mot infini de Kolakoski}
\par Soit $\Sigma=\{1,3\}$ .
\subsection{Définition}
 \par On appelle mot de Kolakoski, le point fixe sous l'opérateur $\Delta$ ayant 1 comme première lettre. On le note K.
\subsection{Construction du mot de Kolakoski}
\par Pour construire le mot de Kolakoski on utilise la notation suivante:
 $$K^p=K_1K_2...K_p \hspace{1\baselineskip} \forall p \geq 1.$$
\par On sait que K commence par 1, c'est-à-dire $K_1=1$, donc forcement $K_2=3$, ce qui implique que $K^2=13$, alors $(\Delta^{-1}(K_2))=1333=K^4$, donc en appliquant  
$\Delta^{-1}$ à $K^4$ on aura $K^{10}=1333111333$. \\ D'où par applications successives de $\Delta^{-1}$ on a \\
$K^{22}=1333111333131333111333$\\
$K^{52}=1333111333131333111333131333111333111333111333111333$ \\
On peut continuer autant de fois que l'on veut pour avoir une certaine longueur.
\subsection{Propriétés}
\begin{tabular}{ll}
\textbf{i)}   & Le mot de Kolakoski est un mot infini.\\
\textbf{ii)} & Le mot de Kolakoki est unique.\\
\textbf{iii)} & Le mot de Kolakoski est un mot lisse: $K=\Delta^k(K), \forall k \geq 0$.
\end{tabular}
\subsubsection*{Question: Quelle est la densité asymptotique de chaque lettre?}
Nous allons essayer de répondre à cette question dans le cas particulier a=1, b=3. Noté aussi Kol(1,3).
\end{chapter}
%chapitre 2 
\begin{chapter}{Suite de KOLAKOSKI-(1,3)}
\section*{Introduction}
\par Ce chapitre présente d'une manière détaillée le contenu principal de la recherche. Notre objectif est de traiter le problème suivant: Quelle est la valeur de la densité de 3 dans le mot infini de Kolakoski défini dans la section 1.4 . Pour résoudre ce problème, on propose deux méthodes, la première sera traitée dans la deuxième section de ce chapitre, alors que l'autre est dans la troisième, et on commence par tout ce qui commun dans la première section.
\section{Matrice de passage }
\par Soit $K=K_1K_2...$ le mot infini de Kolakoski défini au chapitre 1. Comme $n$ et $S_n$ ont la m\^eme parité. Donc on peut faire les identifications du Tableau 1, Annexe A. De ce tableau on peut extraire les relations suivantes:
\[ \left\{ \begin{array}{l}
1_{S_n}^i=1_n^i+2 \times 3_n^i \\ 
1_{S_n}^p=3_n^i \\ 
3_{S_n}^i=3_n^p \\ 
3_{S_n}^p=1_n^p+2 \times 3_n^p
\end{array} \]\\
On écrit se système sous la forme matricielle:
\[ \left( \begin{array}{c}
1_{S_n}^i \\ 
1_{S_n}^p \\ 
3_{S_n}^i \\ 
3_{S_n}^p
\end{array} \right) = 
\left( \begin{array}{cccc}
1 & 0 & 2 & 0 \\ 
0 & 0 & 1 & 0 \\ 
0 & 0 & 0 & 1 \\ 
0 & 1 & 0 & 2
\end{array} \right)
\left( \begin{array}{c}
1_n^i \\ 
1_n^p \\ 
3_n^i \\ 
3_n^p
\end{array} \right) \] \\
On pose \,\ $X_{S_n}=\left( \begin{array}{c}
1_{S_n}^i \\ 
1_{S_n}^p \\ 
3_{S_n}^i \\ 
3_{S_n}^p
\end{array} \right) \,\ , \,\ X_n=
\left( \begin{array}{c}
1_n^i \\ 
1_n^p \\ 
3_n^i \\ 
3_n^p
\end{array} \right)$ \,\ et \,\ $ M=
\left( \begin{array}{cccc}
1 & 0 & 2 & 0 \\ 
0 & 0 & 1 & 0 \\ 
0 & 0 & 0 & 1 \\ 
0 & 1 & 0 & 2
\end{array} \right)$.\\ \\Donc M est la  matrice de passage de $X_n$ à $X_{S_n}$.\\
En faisant appelle au Tableau 2, Annexe A, on vérifie les cas suivants respectivement pour n=1,2 et 3:\\
\begin{footnotesize}
$$ \left( \begin{array}{c}
1 \\ 
0 \\ 
0 \\ 
0
\end{array} \right) = 
\left( \begin{array}{cccc}
1 & 0 & 2 & 0 \\ 
0 & 0 & 1 & 0 \\ 
0 & 0 & 0 & 1 \\ 
0 & 1 & 0 & 2
\end{array} \right)
\left( \begin{array}{c}
1 \\ 
0 \\
0\\ 
0
\end{array} \right) \left| \left|
\left( \begin{array}{c}
3 \\ 
1\\ 
1 \\ 
2
\end{array} \right) = 
\left( \begin{array}{cccc}
1 & 0 & 2 & 0 \\ 
0 & 0 & 1 & 0 \\ 
0 & 0 & 0 & 1 \\ 
0 & 1 & 0 & 2
\end{array} \right)
\left( \begin{array}{c}
1 \\ 
0\\ 
0\\ 
1
\end{array} \right) \left| \left|
\left( \begin{array}{c}
3 \\ 
1 \\ 
1\\ 
2
\end{array} \right) = 
\left( \begin{array}{cccc}
1 & 0 & 2 & 0 \\ 
0 & 0 & 1 & 0 \\ 
0 & 0 & 0 & 1 \\ 
0 & 1 & 0 & 2
\end{array} \right)
\left( \begin{array}{c}
1 \\ 
0\\ 
1 \\ 
1
\end{array} \right)$$
\end{footnotesize}.
\newpage 
\section{Méthode des valeurs propres}
On a  \,\ $\left( \begin{array}{c}
1_{S_n}^i \\ 
1_{S_n}^p \\ 
3_{S_n}^i \\ 
3_{S_n}^p
\end{array} \right) = 
\left( \begin{array}{cccc}
1 & 0 & 2 & 0 \\ 
0 & 0 & 1 & 0 \\ 
0 & 0 & 0 & 1 \\ 
0 & 1 & 0 & 2
\end{array} \right)
\left( \begin{array}{c}
1_n^i \\ 
1_n^p \\ 
3_n^i \\ 
3_n^p
\end{array} \right) $ \,\ , \,\
avec \,\ $M=\left( \begin{array}{cccc}
1 & 0 & 2 & 0 \\ 
0 & 0 & 1 & 0 \\ 
0 & 0 & 0 & 1 \\ 
0 & 1 & 0 & 2
\end{array} \right). $ \\
\subsection{Polyn\^ome caractéristique}
Le polyn\^ome caractéristique de M est le suivant: \[P_M(\lambda)=(\lambda -1)(\lambda^3-2\lambda^2-1) \]
\subsection{Valeurs propres}
\par Il est claire que 1 est une racine de $P_M$, on applique le théorème  de cardan énoncé  dans l'Annexe B, au polyn\^ome $P(\lambda)=\lambda^3-2\lambda^2-1$ pour trouver ses racines, afin d'avoir toutes les racines de $P_M$, qui sont les valeurs propres de M.\\
\par On a D=59, donc on est dans le cas où D>0, donc les racines de P sont: \\
 $\begin{flushleft}
\lambda_1=(\dfrac{43}{54}+\dfrac{\sqrt{177}}{18})^{\dfrac{1}{3}}+(\dfrac{43}{54}-\dfrac{\sqrt{177}}{18})^{\dfrac{1}{3}}+\dfrac{2}{3} $\\
${\tiny \lambda_2= -\dfrac{1}{2}\left[ \left(\dfrac{43}{54}+\dfrac{\sqrt{177}}{18}\right)^{\dfrac{1}{3}}+\left(\dfrac{43}{54}-\dfrac{\sqrt{177}}{18}\right)^{\dfrac{1}{3}}\right]+i\dfrac{\sqrt{3}}{2} \left[ \left(\dfrac{43}{54}+\dfrac{\sqrt{177}}{18}\right)^{\dfrac{1}{3}}-\left(\dfrac{43}{54}-\dfrac{\sqrt{177}}{18}\right)^{\dfrac{1}{3}}\right]+\dfrac{2}{3}}$\\
${\small \lambda_3=-\dfrac{1}{2}\left[ \left(\dfrac{43}{54}+\dfrac{\sqrt{177}}{18}\right)^{\dfrac{1}{3}}+\left(\dfrac{43}{54}-\dfrac{\sqrt{177}}{18}\right)^{\dfrac{1}{3}}\right]-i\dfrac{\sqrt{3}}{2} \left[ \left(\dfrac{43}{54}+\dfrac{\sqrt{177}}{18}\right)^{\dfrac{1}{3}}-\left(\dfrac{43}{54}-\dfrac{\sqrt{177}}{18}\right)^{\dfrac{1}{3}}\right]+\dfrac{2}{3}}$
\end{flushleft} \\
\par En valeurs approchées on a:\\
\begin{center}
$\lambda_1\simeq2,205569$\\
$|\lambda_2|=|\lambda_3|\simeq0,67 $\\
\end{center}
D'où l'ensemble des valeurs propres, le spectre, de M est: $S_p(M)=\{ \lambda_1, \lambda_2, \lambda_3, 1 \}$.
\subsection{Vecteur propre}
On cherche un vecteur propre $\overrightarrow{V_1}=\left( \begin{array}{c}
x \\ 
y \\ 
z \\ 
t
\end{array} \right)$, associé à la valeur propre  $\lambda_1$,\\
 donc $\overrightarrow{V_1}$ est tel que $M\overrightarrow{V_1}=\lambda_1\overrightarrow{V_1}$, ce qui est équivalent au système suivant: \\
\begin{center}
$(S): \left\{ \begin{array}{rcl}
x+2z & = & \lambda_1 x \\
z & = & \lambda_1 y \\
t & = & \lambda_1 z \\
y+2t&= & \lambda_1 t
\end{array} $
\end{center}\\
pour $y =1$, on a $z=\lambda_1$ donc $t=\lambda_1^2$ et alors $x=\dfrac{-2\lambda_1}{1-\lambda_1}$, avec $1+2\lambda_1^2=\lambda_1^3$,\\
 d'où $\overrightarrow{V_1}=\left(\begin{array}{c}
\dfrac{2\lambda_1}{\lambda_1-1} \\ 
1 \\ 
 \lambda_1\\ 
\lambda_1^2
\end{array} \right)$ est solution du système (S).\\
 Donc $\overrightarrow{V_1}$ est un vecteur propre associé à la valeur propre $\lambda_1$.
\subsection{Densité de 3}
Soit $\overrightarrow{X_0}=\left( \begin{array}{c}
1 \\ 
0 \\ 
0 \\ 
1
\end{array} \right)$, on pose pour tout n entier naturel: $\overrightarrow{X_{n+1}}=M\overrightarrow{X_n}$.\\
Soient $\overrightarrow{V_1}$, $\overrightarrow{V_2}$, $\overrightarrow{V_3}$, et $\overrightarrow{V_4}$ respectivement les vecteurs propres associés aux valeurs propres $\lambda_1$, $\lambda_2$, $\lambda_3$ et $1$.\\
Alors il existe des constantes a, b, c et d réelles telles que $$\overrightarrow{X_0}=a\overrightarrow{V_1}+b\overrightarrow{V_2}+c\overrightarrow{V_3}+d\overrightarrow{V_4}$$ 
ceci implique que $$ 
\begin{tabular}{rl}
\overrightarrow{X_1}=M\overrightarrow{X_0}&=aM\overrightarrow{V_1}+bM\overrightarrow{V_2}+cM\overrightarrow{V_3}+dM\overrightarrow{V_4} \\
&=a\lambda_1\overrightarrow{V_1}+b\lambda_2\overrightarrow{V_2}+c\lambda_3\overrightarrow{V_3}+d\overrightarrow{V_4}
\end{tabular}$$
donc $$
\begin{tabular}{rl}
\overrightarrow{X_2}=M\overrightarrow{X_1} & =a\lambda_1M\overrightarrow{V_1}+b\lambda_2M\overrightarrow{V_2}+c\lambda_3M\overrightarrow{V_3}+dM\overrightarrow{V_4} \\
&=a\lambda_1^2\overrightarrow{V_1}+b\lambda_2^2\overrightarrow{V_2}+c\lambda_3^2\overrightarrow{V_3}+d\overrightarrow{V_4}.
\end{tabular}$$
de proche on proche, en multipliant par $M$ à gauche les deux membres, on obtient 
$$\overrightarrow{X_n}=M^n\overrightarrow{X_0}=a\lambda_1^n\overrightarrow{V_1}+b\lambda_2^n\overrightarrow{V_2}+c\lambda_3^n\overrightarrow{V_3}+d\overrightarrow{V_4}$$
Quand n tend vers l'infini, puisque $\lambda_1$ est la seule valeur propre de module strictement supérieur à 1, on aura $$\overrightarrow{X_n}\underset{n\rightarrow + \infty}{\sim}a\lambda_1^n\overrightarrow{V_1}.$$
c'est-a-dire $$\overrightarrow{X_n}\underset{n\rightarrow + \infty}{\sim}a\lambda_1^n\left(
\begin{array}{c}
\dfrac{-2\lambda_1}{1-\lambda_1} \\ 
1 \\ 
 \lambda_1\\ 
\lambda_1^2
\end{array} \right)$$ 
Dans $\overrightarrow{X_n}$ on peut calculer $1_n^i$, $1_n^p$, $3_n^i$ et $3_n^p$. \\ D'où on peut calculer la densité de 1 ou de 3, celle de 3 est donnée par:

\begin{center}
 \begin{tabular}{rl}
$\rho_3=$ & $\dfrac{\lambda_1+\lambda_1^2}{\dfrac{-2\lambda_1}{1-\lambda_1}+1+\lambda_1+\lambda_1^2}$ \\
$=$ & $\dfrac{(1-\lambda_1)(\lambda_1+\lambda_1^2)}{-2\lambda_1+(1-\lambda_1)(1+\lambda_1+\lambda_1^2)}$ \\
\end{tabular}
 \end{center} or $$-2\lambda_1^2=1-\lambda_1^3=(1-\lambda_1)(1+\lambda_1+\lambda_1^2) $$ donc la densité devient $$\rho_3=\dfrac{(1-\lambda_1)(\lambda_1+\lambda_1^2)}{-2(\lambda_1+\lambda_1^2)}$$ en simplifiant par $\lambda_1+\lambda_1^2$, on aura
 $$\fbox{\rho_3=\dfrac{\lambda_1-1}{2}}$$
en valeur approximative on a
 $$\rho_3\simeq0,6027847152$$
\section*{}
\section{Méthode de récurrence}
\subsection{Principe}
Le principe sur lequel on se base dans  cette méthode,
 consiste à établir une relation de récurrence entre 
 $\rho_n ,\rho_{S_n}$ et $\rho_{S_{S_n}}=\rho_{S_{2,n}}$. 
 Puis puisque  $(\rho_{S_n})_n$ et $(\rho_{S_{2,n}})_n$
  sont deux suites extraites de $(\rho_n)_n$, 
  alors si elles convergent, elles convergent vers la 
  m\^eme limite. Et donc la limite de la densité est 
  solution d'une equation à résoudre.
\subsection{Démonstration}
Soit $n \in N^*$, 
on a $S_n=1_{S_n}^i+1_{S_n}^p+3_{S_n}^i+3_{S_n}^p $, 
en appliquant les transformations à l'aide de la matrice de passage, on trouve\\
 $$S_n=1_n^i+1_n^p+3_n^i+3_n^p+2 \times (3_n^i+3_n^p) \\
 =n+2n \rho_n\\ =n(1+2 \rho_n).$$ \\
D'où $$S_n=n(1+2 \rho_n).$$\\
On remplaçant n par $S_n$, on aura $$ S_{2,n}=S_n(1+2 \rho_{S_n}),$$\\
et comme, par construction de K, $$ S_{2,n}=\sum_{j=1}^n K_j(2+(-1)^j), $$ 
donc $$(1+2 \rho_{S_n})S_n= 2S_n+ \sum_{j=1}^n(-1)^jK_j,$$ \\
donc $$ 
 (2\rho_{S_n}-1)S_n =\sum_{j=1}^n(-1)^jK_j
 =\sum_{j=1,K_j=3}^n(-1)^jK_j+\sum_{j=1,K_j=1}^n(-1)^jK_j
 =3(3_n^p-3_n^i)+(1_n^p-1_n^i)
$$ \\  
Or, si on fait le bilan des indices dans $K^n$, on trouve $$ 3_n^p+1_n^p=3_n^i+1_n^i+\dfrac{(-1)^n-1}{2}$$ \\ donc 
$$(2\rho_{S_n}-1)S_n=2(3_n^p-3_n^i)+\frac{(-1)^n-1}{2},$$\\ 
et si l'on remplace $n$ par $S_n$, on obtient $$(2\rho_{S_{2,n}}-1)S_{2,n}=2(2 \times 3_n^p+1_n^p-3_n^p)+ \frac{(-1)^n-1}{2}=2(3_n^p+1_n^p)+\frac{(-1)^n-1}{2},$$
Or $$3_n^p+1_n^p=3_n^i+1_n^i+\frac{(-1)^n-1}{2}=\frac{1}{2}(n+\frac{(-1)^n-1}{2})=\frac{n}{2}+\frac{(-1)^n-1}{4}$$\\ 
donc $$(2\rho_S_{2,n}-1)S_{2,n}=n+\frac{(-1)^n-1}{2}+\frac{(-1)^n-1}{2}=n+(-1)^n-1.$$ 
ou bien, sachant que $S_{2,n}=(1+2\rho_{S_n})(1+2\rho_n),$
\begin{equation}
(2 \rho_{S_{2,n}}-1)(1+2\rho_{S_n})(1+2\rho_n)=1+\frac{(-1)^n-1}{n}.\end{equation}
\par On pose: $u_n= \frac{(-1)^n-1}{n}$, $\forall n \geq 1$. Comme $\forall k \geq 1: u_{2k}=0$ \,\ et \,\ $ u_{2k+1}=\frac{-2}{2k+1}\underset{k\rightarrow + \infty}{\longrightarrow} 0$.\\
 donc $ u_n\underset{n \longmapsto + \infty}{\longrightarrow} 0$.\\
\par On suppose que $(\rho_n)_n$ converge, soit alors $l=\underset{n\rightarrow + \infty}{l i m}\rho_n$, \\
 donc 
$\underset{n\rightarrow + \infty}{l i m}\rho_{S_n}=\underset{n\rightarrow + \infty}{l i m}\rho_{S_{2,n}}=l$, car $(\rho_{S_n})_n$ et $(\rho_{S_{2,n}})_n$ sont deux sous suites de $(\rho_n)_n$.\\ 
\par En faisant tendre n vers l'infini dans la formule $(2.1)$, on obtient
\[(2l-1)(1+2l)^2=1\]
\par Ceci est équivalent, après un développement et simplification, à
\[(2l)^3+(2l)^2-(2l)-2=0.\]
\par On pose alors $L=2l$. donc on aura
\begin{equation}
L^3+L^2-L-2=0.
\end{equation}
\par Pour résoudre l'équation (2.2) on utilise le théorème énoncé dans l'Annexe B.
\\ On a D=59>0, donc il y a deux racines complexes et une racine réelle. \\ D'où la densité est 
$\rho_3=l=\dfrac{L}{2}$, où L est la racine réelle de l'équation (2.2). \\ \\ \\
Finalement 
$$\fbox{\rho_3=\dfrac{1}{2} 
\left[(\dfrac{43}{54}+\dfrac{\sqrt{177}}{18})^{\dfrac{1}{3}}+(\dfrac{43}{54}-\dfrac{\sqrt{177}}{18})^{\dfrac{1}{3}}-\dfrac{1}{3}\right]}$$ \\ 
$$\rho_3\simeq0,6027847152$$
\section{Conclusion}
\par Dans ce chapitre on a traité le m\^eme problème par deux méthodes, ce qui nous permet d'évaluer le résultat obtenu, On termine par cette représentation graphique de la suite $(\rho_n)_{n \geq 1}$ qui illustre la valeur de la densité obtenue dans les deux méthodes citées.\\ 
\begin{center}
 \begin{tabular}{c} 
\includegraphics[scale=1]{KOLAKO13.jpg} \\ 
Evaluation du résultat obtenu par les deux méthodes. 
\end{tabular}
 \end{center} 
\end{chapter}
                  %conclusion
\newpage                  
\rule{1\textwidth}{0pt}\\ 
\rule{1\textwidth}{0pt}\\ 
\rule{1\textwidth}{0pt}\\ 
\rule{1\textwidth}{0pt}\\
\rule{1\textwidth}{0pt}\\
    \begin{center}
          {\Huge \textbf{Conclusion}}
    \end{center}
\rule{1\textwidth}{0pt} \\ 
\rule{1\textwidth}{0pt} \\
 \rule{1\textwidth}{0pt} \\
 \rule{1\textwidth}{0pt} \\
  \rule{1\textwidth}{0pt} \\ 
  \rule{1\textwidth}{0pt} \\
   \rule{1\textwidth}{0pt} \\ 
   \rule{1\textwidth}{0pt} \\
\addcontentsline{toc}{chapter}{Conclusion} 
\par Dans ce travail on a étudié la suite $Kol(1,3)$, et on a donné deux méthodes qui se basent sur \rule{1\textwidth}{0pt} \\ le fait que $n$ et $S_n$ ont la m\^eme parité. Cette propriété se conserve m\^eme pour un alphabet à \rule{1\textwidth}{0pt} \\ deux lettres impaires, ce qui facilite l'étude de la suite $Kol(2p+1,2q+1)$. Il suffit de généraliser \rule{1\textwidth}{0pt} \\ les deux méthodes employées. Pour la suite $Kol(2p,2q)$, $S_n$ est toujours pair indépendemment \rule{1\textwidth}{0pt} \\ de la parité de n, ce qui rend l'étude de cette suite très facile, on montre que les densités des \rule{1\textwidth}{0pt} \\ deux lettres sont identiques et sont égales à $0,5$. Il reste le cas où l'une des lettres est paire et \rule{1\textwidth}{0pt} \\ l'autre est impaire, qui n'a pas été résolu à cause de l'instabilité de la parité de $S_n$ par rapport \rule{1\textwidth}{0pt} \\à celle de $n$.
%annexes
\chapter*{Annexe A}
\addcontentsline{toc}{chapter}{Annexe A}
\section*{}
\begin{center}
\begin{tabular}{||c||c||}
\multicolumn{2}{l}{Tableau 1: Identifications des lettres de K et de \Delta^{-1}(K).} \\  
\hline \hline
K & \Delta^{-1}(K) \\
\hline \hline
 (1)^i &(1)^i \\ 
\hline 
 (1)^p & (3)^p\\ 
\hline 
(3)^i  & (111)=2\times (1)^i+(1)^p \\ 
\hline 
(3)^p & (333)=2\times(3)^p+(3)^i \\ 
\hline \hline
\end{tabular} 
\end{center} \\
\vspace{8\baselineskip} \\
\section*{}
\begin{center}
\begin{tabular}{||c||c||l||l||c|c|c|c||c|c|c|c||}
\multicolumn{12}{l}{Tableau 2: Comment les lettres de K se transforment par $\Delta^{-1}$.} \\
\hline \hline 
n & $S_n$ & $K^n=K_1...K_n$ & $K^{S_n}=K_1...K_{S_n}$ & $1_n^i$ & $1_n^p$ & $3_n^i$ & $3_n^p$ & $1_{S_n}^i$ & $1_{S_n}^p$ & $3_{S_n}^i$ & $3_{S_n}^p$ \\ 
\hline \hline
1 & 1 & 1 & 1 & 1 & 0 & 0 & 0 & 1 & 0 & 0 & 0 \\ 
\hline 
2 & 4 & 13 & 1333 & 1 & 0 & 0 & 1 & 1 & 0 & 1 & 2 \\ 
\hline 
3 & 7 & 133 & 1333111 & 1 & 0 & 1 & 1 & 3 & 1 & 1 & 2 \\ 
\hline 
4 & 10 & 1333 & 1333111333 & 1 & 0 & 1 & 2 & 3 & 1 & 2 & 4 \\ 
\hline 
5 & 11 & 13331 & 13331113331 & 2 & 0 & 1 & 2 & 3 & 1 & 2 & 4 \\ 
\hline 
6 & 12 & 133311 & 133311133313 & 2 & 1 & 1 & 2 & 4 & 1 & 2 & 5 \\ 
\hline 
7 & 13 & 1333111 & 1333111333131 & 3 & 1 & 1 & 2 & 5 & 1 & 2 & 5 \\ 
\hline \hline
\end{tabular} 
\end{center}
\chapter*{Annexe B}
\addcontentsline{toc}{chapter}{Annexe B}
\section*{Théorème de Cardan}
 Soit $$ P(X)=X^3+bX^2+cX+d $$ un polyn\^ome à coefficients réels, et soit \\ 
  \begin{center}
\( p=c-\frac{b^2}{3}\) \hspace{1,5\baselineskip} et \hspace{1,5\baselineskip} \(q=d-\frac{bc}{3}+\frac{2b^3}{27}\).
\end{center} \\
On désigne par $\Delta$ le discriminant de  
\[R(T)=T^2+qT-\frac{p^3}{27}.\]\\
On pose \begin{center}
$D=27\Delta=4p^3+27q^2.$
\end{center} \\
Alors: \\
$_*$ Si D > 0, R possède deux racines réelles distinctes t et t'.\\ Alors les racines de P sont:\\
 \begin{center}
 $t^{\frac{1}{3}}+{t'}^{\frac{1}{3}}-\frac{b}{3}$ \hspace{2\baselineskip},\hspace{2\baselineskip}  $j^2t^{\frac{1}{3}}+j {t'}^{\frac{1}{3}}-\frac{b}{3}$ \hspace{2\baselineskip},\hspace{2\baselineskip} $j+t^{\frac{1}{3}}+j^2{t'}^{\frac{1}{3}}-\frac{b}{3}.$ 
  \end{center} \\
 $_*$ Si D < 0, R admet deux racines complexes conjuguées x et $\overline{x}$.\\ soit t et $\overline{t}$ tels que: $x={t}^{\frac{1}{3}}$ et $\overline{x}={\overline{t}}^{\frac{1}{3}}$.\\
 Alors les racines de P sont:\\
 \begin{center}
 $x+\overline{x}-\frac{b}{3}$\hspace{1,5\baselineskip} ,\hspace{1,5\baselineskip} $jx+j^2 \overline{x}-\frac{b}{3}$ \hspace{1,5\baselineskip} , \hspace{1,5\baselineskip} $j^2x+j\overline{x}-\frac{b}{3}$.
 \end{center}
 $_*$ Si D=0, R à une racine double $t_0=-\frac{q}{2}$.
 \\ Alors les racines de P sont:\\
 \begin{center}
 $2{t_0}^{\frac{1}{3}}$ \hspace{1,5\baselineskip} , \hspace{1,5\baselineskip} $-t_0^{\frac{1}{3}}$
 \end{center} \\
  N.B: \hspace{2\baselineskip} $j=-\dfrac{1}{2}+i\dfrac{\sqrt{3}}{2}$
\end{titlepage}
\end{document}