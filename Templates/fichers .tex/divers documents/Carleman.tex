\documentclass{beamer}
\usepackage[francais]{babel}
\usepackage[utf8]{inputenc}
\usepackage[T1]{fontenc}
%\usepackage{fontspec}
\usepackage{amsmath}
\usepackage{amsfonts}
\usepackage{amssymb}
\usepackage{amsthm}
\usepackage{makeidx}
\usepackage{gensymb}
\usepackage{graphicx}
\usepackage{hyperref}
\usepackage{cancel}
\usepackage{siunitx}
\usepackage{blindtext}
\usepackage{xcolor}
\usepackage{cite}
%\usepackage{kpfonts}
%\usepackage{fourier}
\everymath{\displaystyle}

\setbeamertemplate{page number in head/foot}[totalframenumber]

\mode<presentation>

  \usetheme{Warsaw}      % or try Darmstadt, Madrid, Warsaw, ...\textbf{\(\(\(\)\)\)}}

  %\usecolortheme{default} % or try albatross, beaver, crane, ...
 % \usefonttheme{default}  % or try serif, structurebold, ...
  %\setbeamertemplate{navigation symbols}{}
  %\setbeamertemplate{caption}[numbered]
%\usepackage{ragged2e}

\usefonttheme{professionalfonts} % using non standard fonts for beamer
%\usefonttheme{serif} % default family is serif
%\setmainfont{Liberation Serif}
%------------------------------raccourcis pour les notations---------------------------
\newcommand{\Cn}{\ensuremath{\mathbb{C}^n}}
\newcommand{\Cnn}{\ensuremath{\mathbb{C}^{n+n'}}}
\newcommand{\Cnm}{\ensuremath{\mathbb{C}^{n+m}}}
\newcommand{\Cnmm}{\ensuremath{\mathbb{C}^{n+2m}}}
\newcommand{\C}{\ensuremath{\mathbb{C}}}
\newcommand{\Z}{\ensuremath{\mathbb{Z}}}
\newcommand{\N}{\ensuremath{\mathbb{N}}}
\newcommand{\Np}{\ensuremath{\mathbb{N}^p}}
\newcommand{\Nnn}{\ensuremath{\mathbb{N}^{n'}}}
\newcommand{\Nq}{\ensuremath{\mathbb{N}^q}}
\newcommand{\Cx}{\ensuremath{\mathbb{C} \{x\}}}
\newcommand{\Cfx}{\ensuremath{\mathbb{C} [[x]] }}
\newcommand{\Nn}{\ensuremath{\mathbb{N}^n}}
\newcommand{\Zn}{\ensuremath{\mathbb{Z}^n}}
\newcommand{\Zp}{\ensuremath{\mathbb{Z}^p}}
\newcommand{\M}{\ensuremath{\mathbb{R}}}
\newcommand{\Mpp}{\ensuremath{\mathbb{R}^p}}
\newcommand{\Mq}{\ensuremath{\mathbb{R}^q}}
\newcommand{\Mr}{\ensuremath{\mathbb{R}^r}}
\newcommand{\Mn}{\ensuremath{\mathbb{R}^n}}
\newcommand{\Mpq}{\ensuremath{\mathbb{R}^p\times \mathbb{R}^q}}
\newcommand{\ZpNq}{\ensuremath{\mathbb{Z}^p\times \mathbb{N}^q}}
\newcommand{\Me}{\ensuremath{\mathbb{R}^*}}
\newcommand{\Mp}{\ensuremath{{\mathbb{R}}_+}}
\newcommand{\Mep}{\ensuremath{{\mathbb{R}}_+^*}}
\newcommand{\xn}{\ensuremath{(x_1,\cdots ,x_n)}}
\newcommand{\xp}{\ensuremath{(x_1,\cdots ,x_p)}}
\newcommand{\xr}{\ensuremath{(x_1,\cdots ,x_r)}}
\newcommand{\yq}{\ensuremath{(y_1,\cdots ,y_q)}}
\newcommand{\alf}{\ensuremath{(\alpha_1,\cdots ,\alpha_n)}}
\newcommand{\fr}{\ensuremath{\varphi_R}}
\newcommand{\frx}{\ensuremath{\varphi_R(\xi .x)}}
\newcommand{\Brx}{\ensuremath{B_R(\xi)}}
\newcommand{\Br}{\ensuremath{B_R}}



%---------------------------------First page ----------------------------------
\institute{
\textbf{Université de Ghardaia\\
Département de Mathématiques \\
et de l'Informatiques}
}

\title[]{Problème de Goursat non-linéaire dans un espace de Carleman}
\author{Présenté par:  Abdelhakim Dahmani }
\date{\today}
\subject{Soutenance de Master}
%---------------------------------------------------------------------------










%______________________frame title _______________________________________________
\iffalse
\addtobeamertemplate{frametitle}{
   \let\insertframetitle\insertsectionhead}{}
\addtobeamertemplate{frametitle}{
   \let\insertframesubtitle\insertsubsectionhead}{}


\makeatletter
  \CheckCommand*\beamer@checkframetitle{\@ifnextchar\bgroup\beamer@inlineframetitle{}}
  \renewcommand*\beamer@checkframetitle{\global\let\beamer@frametitle\relax\@ifnextchar\bgroup\beamer@inlineframetitle{}}
\makeatother
\fi

%_____________________________________________________________________________________
\begin{document}
%----------------------------Pour inverser l'ordre de instution et auteur ------------
    \begin{frame}[plain]
    % \vfill
     \centering
     \begin{beamercolorbox}[sep=8pt,center,colsep=-4bp,rounded=true,shadow=true]{institute}
        \usebeamerfont{institute}\insertinstitute
     \end{beamercolorbox}
     {\usebeamercolor[fg]{titlegraphic}\inserttitlegraphic\par}
     \begin{beamercolorbox}[sep=8pt,center,colsep=-4bp,rounded=true,shadow=true]{title}
        \usebeamerfont{title}\inserttitle\par%
        \ifx\insertsubtitle\@empty%
        \else%
        \vskip0.25em%
        {\usebeamerfont{subtitle}\usebeamercolor[fg]{subtitle}\insertsubtitle\par}%
      \fi%     
     \end{beamercolorbox}%
     \vskip1em\par
     \begin{beamercolorbox}[sep=8pt,center,colsep=-4bp,rounded=true,shadow=true]{author}
        \usebeamerfont{author}\insertauthor
     \end{beamercolorbox}
     \begin{beamercolorbox}[sep=8pt,center,colsep=-4bp,rounded=true,shadow=true]{date}
        \usebeamerfont{date}\insertdate
     \end{beamercolorbox}\vskip0.5em
  \end{frame}
%---------------------------------------------------------------------------------------------

\begin{frame}
\frametitle{Plan de Présentation}
\tableofcontents
\end{frame}

%_______________________________________Carleman____________________________________________________
\section{Inroduction et Motivations}

\begin{frame}{\secname : \subsecname}

L'utilité des équations aux dérivées partielles pour les problèmes de la vie réelle dans les différents disciplines leurs rendre un domaine de recherche très actif, mais aussi très vaste, où on en trouve l'étude de l'existence, l'unicité, la stabilité, les traitements numériques et aussi bien d'autres sujets qui se choisissent selon les besoin rencontrés. Dans cet exposé on s'intéresse à un résultat d'existence et unicité de solution pour le problème de Goursat (problème de Cauchy à caractéristique) non-linéaire dans un espace de Carleman, un résultat purement théorique, qui s'y  développait comme suit:

%Dans la littérature on trouve assez des téchniques et méthodes pour montrer l'existence et l'unicité des solutions pour les différents types d'équations aux dérivées partielles, on s'intérese55 au problème de Goursat non-linéaire et à la méthode de Claude Wagschal qui a été développée comme suit: 
\end{frame}

\begin{frame}
\begin{itemize}
\item Louis Augestin Cauchy 
\item Sophia kowalowskia 
\item Edouard Goursat 
\item Andreevich Nikolai Lednev  	 
\item Claude Wagschal\\
 Ce dernier a arrivé à montrer l'existence et l'unicité pour le probleme de Goursat non-linéaire suivant 
 %dans un espace de Gevrey-$d$
\end{itemize}
\begin{equation*}
 \left\{
 \begin{array}{r c l}
 D^\alpha_x u(x,y)&=& f(x,y,D^Bu(x,y)),\\
 u&=&O(x^\alpha)
 \end{array}
 \right.
\end{equation*}

$ \alpha \in  \Np$, $ B $  est une partie finie de $$ \{ ( \gamma , \delta) \in \ZpNq ; |\gamma| +d|\delta|\leq | \alpha| \text{ et } \gamma < \alpha\}$$


\end{frame}



\begin{frame}
Dans un espace de Gevrey-$d$ définie comme suit:\\ 
\medskip
Soit $ \Omega $ un ouvert de $ \Mpq $, $\alpha \in \Np $ et $d \geq 1 $, on note $G^{\alpha , d}(\Omega)$ l'espace de fonctions $ u : \Omega \rightarrow R$ qui admettent des dérivées partielles continues $$D^\gamma_x D^\delta_yu: \Omega \rightarrow R$$ 
telles que 
$$ \sup_\Omega | D^\gamma_x D^\delta_yu| \leq c^{|\delta|+}\delta!^d \quad \forall \gamma \leq \alpha \text{ et } \forall \delta$$
\textcolor{red}{Existe-t-il des espaces plus large où ce résultat reste valable? }
\end{frame}

\begin{frame}
La réponse est \textcolor{red}{Oui}\\
 \bigskip
On va introduire l'espace de Carleman où on va établir notre résultat 
\end{frame}
\section{Problème de Goursat Carleman}

\subsection{Hypothèses et résultats}


\begin{frame}
\frametitle{L'espace de Carleman $C^{\alpha,M}$ }
Soit $ \Omega $ un ouvert de $ \Mpq $ , $ (M_n)_{n \in \N} $  une suite croissante positive et $\alpha \in \Np $\\

$ C^{\alpha,M}(\Omega)$ :l'espace des fonctions $u:\Omega \rightarrow \M $, admettant $\forall \gamma \in \Np,\  \gamma \leq \alpha$ et $\forall \delta \in \Nq$ des dérivées partielles continues $$ D^\gamma_xD^\delta_yu:\Omega \rightarrow \M$$

et $\exists c >0$
 
 $$ \sup_\Omega |D^\gamma_xD^\delta_yu(x,y)| \leq c^{|\delta|+1} \delta ! M_{|\delta|}, \quad  \forall \gamma \leq \alpha \text{ et } \forall \delta \in \Nq $$
\end{frame}

\begin{frame}
La suite $(M_n)_{n \in \N}$ considérée doit satisfaire les hypothèses suivantes:
\begin{enumerate}
\item 
$ M=(M_n)_{n \in \N} $ logarithmiquement convexe
\begin{equation*}
i.e.\ M_k^2 \leq M_{k-1} M_{k+1} \quad \forall k \in \N
\end{equation*}
et $M_0=1$
\item 
\begin{equation*}
\sup_k (  \frac{M_{k+1}}{M_k})^\frac{1}{k} < \infty, \quad k \in \N
\end{equation*}
\end{enumerate}

\end{frame}




%___________________________________________________________________________________________
\begin{frame}
On considère  alors le problème de Goursat non linéaire au voisinage  de l'origine de $\Mpq$

\begin{equation}\label{goursatcarleman}
 \left\{
 \begin{array}{c}
 D^\alpha_x u(x,y)= f(x,y,D^Bu(x,y)),\\
 u=O(x^\alpha)
 \end{array}
 \right.
\end{equation}

$ \alpha \in  \Np$, $ B $  est une partie finie de $$ \{ ( \gamma , \delta) \in \ZpNq ; |\gamma| +|\delta|\leq | \alpha| \text{ et } \gamma < \alpha\}$$

Où $f$ est une fonction de  classe $C^{0,M }$ au voisinage de l'origine de $ \mathbb{R}^p \times \mathbb{R}^{q+r}$ \\
$u=O(x^\alpha) $ signifie que $  u(x,y)=x^\alpha g(x,y)$, où $g$ est de  classe $C^{\alpha,M }$ au voisinage de l'origine ce qui implique que $ D^\beta u(0)=0$  pour tout $ \beta \in B$
\end{frame}


\begin{frame}
\begin{block}{Théorème}

S'il existe une constante $ c_{\gamma, \delta} \geq 0$ telle que :
\begin{equation*}
\frac{M_{k+|\delta|}}{M_k}  \frac{(k+|\delta|)!}{ (k-|\gamma|+|\alpha|)!}  \leq c_{\gamma, \delta}    \quad \forall k \geq 0 
\end{equation*} 
 Alors le problème de Goursat (\ref{goursatcarleman}) admet une solution unique au voisinage de l'origine de $\Mpq$
\end{block}

\begin{block}{Remarque}
Lorsque
\begin{itemize}
\item $(M_n)_{n \in \N}$ ou bien $ (\frac{M_{n+1}}{M_n})_{n \in \N}$ est bornée, on y dans le cas holomorphe.
\item $M_n=n!^{d-1}$, on y dans le cas Gevrey-$d$
\end{itemize} auxquelles ces cas sont traités par C.Wagschal et notre travail les généralise
\end{block}

\end{frame}

\begin{frame}
 Pour prouver ce théorème on peut supposer $ \alpha=0$ et $f(0,0)=0$, il s'agit donc d'étudier le problème 
 \begin{equation*}
 u(x,y)=f(x,y,D^Bu(x,y))
\end{equation*}

et $B$ est un sous ensemble finie de 

 $$\{ ( \gamma , \delta) \in \ZpNq ; |\gamma| +|\delta|\leq 0, \gamma < 0\  \}$$
 Pour ceci on va vérifier que l'application 
$$ T: u \rightarrow f(x,y,D^Bu(x,y))$$ 
est une contraction stricte dans une boule fermée d'une algèbre de Banach.\\
Une telle algèbre de Banach sera définie par certaine série formelle.

\end{frame}
\subsection{Séries Formelles et espace associe}

\begin{frame}
On construit maintenant l'espace de travail. Soit la série formelle 

$$ \Phi = \sum_{ \delta \in \Nq} \frac{y^\delta}{\delta !} \Phi_\delta, \quad \Phi_\delta \in \Mp$$
 et une fonction $ u \in \mathcal{C}^\infty ( \Omega, \mathbb{R})$.\\
 
 On note $ u \ll \Phi $ la relation :
  
  $$ \forall \delta \in \Nq , \quad \sup_\Omega | D^\delta _y u | \leq \Phi_\delta $$ 


\end{frame}

\begin{frame}
Pour $\xi \in (\Mep)^p,  \zeta \in (\Mep)^q , R>0$\\
Soit maintenant la série 
\begin{equation*}
\Phi_R^M(t,y)=\sum_{k=0}^\infty \frac{(\zeta.y)^k}{k!}M_kD^k\phi_R(\xi.t) 
\end{equation*} 
avec $ \varphi_R(\xi.t)= K^{-1}\sum_{n=0}^\infty  \frac{(\xi.t)^n}{R^n(n+1)^2}$\\
Alors l'algèbre de Banach associe à la série $\Phi_R^M$ définie par:  
$$C_R^M(\Omega_R)= \{ u \in   \mathcal{C}^{0,\infty} (\Omega_R ); \exists c \geq 0 : u \ll c \Phi_R^M \}$$  
muni de la norme 
$$ \|u\|= \min \{ c \geq 0 ; u \ll c \Phi_R^M \}$$
\end{frame}


\begin{frame}
Indiquons maintenant les relations entre  $C^{0,M}$ et $  C_R^M$

\begin{block}{Lemme}
Pour $0<R'<R$ on a $$ C_R^M(\Omega_R) \subset C^{0,M}(\Omega_{R'})$$
\end{block}


Pour établir l'inclusion inverse, on utilise la série formelle $ \Theta^M_R$ définie par:
 \begin{equation*}
 \Theta^M_R(t)=\sum_{k=0}^\infty \frac{t^k}{R^k}M_k
 \end{equation*}

\end{frame}

\begin{frame}

\begin{block}{Lemme}
Pour tout $ \eta >1 $, il existe  une constante $c=c(\eta) >0$ telle que 

$$\Theta^M_{\eta R} (\zeta.y) \ll c  \Phi^M_R(t,y)  $$
\end{block}
grâce à ce lemme on a:


\begin{block}{Lemme}
 Si  $c\eta R  \leq \min_{1\leq i \leq p} \zeta_i$; Alors $$ C^{0,M}(\Omega_R) \subset  C_R^M (\Omega_R)$$
\end{block}
Ce qui signifie que les deux espaces $C^{0,M}$ et $  C_R^M$ sont presque le même
\end{frame}


\begin{frame}
%\begin{block}{Proposition}
%L'espace $C^{0,M}(\Omega_R)$ est stable par derivation par rapport à la deuxiemme variable $y$ si et seulement si $\sup_k \left(  \frac{M_{k+1}}{M_k}\right)^\frac{1}{k} < \infty, \quad \forall k \in \N$
%\end{block}
\begin{block}{Proposition}
Pour tout $(\gamma ,\delta) \in B$  tel que $ \frac{M_{k+|\delta|}}{M_k}  \frac{(k+|\delta|)!}{ (k-|\gamma|)!}  \leq c_{\gamma, \delta}$, l'opérateur $ D_x^\gamma D_y^\delta : C^M_R\rightarrow C^M_R $  est lineaire continue et  
$\| D_x^\gamma D_y^\delta\|\leq c_{\gamma ,\delta} \xi^\gamma\zeta^\delta R^{-|\gamma|-|\delta|}$
\end{block}

\end{frame}













%_________________________________________________________________

\begin{frame}
\begin{block}{Lemme}
la fonction $f$ puisse se décomposée de la forme
 $$f(x,y,z)=f(x,y,z')+ \sum_{\sigma \in B} G_\sigma (x,y,z,z') (z_\sigma - z'_\sigma)$$
 telle que les fonctions $G_\sigma, \sigma \in B$  sont de classe $C^{0,M }$ au voisinage de l'origine de $ \mathbb{R}^p \times \mathbb{R}^{q+2r}$ 

\end{block}

d'autre part on a: Il existe un $R_0>0$ tel que pour tout $R\in ]0,R_0]$ on ait

\begin{equation*}
\left\{
\begin{array}{l}
f(x,y,z) \ll c \Phi^M_R(t,y) \prod_{\sigma \in B} \Theta^M_{R'}(z_\sigma)\\
G_\sigma(x,y,z,z') \ll c\Phi^M_R(t,y) \prod_{\nu \in B} \Theta^M_{R'}(z_\nu) \Theta^M_{R'}(z'_\nu)
\end{array}
\right.
\end{equation*}

\end{frame}
\subsection{Idées de Démonstration}
\begin{frame}

\begin{block}{Proposition}
Il existe $a_0>0$ tel que, pour tout $ a\geq a_0$ et tout $ R \in ]0,R_0]$ sufisamment petit, l'application $$T:u\rightarrow f(x,y,D^B(x,y))$$ soit une contraction stricte dans la boule fermée $B'(0;a)$ de l'algèbre de Banach $C^M_R(\Omega_R)$
\end{block}
 Les grandes lignes de la démonstration sont les suivantes:
\end{frame}

\begin{frame}
\begin{itemize}
 \item[1] Pour que $T(B'(0,a)) \subset B'(0,a) $
  \end{itemize}

 \begin{itemize}
 \item Assurer que $Tu = f(x, D^B u(x))$ est bien définie pour $u \in B'(0;a)$ c'est-à-dire:\\
 $\sup_\Omega|D^\gamma_xD^\delta_y u(x,y)| \leq c \quad \forall (\gamma, \delta) \in B$
 
 
 \item Majorer $Tu$ pour $u \in B'(0;a)$, on a 
 $$\Theta^M_{R'}\circ[a\varepsilon(R) \Phi^M_R] \ll K \Phi^M_R$$
  $$\text{ et } f(x,y,z) \ll c \Phi^M_R(t,y) \prod_{\sigma \in B} \Theta^M_{R'}(z_\sigma)$$
 \begin{align*}
 \text{ d'où  }Tu=f(x,y,D^Bu) & \ll c \Phi^M_R(t,y) \left( \Theta^M_{R'}\circ[a\varepsilon(R) \Phi^M_R] \right)^r \\
  & \ll c \Phi^M_R(t,y) \left(  K \Phi^M_R \right)^r \\
   & \ll c_2 \Phi^M_R(t,y)
\end{align*}  
 
 \end{itemize}
\end{frame}

\begin{frame}
Finalement $T(B'(0,a)) \subset B'(0,a) $ si $c_2 \leq a$ 
\begin{itemize}
\item[2] Pour que $T$ soit une contraction stricte
\end{itemize}

\begin{itemize}
\item Supposons que les conditions précédentes sont vérifiés, alors
$T$ est bien définie
\item Développer $f$ de sous la forme 
 $$f(x,y,z)-f(x,y,z')= \sum_{\sigma \in B} G_\sigma (x,y,z,z') (z_\sigma - z'_\sigma)$$
 \item Écrire la majoration
 $$ G_\sigma(x,y,z,z') \ll c\Phi^M_R(t,y) \prod_{\nu \in B} \Theta^M_{R'}(z_\nu) \Theta^M_{R'}(z'_\nu) $$  et  $$\Theta^M_{R'}\circ[a\varepsilon(R) \Phi^M_R] \ll K \Phi^M_R$$
\end{itemize}
\end{frame}

\begin{frame}
\begin{itemize}
\item Vérifier la contraction, soit $u, u' \in  B'(0; a)$
\begin{align*}
Tu-Tu'&= f(x,y,D^Bu)- f(x,y,D^Bu')\\
& \ll c_3\varepsilon(R) \|u-u'\| \Phi^M_R(t,y)
\end{align*}
\end{itemize}

donc $T$ est une contraction stricte si $c_3\varepsilon(R)<1$.\\
On résume.\\
pour que $T:B'(0;a)\rightarrow B'(0;a)$ soit une contraction stricte, il suffit que 

$$ a\varepsilon(R)\leq c_1, \quad c_2\leq a, \quad c_3\varepsilon(R)<1$$




\end{frame}



\begin{frame}
Pour l'unicité: \\
\begin{itemize}


\item soit $u,u'$ deux fonctions de classe $C^{0,M}$ solutions de notre problème.

\item Il existe un nombre $R_1 \in ]0,R_0]$ tels que $u,u' \in C^M_{R_1}(O_{R_1})$.

\item On choisit alors $a \geq \max(a_0,\|u\|_{R_1},\|u'\|_{R_1})$,et $R\in ]0,R_1]$ suffisamment petit pour que $T$ soit une contraction stricte dans la boule fermée $B'(0;a) \subset C^M_R(O_R)$
\item Donc $u$ et $u'$ sont deux points fixes pour la même contraction $T$, d'où ils sont identiques

\end{itemize}





\end{frame}
\section{Conclution}
\begin{frame}
Dans ce travail on a généralisé les résultats obtenus par C.Wagschal dans un espace de type Carleman associe a une suite numérique vérifiant certaines hypothèses, et on a donné une formule qui nous spécifie les ordres de dérivation qui peuvent être traité avec ce formalisme chaque fois quand on détermine la forme explicite de la suite $(M_n)_{n \in \N}$.
\end{frame}
\end{document}
