\documentclass[12pt]{article}
\usepackage[utf8]{inputenc}
\usepackage{amsmath,amsfonts,amssymb,amsthm,mathrsfs}
\usepackage[most]{tcolorbox}
\usepackage[margin=2.5cm,a4paper]{geometry}
\usepackage{graphicx,lmodern}
\usepackage[pages=some]{background}
\usepackage{mdframed,pgfplots,tikz}
\backgroundsetup{
	scale=1,
	color=black,
	opacity=1,
	angle=0,
	contents={%
		\includegraphics[width=22.5cm,height=29cm]{r1.jpg}
	}%
}
\usepackage{wallpaper}
\usepackage{polyglossia}
\setmainlanguage[numerals=maghrib]{arabic}
\setotherlanguage{french}
\newfontfamily\arabicfont[Script=Arabic,Scale=1.2]{Amiri}
\newfontfamily\arabicfonts[Script=Arabic,Scale=1.2]{Simplified Arabic}
\newfontfamily\frenchfont[Scale=1.2]{Times New Roman}
\parindent 0cm
\begin{document}
\BgThispage
\begin{titlepage}
\begin{center}
\vspace*{5mm}
\includegraphics[scale=0.2]{r2.png}
\parbox[b][3cm][t]{8cm}{\centering
جامعة محمد خيضر - بسكرة -\\[2mm] كلية العلوم الإنسانية والاجتماعية\\[3mm] قسم التربية البدنية والرياضية
}
\includegraphics[scale=0.2]{r2.png}
\vskip1cm
\textit{\centering
مذكرة تخرج ضمن متطلبات نيل شهادة الماستر في التربية البدينة والرياضية
\vskip3mm
تخصص: تربية حركية عند الطفل والمراهق
\vskip1cm
\textbf{
الموضوع:
}}
\begin{tcolorbox}[enhanced,sharp corners=uphill,
	colback=white!50!white,colframe=blue!25!black,coltext=black,
	fontupper=\Large\bfseries,arc=6mm,boxrule=2mm,boxsep=5mm,
	borderline={0.3mm}{0.3mm}{white}]\centering
	دور حصة التربية البدنية في التقليل من العنف المدرسي عند تلاميذ المرحلة المتوسطة
\end{tcolorbox}
\centering
\vskip3mm
دراسة ميدانية على مستوى متوسطات بلدية قمار.
\vskip1cm
\begin{minipage}{3cm}
من إعداد الطلبة:\\
\begin{itemize}
	\item الطالب
\end{itemize}
\end{minipage}
\hfill
\begin{minipage}{3cm}
إشراف الأستاذ:\\
\begin{itemize}
	\item الأستاذ
\end{itemize}
\end{minipage}
\vfill
\includegraphics[scale=0.4]{image2.jpg}
\vfill
\centering
السنة الجامعية:
..20/..20
\end{center}
\end{titlepage}
\end{document}