\documentclass[16pt]{beamer}
\usepackage[utf8]{inputenc}
%\inputencoding{utf8}
\usepackage{amsmath,amssymb,enumerate,epsfig,bbm,calc,color,ifthen,capt-of}
\usepackage{setspace}
\usepackage{fontspec}
\usefonttheme{serif}
\usepackage{polyglossia}
\usepackage{xcolor}
\usepackage{graphicx}
\usepackage{float}
\usepackage{tikz}
\usetikzlibrary{shapes}
\usetikzlibrary{calc}
\usepackage{listings}
\usepackage{multicol}
\usepackage{enumerate}

\usetheme{AnnArbor}
\setdefaultlanguage[numerals=maghrib,calendar=gregorian]{arabic}
\setotherlanguage{english}
\newfontfamily\arabicfont[Script=Arabic,Scale=1.2]{Amiri}
\setbeamercolor{alerted text}{fg=orange}
\setbeamercolor{background canvas}{bg=white}
\setbeamercolor{block body alerted}{bg=normal text.bg!90!black}
\setbeamercolor{blockbody}{bg=normal text.bg!90!black}
\setbeamercolor{palettesidebarsecondary} {use=structure,fg=structure.fg}
\setbeamercolor{palettesidebartertiary} {use=normal text,fg=normal text.fg}
\setbeamercolor{section in sidebar}{fg=brown}
\title[الإنزلاقات الديداكتيكية]{الإنزلاقات الديداكتيكية}
\author[المركز الجهوي لمهن التربية والتكوين بني ملال-خنيفرة]{من انجاز:\\ سعيد شاطر- محمد حجاج- عبد الحكيم بوشيار}
\date[\today] {\today}
\AtBeginSection[]
{
\begin{frame}{محاور العرض }
 \tableofcontents[currentsection]
\end{frame}
}
\begin{document}
%\setbeamertemplate{navigation symbols}{}
\begin{frame}
\begin{center}
\includegraphics[scale=0.4]{crmef}
\end{center}
\titlepage
\end{frame}
\begin{frame}\frametitle{محاور العرض }
\tableofcontents
\end{frame}
%\section*{وضعية الإنطلاق}

\section{الإنزلاقات الديداكتيكية}
\begin{frame}
\begin{block}{}
عادة ما يدرج الباحثون هذه الآثار الإنزلاقات ضمن انحرافات المدرس عن العقد الديداكتيكي المبرم صراحة أو ضمنا مع المتعلمين.
\end{block}
\pause
\begin{block}{}
اثناء حرص المدرس على تحقيق الأهداف التي سطرها لفعله الديداكتيكي، قد يسقط في بعض الانحرافات حين يقدم للتلاميذ مساعدات تسهل وصولهم الى النتائج، وبالتالي تحول دون بنائهم للتعلمات.
\end{block}
\end{frame}
\section{اثار طوباز}
\begin{frame}{}

\begin{block}{}
 يتمثل في الحالة التي يهيئ فيها المدرس أسئلة الدرس على مقاس الأجوبة التي يريد سماعها، وهكذا يضع المدرس الجواب الذي يريده، ويشرع في صياغة الأسئلة على ضوئها، لطرحها على المتعلمين.
\end{block}
\pause
\begin{block}{}
 و قد يتجلى هذا الأثر في حالات أخرى، و منها الحالة التي يقف فيها المتعلم أمام صعوبة لمواصلة حل وضعية مشكلة، و يقتضي الأمر أن يواجه تلك الصعوبة في حينها، ولكنه، عوض ذلك قد يتلقى مساعدة حاسمة من طرف المدرس، الشيء الذي يفوت عليه فرصة لبناء تعلماته و بلوغ مستوى أعلى من التعلم
\end{block}
\end{frame}
\section{اثار جوردان}

\begin{frame}{}

\begin{block}{}
 هو عبارة عن سوء تفاهم عميق، يحدث أحيانا عندما يتفادى المدرس عن قصد كل نقاش مع المتعلمين حول معلومة أو مفهوم معين، و يكتفي بتقبل أدنى مؤشر سلوكي صادر عنهم، معتبرا إياه دليلا على الاستجابة لما طلب منهم إنجازه، حتى و إن كان ذلك المؤشر عاديا و غير مقنع.
\end{block}
\pause
\begin{block}{}
 و قد يتجلى هذا الأثر أيضا عندما يعتبر المدرس أن إشارة بسيطة يبديها المتعلم، دليل على فهمه و استيعابه لما قدم له، حتى وإن كانت سلوكات وإجابات غير مقنعة وساذجة
\end{block}
\end{frame}

\section{الانزلاق الميتامعرفي}

\begin{frame}{}

\begin{block}{}
قد لا يتوفق المدرس أحيانا، في إبلاغ ما يريد إبلاغه للمتعلمين، فيعجز بالتالي، عن دفعهم نحو تحقيق الهدف المتوخى، فيلجأ (كتعويض عن فشله) إلى تبريرات متعددة، ويتحول إلى موضوعات أخرى، مستبدلا بذلك الموضوع الذي يشكل المحور الفعلي للدرس، أو قد يركز شرحه على طريقة أو تقنية معينةويتوقف عندها كبديل عن الموضوع المرغوب فيه
\end{block}
\end{frame}
\section{ الإستعمال المفرط للمماثلة}
\begin{frame}{}
\begin{block}{}
  تعتبر المماثلة من "التقنيات" الجديدة في الشرح و التفسير، إلا أن الإفراط في استعمالها قد يؤدي إلى نتيجة عكسية أو غير متوقعة.
  \end{block}
  \pause
\begin{block}{}
   و قد لاحظ الديدكتيكيون أن هذا الاستعمال الـمفرط للمماثلة على مستوى التعاقد الديدكتيكي، أمر غير مفيد، بل بالعكس، يمكن أن يفضي إلى السقوط في  أثر طوباز أو بعبارة أخرى إلى تباطؤ في الفهم وتأخر في اكتساب المعلومات.
\end{block}
\end{frame}



%\section{اثار الإنتظار الغامض}
%
%\begin{frame}{}
%
%\begin{tikzpicture}
%{Large
%\node [draw,cloud callout,brown,callout pointer start size=.1] { بعد و طيبة تحية };
%}
%\end{tikzpicture}
%
%\end{frame}

\section{شيخوخة الوضعيات}

\begin{frame}{}


\begin{block}{}
إن مرور الزمن والتغيرات المستمرة للبرامج و المناهج، قد يؤدي إلى نوع من التقادم في الوضعيات الديدكتيكية، فيصبح المدرس غير قادر على إعادة إنتاج نفس الوضعيات لتؤدي الغرض المنتظر منها.

\end{block}
\pause
\begin{block}{}
 وهذا الإحساس بالتقادم أو التقادم الفعلي، في أغلب الأحيان، يطرح إشكالية ديدكتيكية أساسية
 \end{block}
 \begin{block}{}
 خاصة إذا انتبهنا إلى أن بعض التغييرات التي تطرأ على المناهج قد لا تمليها ضرورات تربوية بقدر ما تترجم نوعا من ملائمة هذه الوضعيات لعقلية المتعلم.
\end{block}
\end{frame}
%\section*{خاتمة}
%\begin{frame}{خاتمة}
%
%\end{frame}
\section*{}
%\begin{frame}
%
%\end{frame}
\begin{frame}
\begin{center}
\begin{tikzpicture}
{Large
\node [draw,cloud callout,brown,callout pointer start size=.1] { \Large{\emph{\textbf{ \textcolor[rgb]{1.00,0.00,0.00}{ انتباهكم} على   شكرا}}}};
}
\end{tikzpicture}
\end{center}
\end{frame}
\end{document} 