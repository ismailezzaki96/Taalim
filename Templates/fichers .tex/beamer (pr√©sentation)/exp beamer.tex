\documentclass[12pt]{beamer}
\usetheme{Darmstadt}
%\usetheme{AnnArbor} 
\usepackage[utf8]{inputenc}
\usepackage[T1]{fontenc}
\usepackage[francais]{babel}
\usepackage{amsmath}
\usepackage{amsfonts}
\usepackage{amssymb}
\usepackage{graphicx}
%%%%%%%%%%%%%%%%%%%%%%%%%%%%%%%%%%%%%%%%%%%%%%%%%%%%%%%%%%%%%%%%%%%%%%%%%%%%%%%%%%%

%%%%%%%%%%%%%%%%%%%%%%%%%%%%%%%%%%%%%%%%%%%%%%%%%%%%%

%%%%%%%%%%%%%%%%%%%%%%%%%%%%%%%%%%%%%%%%%%%%%%%%%%%%%%%%%%%%%%%%%%%%%%%%%%%%%%%%%%
%..............................................1
\title[Block Partitioning and Perfect Phylogenies] 
{
  Soutenance de mémoire de fin d'étude
}
\author[Gramm, Hartman, Nierhoff, Sharan, Tantau]
{
                    intitulée\\
  \and 
  \textcolor{green}{ETUDE DE LA SUITE DE KOLAKOSKI-(1,3)}
}
\institute[T�bingen and others]
{
                    Présentée par
  \and
  \vskip-2mm
  \textcolor{blue}{Abdelhakim BOUCHIAR} et   
  \textcolor{blue}{Ismail ELMARJANI}
  \and
  \vskip-2mm
              Devant le jury composé par
  \and
  \vskip-2mm
  \textcolor{red}{Pr.Abdellah HAMMAM}
  \and
  \vskip-2mm
  \textcolor{red}{Pr.Lhassane SADDEK}
  \and
  \vskip-2mm
  \textcolor{red}{Pr.Jilali ASSIM}
}
\date[WABI 2006]
   {\textcolor{blue}{Le 07/07/2017}}
%%%%%%%%%%%%%%%%%%%%%%%%%%%%%%%%%%%%%%%%%%%%%%%%
%%%%%%%%%%%   The main document   %%%%%%%%%%%%%%
\begin{document}
\begin{frame}
  \titlepage
\end{frame}
%...............................................2
\begin{frame}{Plan de la présentation}
  \tableofcontents
\end{frame}

%================================================
\section{Terminologie}
%...............................................3
\begin{frame}[t]{Définitions}
  \begin{block}{Alphabet}
    \begin{itemize}
    \item On appelle Alphabet tout ensemble fini d'entiers naturels. 
    \end{itemize}
  \end{block}
\end{frame}
%%%%%%%%%%%%%%%%%%
\begin{frame}[t]{Définitions}
  \begin{block}{Alphabet}
    \begin{itemize}
    \item On appelle Alphabet tout ensemble fini d'entiers naturels. 
    \end{itemize}
  \end{block}
  \begin{block}{Lettre}
    \begin{itemize}
    \item Les éléments d'un alphabet s'appellent des lettres.
    \end{itemize}
  \end{block}
\end{frame}
%%%%%%%%%%%%%%%%%%
\begin{frame}[t]{Définitions}
  \begin{block}{Alphabet}
    \begin{itemize}
    \item On appelle Alphabet tout ensemble fini d'entiers naturels. 
    \end{itemize}
  \end{block}
  \begin{block}{Lettre}
    \begin{itemize}
    \item Les éléments d'un alphabet s'appellent des lettres.
    \end{itemize}
  \end{block}
  \begin{block}{Notation}
    \begin{itemize}
    \item On note $\Sigma^*$, l'ensemble des mots formés par les lettres de $\Sigma$.
    \end{itemize}
  \end{block}
\end{frame}
%...............................................4
\begin{frame}[t]{Définitions}
  \begin{block}{Mot fini}
    \begin{itemize}
    \item Soit $\Sigma$ un alphabet. Un mot fini est une suite finie d'éléments de $\Sigma$.
    \item<alert@1->
      Soit M un mot fini, alors $\exists n \in \mathbb{N}^*$ tel que $M: \{ 1,2,...,n\} \underset{i \longmapsto M_i}{\longrightarrow} \Sigma$ . On écrit $M=M_1M_2...M_n$.
    \item<alert@1->
      On dit que M est de longueur n, et on note long(M)=n.
    \end{itemize}
  \end{block}
\end{frame}
%...............................................5
\begin{frame}[t]{Définitions} 
  \begin{block}{Mot infini}
    \begin{itemize}
    \item Soit $\Sigma$ un alphabet. On appelle mot infini, toute suite infinie d'éléments de $\Sigma$.
    \item<alert@1->
      Soit K un mot infini, donc  $K: \mathbb{N}^* \underset{i \longmapsto K_i}{\longrightarrow} \Sigma $.
    \item<alert@1->
      On écrit dans ce cas $K=K_1K_2...$ .
    \end{itemize}
  \end{block}
\end{frame}
%..............................................
\begin{frame}{Définitions.}
  On considère un mot M non vide (fini ou infini).
  \begin{block}{Bloc}
    \begin{itemize}
    \item On appelle bloc, toute chaîne $M_k...M_{k+p}$ avec $k \geq 1$ et $p \geq 0$ deux entiers telle que:
  \item<alert@1->
   Si $k=1, M_1=M_2=...=M_{p+1}$ et $M_{p+1} \neq M_{p+2}$. 
  \item<alert@1->
   Si $k \geq 2$,\\
    $M_k=M_{k+1}=...=M_{k+p}$ et  \left\{ \begin{array}{l}
                                         M_{k-1} \neq M_{k} \\
                                            M_{k+p} \neq M_{k+p+1}
                                           \end{array} \left.   
  
    \end{itemize}
  \end{block}
  \end{frame}
%...............................................
\begin{frame}{Définitions.}
  \begin{block}{Exemple}
    \begin{itemize}
    \item $M=1331$.
    \item<alert@1-> $M_1 \neq M_2 \Rightarrow B_1=1$.  
    \item<alert@1->
     $M_1 \neq M_2=M_3 \neq M_4 \Rightarrow B_2=33$.
    \item<alert@1-> $M_3 \neq M_4 \Rightarrow B_3=1$.
    \item $M=B_1B_2B_3$.
     \end{itemize}
  \end{block}
  \end{frame}
%...............................................8
\begin{frame}[t]{Définitions}
  \begin{block}{Densité}
    \begin{itemize}
    \item<alert@1->
      Soit $\Sigma$ un alphabet, et soit $\alpha \in \Sigma$ une lettre.
    \item La densité de $\alpha$ dans un mot M de longueur n est donnée par la formule suivante: 
 $$\rho_n = \frac{|\{ 1 \leq j \leq n : M_j =\alpha \} |}{n}$$
    \end{itemize}
  \end{block}
\end{frame}
%...............................................9
\begin{frame}[t]{Définitions}
  \begin{block}{Fonction de codage par blocs}
   Soit $\Sigma$ un alphabet.
    \begin{itemize}
    \item  La fonction de codage par blocs est définie par:   
     $$\Delta: \Sigma^* \longrightarrow \Sigma^* $$ telle que $$M \longmapsto M'=long(B_1)long(B_2)...$$ 
où  $B_i$    est le i-iéme bloc du mot M pour tout $i\in \mathbb{N}^*$
     \end{itemize}
  \end{block}
\end{frame}
%..............................................
\begin{frame}[t]{Définitions}
   \begin{block}{Exemple d'application}
    \begin{itemize}
    \item  M=1333113331311
    \item $M=\underset{1}{\underbrace{1}} \underset{3}{\underbrace{333}} \underset{2}{\underbrace{11}} \underset{3}{\underbrace{333}} \underset{1}{\underbrace{1}} \underset{1}{\underbrace{3}} \underset{2}{\underbrace{11}}$
    \item<alert@1->
    $\Delta(M)=1323112$.
  \end{itemize}
  \end{block}
\end{frame}
%..............................................11
\begin{frame}[t]{Définitions}
  \begin{block}{Exemple d'utilisation}
   \item Les imprimantes et les photocopieuses se basent sur l'application de la fonction de codage par blocs lors de l'impression.
  \end{block}
\end{frame}
%..............................................12
\begin{frame}[t]{Définitions}
 Soit $\Sigma = \{ a,b \}$ , avec $a \neq b$, un alphabet.
  \begin{block}{Fonction pseudo-inverse} 
    \begin{itemize}
    \item La fonction pseudo-inverse est définie par l'opération inverse de la fonction de codage par blocs.\\ On la note $\Delta^{-1}$.
\end{itemize}
\end{block}
\end{frame}
%..............................................12
\begin{frame}[t]{Définitions}
 Soit $\Sigma = \{ a,b \}$ , avec $a \neq b$, un alphabet.
  \begin{block}{Fonction pseudo-inverse} 
    \begin{itemize}
    \item La fonction pseudo-inverse est définie par l'opération inverse de la fonction de codage par blocs.\\ On la note $\Delta^{-1}$.
\end{itemize}
\end{block}
 \begin{itemize}   
    \item<alert@1->
    On a $$\Delta^{-1}: \Sigma^* \underset{M \longmapsto W}{\longrightarrow} \Sigma^*=\{a,b\}^*$$ 
avec $W= W_1 W_2 ...$ où $W_1=w_1^1=a$, et $\forall i \geq 2$ on a:  $ W_i=w_1^iw_2^i...w_{M_i}^i$  tel que pour $j=1,...,M_{i}$ , 
$$\left\{ \begin{array}{lll} 
w_j^i = a & si & w_1^{i-1} = b \\
w_j^i = b & si & w_1^{i-1} = a$$
\end{array} 
\end{itemize}
\end{frame}
%..............................................13
\begin{frame}[t]{Définitions}
  \begin{block}{Exemple d'application}
    \begin{itemize}
    \item  $M=1333111$
    \item $M=\underset{1}{\underbrace{1}} 
            \underset{333}{\underbrace{3}} 
            \underset{111}{\underbrace{3}} 
            \underset{333}{\underbrace{3}} 
            \underset{1}{\underbrace{1}}
            \underset{3}{\underbrace{1}}
            \underset{1}{\underbrace{1}}$
    \item<alert@1->
    $\Delta^{-1}(M)= 1333111333131$.
  \end{itemize}
  \end{block}
\end{frame}
%..............................................14
\begin{frame}[t]{Définitions}
Désormais on prend $\Sigma = \{ 1,3 \}$.
  \begin{block}{Mot de Kolakoski}  
    \begin{itemize}
    \item On appelle mot de Kolakoski, le point fixe sous l'opérateur $\Delta$ ayant 1 comme première lettre.
    \item<alert@1->
     On note K le mot de Kolakoski.
    \end{itemize}
  \end{block}
\end{frame}
%..............................................17
\begin{frame}[t]{Construction du mot de Kolakoski}
 Notation: $K^p=K_1K_2...K_p \hspace{1\baselineskip} \forall p \in \mathbb{N}^*.$ \\
    \begin{itemize}
    \item $K_1=1$ \Rightarrow $K_2=3$ \Rightarrow $K^2=13$ 
    \item $\Delta^{-1}(K^2)=1333=K^4$, 
    \item   
$\Delta^{-1}(K^4)=1333111333=K^{10}$.
    \item $\Delta^{-1}(K^{10})=1333111333131333111333=K^{22}$
    \item $\Delta^{-1}(K^{22})=13331113331313331113331313...=K^{52}$
    \end{itemize}
\end{frame}
%..............................................18
\begin{frame}[t]{Propriétés}
Soit M un mot fini.
 \begin{block}{Propriétés de la longueur}
    \begin{itemize}
\item  $long(\Delta(M))\leq long(M).$
\item  $long(M) \leq long(\Delta^{-1}(M)).$
    \end{itemize}
  \end{block}
\end{frame}
%..............................................19
\begin{frame}[t]{Propriétés}
  \begin{block}{Propriétés du mot de Kolakoski}
Le mot de Kolakoski est un mot:    
    \begin{itemize}
\item  Lisse: $\forall k \in \mathbb{N}, \Delta^k(M) \in \Sigma^*= \{ 1,3 \}^*$.  
\item  Infini: $\forall n \in \mathbb{N}^* , K^n$ existe.
\item  Unique.
    \end{itemize}
  \end{block}
\end{frame}
%..............................................20
\begin{frame}[t]{Propriétés}
\begin{block}{Propriétés de passage de K à $\Delta^{-1}(K)$}
    \begin{itemize}
\item  $S_n:= \sum_{i=1}^{n}K_i=long(\Delta^{-1}(K^n)), \forall n \in \mathbb{N}^*$. 
\item 
\begin{tabular}{||c||c||} 
\hline \hline
K & \Delta^{-1}(K) \\
\hline \hline
 (1)^i &(1)^i \\ 
\hline 
 (1)^p & (3)^p\\ 
\hline 
(3)^i  & (111)=2\times (1)^i+(1)^p \\ 
\hline 
(3)^p & (333)=2\times(3)^p+(3)^i \\ 
\hline \hline
\end{tabular}
 \end{itemize}
  \end{block}
\end{frame}
%..............................................21
\begin{frame}[t]{Propriétés}
Soit $n \in \mathbb{N}^*$.
 \begin{block}{Propriétés de passage de $K^n$ à $\Delta^{-1}(K^n)$}
\begin{itemize}
\item 
$\left\{ \begin{array}{l}
1_{S_n}^i=1_n^i+2 \times 3_n^i \\ 
1_{S_n}^p=3_n^i \\ 
3_{S_n}^i=3_n^p \\ 
3_{S_n}^p=1_n^p+2 \times 3_n^p
\end{array}$
\item 
$\left( \begin{array}{c}
1_{S_n}^i \\ 
1_{S_n}^p \\ 
3_{S_n}^i \\ 
3_{S_n}^p
\end{array} \right) = 
\left( \begin{array}{cccc}
1 & 0 & 2 & 0 \\ 
0 & 0 & 1 & 0 \\ 
0 & 0 & 0 & 1 \\ 
0 & 1 & 0 & 2
\end{array} \right)
\left( \begin{array}{c}
1_n^i \\ 
1_n^p \\ 
3_n^i \\ 
3_n^p
\end{array} \right)$
    \end{itemize}
  \end{block}
\end{frame}
%%===============================================
\section{Problème étudié}
%..............................................22
\begin{frame}[t]{Problème et Objectifs}
 \begin{block}{Problème principal}
    \begin{itemize}
     \item  Etude de la suite Kol(1,3).
    \end{itemize}
  \end{block}
\end{frame}
%%%%%%%%%%%%%%
\begin{frame}[t]{Problème et Objectifs}
\begin{block}{Problème principal}
    \begin{itemize}
     \item  Etude de la suite Kol(1,3).
    \end{itemize}
  \end{block}
 \begin{block}{Objectif principal}
    \begin{itemize} 
  \item Calcul de la densité de 3 dans K.
    \end{itemize}
  \end{block}
\end{frame}

%%===============================================
\section{Méthodes de résolutions}
%..............................................24
\begin{frame}[t]{Première méthode}
  \begin{block}{Objectif}
    \begin{itemize}
\item Calcul de la densité de 3 dans le mot de Kolakoski. 
    \end{itemize}
  \end{block}
\end{frame}
%..............................................25
\begin{frame}[t]{Première méthode}
  \begin{block}{Objectif}
    \begin{itemize}
    \item Calcul de la densité de 3 dans le mot de Kolakoski. 
    \end{itemize}
  \end{block}
  \begin{block}{L'idée}
    \begin{itemize}
    \item on applique successivement $\Delta^{-1}$ à $K^2=13 \equiv \left( \begin{array}{c}
1 \\ 
0 \\ 
0\\ 
1
\end{array} \right) =X_0$.
\item On construit une suite récurrente $X_{n+1}=f(X_n)$ afin de déterminer les composantes de $X_n$, $\forall n \in \mathbb{N}$.
\item On calcule la densité par passage à la limite.
   \end{itemize}
  \end{block}
\end{frame}
%..............................................26
\begin{frame}[t]{Première méthode}
 La rédaction est en plusieurs étapes!
  \begin{block}{Rappel}
 \begin{center}
 $\left( \begin{array}{c}
1_{S_n}^i \\ 
1_{S_n}^p \\ 
3_{S_n}^i \\ 
3_{S_n}^p
\end{array} \right) = 
\left( \begin{array}{cccc}
1 & 0 & 2 & 0 \\ 
0 & 0 & 1 & 0 \\ 
0 & 0 & 0 & 1 \\ 
0 & 1 & 0 & 2
\end{array} \right)
\left( \begin{array}{c}
1_n^i \\ 
1_n^p \\ 
3_n^i \\ 
3_n^p
\end{array} \right)$
 \end{center}
  \end{block}
  On pose \begin{center}
  $M=\left( \begin{array}{cccc}
1 & 0 & 2 & 0 \\ 
0 & 0 & 1 & 0 \\ 
0 & 0 & 0 & 1 \\ 
0 & 1 & 0 & 2
\end{array} \right)$
  \end{center}
\end{frame}
%..............................................27
\begin{frame}[t]{Première méthode}
  \begin{block}{Etape 1: Polyn\^ome caractéristique de M}

 $$P_M(\lambda)=det(M-\lambda I_4)=(\lambda-1)(\lambda^3-2\lambda^2-1)$$

  \end{block}
\end{frame}
%..............................................28
\begin{frame}[t]{Première méthode}
  \begin{block}{Etape 2: Les valeurs propres de M}
Le spectre de M est composé de 1 et les valeurs suivantes:
\begin{itemize}
\item 
$\lambda_1=(\dfrac{43}{54}+\dfrac{\sqrt{177}}{18})^{\dfrac{1}{3}}+(\dfrac{43}{54}-\dfrac{\sqrt{177}}{18})^{\dfrac{1}{3}}+\dfrac{2}{3}$ \\ avec $|\lambda_1|=2,205569... >1$.
\item 
$\lambda_2$ avec $|\lambda_2|=0,67 < 1$.
\item 
$\lambda_3$ avec $|\lambda_3|=0,67 < 1$.
\end{itemize}
  \end{block}
\end{frame}
%..............................................29
\begin{frame}[t]{Première méthode}
  \begin{block}{Etape 3: Recherche d'un vecteur propre.}
  \end{block}
  On cherche un vecteur propre $\overrightarrow{V_1}=\left( \begin{array}{c}
x \\ 
y \\ 
z \\ 
t
\end{array} \right)$ associé à la valeur propre  $\lambda_1$.\\
 Donc $\overrightarrow{V_1}$ est tel que $M\overrightarrow{V_1}=\lambda_1\overrightarrow{V_1}$, ce qui est équivalent au système suivant:
\begin{center}
$(S): \left\{ \begin{array}{rcl}
x+2z & = & \lambda_1 x \\
z & = & \lambda_1 y \\
t & = & \lambda_1 z \\
y+2t&= & \lambda_1 t
\end{array} $
\end{center}
\end{frame}
%................................................
\begin{frame}[t]{Première méthode}
  \begin{block}{Etape 3: Recherche d'un vecteur propre.}
  \end{block}
 La résolution du système (S) donne: $$\overrightarrow{V_1}=\left(\begin{array}{c}
\dfrac{2\lambda_1}{\lambda_1-1} \\ 
1 \\ 
 \lambda_1\\ 
\lambda_1^2
\end{array} \right)$$ 
\end{frame}
%..............................................30
\begin{frame}[t]{Première méthode}
 \begin{block}{Etape 4: Relation entre $\overrightarrow{X_0}$ et les vecteurs propres }
  \end{block}
  Soient $\overrightarrow{V_1}$, $\overrightarrow{V_2}$, $\overrightarrow{V_3}$ et $\overrightarrow{V_4}$ respectivement les vecteurs propres associés aux valeurs propres $\lambda_1$, $\lambda_2$, $\lambda_3$ et $1$. \\
\end{frame}
%..............................................30
\begin{frame}[t]{Première méthode}
 \begin{block}{Etape 4: Relation entre $\overrightarrow{X_0}$ et les vecteurs propres }
  \end{block}
  Soient $\overrightarrow{V_1}$, $\overrightarrow{V_2}$, $\overrightarrow{V_3}$ et $\overrightarrow{V_4}$ respectivement les vecteurs propres associés aux valeurs propres $\lambda_1$, $\lambda_2$, $\lambda_3$ et $1$. \\
Alors il existe des constantes a, b, c et d réelles telles que $$\overrightarrow{X_0}=a\overrightarrow{V_1}+b\overrightarrow{V_2}+c\overrightarrow{V_3}+d\overrightarrow{V_4}$$   
\end{frame}
%..............................................30
\begin{frame}[t]{Première méthode}
 \begin{block}{Etape 4: Relation entre $\overrightarrow{X_0}$ et les vecteurs propres }
  \end{block}
  Soient $\overrightarrow{V_1}$, $\overrightarrow{V_2}$, $\overrightarrow{V_3}$ et $\overrightarrow{V_4}$ respectivement les vecteurs propres associés aux valeurs propres $\lambda_1$, $\lambda_2$, $\lambda_3$ et $1$. \\
Alors il existe des constantes a, b, c et d réelles telles que $$\overrightarrow{X_0}=a\overrightarrow{V_1}+b\overrightarrow{V_2}+c\overrightarrow{V_3}+d\overrightarrow{V_4}$$   
\begin{center}
\begin{tabular}{rl}
\overrightarrow{X_1}=M\overrightarrow{X_0}&=aM\overrightarrow{V_1}+bM\overrightarrow{V_2}+cM\overrightarrow{V_3}+dM\overrightarrow{V_4} \\
&=a\lambda_1\overrightarrow{V_1}+b\lambda_2\overrightarrow{V_2}+c\lambda_3\overrightarrow{V_3}+d\overrightarrow{V_4}
\end{tabular}
\end{center}
\end{frame}
%..............................................30
\begin{frame}[t]{Première méthode}
 \begin{block}{Etape 4: Relation entre $\overrightarrow{X_0}$ et les vecteurs propres }
  \end{block}
  Soient $\overrightarrow{V_1}$, $\overrightarrow{V_2}$, $\overrightarrow{V_3}$ et $\overrightarrow{V_4}$ respectivement les vecteurs propres associés aux valeurs propres $\lambda_1$, $\lambda_2$, $\lambda_3$ et $1$. \\
Alors il existe des constantes a, b, c et d réelles telles que $$\overrightarrow{X_0}=a\overrightarrow{V_1}+b\overrightarrow{V_2}+c\overrightarrow{V_3}+d\overrightarrow{V_4}$$   
\begin{center}
\begin{tabular}{rl}
\overrightarrow{X_1}=M\overrightarrow{X_0}&=aM\overrightarrow{V_1}+bM\overrightarrow{V_2}+cM\overrightarrow{V_3}+dM\overrightarrow{V_4} \\
&=a\lambda_1\overrightarrow{V_1}+b\lambda_2\overrightarrow{V_2}+c\lambda_3\overrightarrow{V_3}+d\overrightarrow{V_4}
\end{tabular}\\
\end{center}
\begin{center}
\begin{tabular}{rl}
\overrightarrow{X_2}=M\overrightarrow{X_1} & =a\lambda_1M\overrightarrow{V_1}+b\lambda_2M\overrightarrow{V_2}+c\lambda_3M\overrightarrow{V_3}+dM\overrightarrow{V_4} \\
&=a\lambda_1^2\overrightarrow{V_1}+b\lambda_2^2\overrightarrow{V_2}+c\lambda_3^2\overrightarrow{V_3}+d\overrightarrow{V_4}.
\end{tabular}
\end{center}
\end{frame}
%..............................................30
\begin{frame}[t]{Première méthode}
 \begin{block}{Etape 4: Relation entre $\overrightarrow{X_0}$ et les vecteurs propres }
  \end{block}
  Soient $\overrightarrow{V_1}$, $\overrightarrow{V_2}$, $\overrightarrow{V_3}$ et $\overrightarrow{V_4}$ respectivement les vecteurs propres associés aux valeurs propres $\lambda_1$, $\lambda_2$, $\lambda_3$ et $1$. \\
Alors il existe des constantes a, b, c et d réelles telles que $$\overrightarrow{X_0}=a\overrightarrow{V_1}+b\overrightarrow{V_2}+c\overrightarrow{V_3}+d\overrightarrow{V_4}$$   
\begin{center}
\begin{tabular}{rl}
\overrightarrow{X_1}=M\overrightarrow{X_0}&=aM\overrightarrow{V_1}+bM\overrightarrow{V_2}+cM\overrightarrow{V_3}+dM\overrightarrow{V_4} \\
&=a\lambda_1\overrightarrow{V_1}+b\lambda_2\overrightarrow{V_2}+c\lambda_3\overrightarrow{V_3}+d\overrightarrow{V_4}
\end{tabular}
\end{center}
\begin{center}
\begin{tabular}{rl}
\overrightarrow{X_2}=M\overrightarrow{X_1} & =a\lambda_1M\overrightarrow{V_1}+b\lambda_2M\overrightarrow{V_2}+c\lambda_3M\overrightarrow{V_3}+dM\overrightarrow{V_4} \\
&=a\lambda_1^2\overrightarrow{V_1}+b\lambda_2^2\overrightarrow{V_2}+c\lambda_3^2\overrightarrow{V_3}+d\overrightarrow{V_4}.
\end{tabular}
\end{center}
\begin{center}
$\overrightarrow{X_n}=M^n\overrightarrow{X_0}=a\lambda_1^n\overrightarrow{V_1}+b\lambda_2^n\overrightarrow{V_2}+c\lambda_3^n\overrightarrow{V_3}+d\overrightarrow{V_4}$
\end{center}
\end{frame}
%..............................................31
\begin{frame}[t]{Première méthode}
  \begin{block}{Etape 5: Calcul de la densité}
  \end{block}
On a $$\overrightarrow{X_n}\underset{n\rightarrow + \infty}{\sim}a\lambda_1^n\overrightarrow{V_1}.$$
\end{frame}
%..............................................31
\begin{frame}[t]{Première méthode}
  \begin{block}{Etape 5: Calcul de la densité}
  \end{block}
On a $$\overrightarrow{X_n}\underset{n\rightarrow + \infty}{\sim}a\lambda_1^n\overrightarrow{V_1}.$$
c'est-à-dire $$\overrightarrow{X_n}\underset{n\rightarrow + \infty}{\sim}a\lambda_1^n\left(
\begin{array}{c}
\dfrac{-2\lambda_1}{1-\lambda_1} \\ 
1 \\ 
 \lambda_1\\ 
\lambda_1^2
\end{array} \right)$$ 
\end{frame}
%..............................................31
\begin{frame}[t]{Première méthode}
  \begin{block}{Etape 5: Calcul de la densité}
  \end{block}
On a $$\overrightarrow{X_n}\underset{n\rightarrow + \infty}{\sim}a\lambda_1^n\overrightarrow{V_1}.$$
c'est-a-dire $$\overrightarrow{X_n}\underset{n\rightarrow + \infty}{\sim}a\lambda_1^n\left(
\begin{array}{c}
\dfrac{-2\lambda_1}{1-\lambda_1} \\ 
1 \\ 
 \lambda_1\\ 
\lambda_1^2
\end{array} \right)$$ 
Dans $\overrightarrow{X_n}$ on peut calculer $1_n^i$, $1_n^p$, $3_n^i$ et $3_n^p$.
\end{frame}
%..............................................32
\begin{frame}[t]{Première méthode}
On a
$$\rho_3=\dfrac{\lambda_1+\lambda_1^2}{\dfrac{-2\lambda_1}{1-\lambda_1}+1+\lambda_1+\lambda_1^2}=\dfrac{(1-\lambda_1)(\lambda_1+\lambda_1^2)}{-2\lambda_1+(1-\lambda_1)(1+\lambda_1+\lambda_1^2)}$$
\end{frame}
%..............................................32
\begin{frame}[t]{Première méthode}
On a
$$\rho_3=\dfrac{\lambda_1+\lambda_1^2}{\dfrac{-2\lambda_1}{1-\lambda_1}+1+\lambda_1+\lambda_1^2}=\dfrac{(1-\lambda_1)(\lambda_1+\lambda_1^2)}{-2\lambda_1+(1-\lambda_1)(1+\lambda_1+\lambda_1^2)}$$ 
Or $$-2\lambda_1^2=1-\lambda_1^3=(1-\lambda_1)(1+\lambda_1+\lambda_1^2) $$
\end{frame}
%..............................................32
\begin{frame}[t]{Première méthode}
On a
$$\rho_3=\dfrac{\lambda_1+\lambda_1^2}{\dfrac{-2\lambda_1}{1-\lambda_1}+1+\lambda_1+\lambda_1^2}=\dfrac{(1-\lambda_1)(\lambda_1+\lambda_1^2)}{-2\lambda_1+(1-\lambda_1)(1+\lambda_1+\lambda_1^2)}$$ 
Or $$-2\lambda_1^2=1-\lambda_1^3=(1-\lambda_1)(1+\lambda_1+\lambda_1^2) $$ \\
 Donc $$\rho_3=\dfrac{(1-\lambda_1)(\lambda_1+\lambda_1^2)}{-2(\lambda_1+\lambda_1^2)}$$
\end{frame}
%..............................................32
\begin{frame}[t]{Première méthode}
On a
$$\rho_3=\dfrac{\lambda_1+\lambda_1^2}{\dfrac{-2\lambda_1}{1-\lambda_1}+1+\lambda_1+\lambda_1^2}=\dfrac{(1-\lambda_1)(\lambda_1+\lambda_1^2)}{-2\lambda_1+(1-\lambda_1)(1+\lambda_1+\lambda_1^2)}$$ 
Or $$-2\lambda_1^2=1-\lambda_1^3=(1-\lambda_1)(1+\lambda_1+\lambda_1^2) $$ \\
 Donc $$\rho_3=\dfrac{(1-\lambda_1)(\lambda_1+\lambda_1^2)}{-2(\lambda_1+\lambda_1^2)}$$ \\
 D'où\begin{center}
  $\rho_3=\dfrac{\lambda_1-1}{2}\simeq0,6027847152$
 \end{center}
\end{frame}
%...........................................
\begin{frame}[t]{Deuxième méthode}
  \begin{block}{Objectif}
    \begin{itemize}
    \item Calcul de la densité de 3 dans le mot de Kolakoski. 
    \end{itemize}
  \end{block}
\end{frame}
%...........................................
\begin{frame}[t]{Deuxième méthode}
  \begin{block}{Objectif}
    \begin{itemize}
    \item Calcul de la densité de 3 dans le mot de Kolakoski. 
    \end{itemize}
  \end{block}
  \begin{block}{Principe}
    \begin{itemize}
    \item Le principe sur lequel on se base dans  cette méthode, consiste à établir une relation de récurrence entre $\rho_n ,\rho_{S_n}$ et $\rho_{S_{S_n}}=\rho_{S_{2,n}}$. 
 Puis puisque  $(\rho_{S_n})_n$ et $(\rho_{S_{2,n}})_n$
  sont deux suites extraites de $(\rho_n)_n$, 
  alors si elles convergent, elles convergent vers la 
  m\^eme limite. Et donc la limite de la densité est 
  solution d'une équation à résoudre
   \end{itemize}
  \end{block}
\end{frame}
%..............................................36
\begin{frame}[t]{Deuxième méthode}
  \begin{block}{Démonstration}
  \end{block}
  Soit $n \in \mathbb{N}^*$, on a $S_n=1_{S_n}^i+1_{S_n}^p+3_{S_n}^i+3_{S_n}^p $. Donc\\
 $S_n=1_n^i+1_n^p+3_n^i+3_n^p+2 \times (3_n^i+3_n^p) 
 =n+2n \rho_n =n(1+2 \rho_n)$.\\
\end{frame}
%..............................................36
\begin{frame}[t]{Deuxième méthode}
  \begin{block}{Démonstration}
  \end{block}
  Soit $n \in \mathbb{N}^*$, on a $S_n=1_{S_n}^i+1_{S_n}^p+3_{S_n}^i+3_{S_n}^p $. Donc\\
 $S_n=1_n^i+1_n^p+3_n^i+3_n^p+2 \times (3_n^i+3_n^p) 
 =n+2n \rho_n =n(1+2 \rho_n)$.\\
D'où \, \, \, \, $S_n=n(1+2 \rho_n)$.
\end{frame}
%..............................................36
\begin{frame}[t]{Deuxième méthode}
  \begin{block}{Démonstration}
  \end{block}
  Soit $n \in \mathbb{N}^*$, on a $S_n=1_{S_n}^i+1_{S_n}^p+3_{S_n}^i+3_{S_n}^p $. Donc\\
 $S_n=1_n^i+1_n^p+3_n^i+3_n^p+2 \times (3_n^i+3_n^p) 
 =n+2n \rho_n =n(1+2 \rho_n)$.\\
D'où \, \, \, \, $S_n=n(1+2 \rho_n)$.\\
Alors $ S_{2,n}=S_n(1+2 \rho_{S_n})$ et comme $ S_{2,n}=\sum_{j=1}^n K_j(2+(-1)^j)$
\end{frame}
%..............................................36
\begin{frame}[t]{Deuxième méthode}
  \begin{block}{Démonstration}
  \end{block}
  Soit $n \in \mathbb{N}^*$, on a $S_n=1_{S_n}^i+1_{S_n}^p+3_{S_n}^i+3_{S_n}^p $. Donc\\
 $S_n=1_n^i+1_n^p+3_n^i+3_n^p+2 \times (3_n^i+3_n^p) 
 =n+2n \rho_n =n(1+2 \rho_n)$.\\
D'où \, \, \, \, $S_n=n(1+2 \rho_n)$.\\
Alors $ S_{2,n}=S_n(1+2 \rho_{S_n})$ et comme $ S_{2,n}=\sum_{j=1}^n K_j(2+(-1)^j)$\\ 
Donc $(1+2 \rho_{S_n})S_n= 2S_n+ \sum_{j=1}^n(-1)^jK_j$$
\end{frame}
%..............................................36
\begin{frame}[t]{Deuxième méthode}
  \begin{block}{Démonstration}
  \end{block}
  Soit $n \in \mathbb{N}^*$, on a $S_n=1_{S_n}^i+1_{S_n}^p+3_{S_n}^i+3_{S_n}^p $. Donc\\
 $S_n=1_n^i+1_n^p+3_n^i+3_n^p+2 \times (3_n^i+3_n^p) 
 =n+2n \rho_n =n(1+2 \rho_n)$.\\
D'où \, \, \, \, $S_n=n(1+2 \rho_n)$.\\
Alors $ S_{2,n}=S_n(1+2 \rho_{S_n})$ et comme $ S_{2,n}=\sum_{j=1}^n K_j(2+(-1)^j)$\\ 
Donc $(1+2 \rho_{S_n})S_n= 2S_n+ \sum_{j=1}^n(-1)^jK_j$$\\
\Rightarrow  \, \, $$(2\rho_{S_n}-1)S_n =\sum_{j=1}^n(-1)^jK_j
 =\sum_{j=1,K_j=3}^n(-1)^jK_j+\sum_{j=1,K_j=1}^n(-1)^jK_j
 =3(3_n^p-3_n^i)+(1_n^p-1_n^i)
$
\end{frame}
%..............................................37
\begin{frame}[t]{Deuxième méthode}
si on fait le bilan des indices dans $K^n$, on trouve $$ 3_n^p+1_n^p=3_n^i+1_n^i+\dfrac{(-1)^n-1}{2}$$
\end{frame}
%..............................................37
\begin{frame}[t]{Deuxième méthode}
si on fait le bilan des indices dans $K^n$, on trouve $$ 3_n^p+1_n^p=3_n^i+1_n^i+\dfrac{(-1)^n-1}{2}$$ \\ donc 
$$(2\rho_{S_n}-1)S_n=2(3_n^p-3_n^i)+\frac{(-1)^n-1}{2},$$
\end{frame}%..............................................37
\begin{frame}[t]{Deuxième méthode}
si on fait le bilan des indices dans $K^n$, on trouve $$ 3_n^p+1_n^p=3_n^i+1_n^i+\dfrac{(-1)^n-1}{2}$$ \\ donc 
$$(2\rho_{S_n}-1)S_n=2(3_n^p-3_n^i)+\frac{(-1)^n-1}{2},$$\\ 
et si l'on remplace $n$ par $S_n$, on obtient \\$(2\rho_{S_{2,n}}-1)S_{2,n}=2(2 \times 3_n^p+1_n^p-3_n^p)+ \frac{(-1)^n-1}{2}$ \\ $$=2(3_n^p+1_n^p)+\frac{(-1)^n-1}{2},$$
\end{frame}
%..............................................38
\begin{frame}[t]{Deuxième méthode}
Or $$3_n^p+1_n^p=3_n^i+1_n^i+\frac{(-1)^n-1}{2}=\frac{1}{2}(n+\frac{(-1)^n-1}{2})=\frac{n}{2}+\frac{(-1)^n-1}{4}$$
\end{frame}
%..............................................38
\begin{frame}[t]{Deuxième méthode}
Or $$3_n^p+1_n^p=3_n^i+1_n^i+\frac{(-1)^n-1}{2}=\frac{1}{2}(n+\frac{(-1)^n-1}{2})=\frac{n}{2}+\frac{(-1)^n-1}{4}$$
donc $$(2\rho_S_{2,n}-1)S_{2,n}=n+\frac{(-1)^n-1}{2}+\frac{(-1)^n-1}{2}=n+(-1)^n-1.$$ 
\end{frame}%..............................................38
\begin{frame}[t]{Deuxième méthode}
Or $$3_n^p+1_n^p=3_n^i+1_n^i+\frac{(-1)^n-1}{2}=\frac{1}{2}(n+\frac{(-1)^n-1}{2})=\frac{n}{2}+\frac{(-1)^n-1}{4}$$
donc $$(2\rho_S_{2,n}-1)S_{2,n}=n+\frac{(-1)^n-1}{2}+\frac{(-1)^n-1}{2}=n+(-1)^n-1.$$ 
ou bien, sachant que $S_{2,n}=(1+2\rho_{S_n})(1+2\rho_n),$ $$(2\rho_{S_{2,n}}-1)(1+2\rho_{S_n})(1+2\rho_n)=1+\frac{(-1)^n-1}{n}$$ 
\end{frame}
%..............................................38
\begin{frame}[t]{Deuxième méthode}
Or $$3_n^p+1_n^p=3_n^i+1_n^i+\frac{(-1)^n-1}{2}=\frac{1}{2}(n+\frac{(-1)^n-1}{2})=\frac{n}{2}+\frac{(-1)^n-1}{4}$$
donc $$(2\rho_S_{2,n}-1)S_{2,n}=n+\frac{(-1)^n-1}{2}+\frac{(-1)^n-1}{2}=n+(-1)^n-1.$$ 
ou bien, sachant que $S_{2,n}=(1+2\rho_{S_n})(1+2\rho_n),$ $$(2\rho_{S_{2,n}}-1)(1+2\rho_{S_n})(1+2\rho_n)=1+\frac{(-1)^n-1}{n}$$
 On suppose que $(\rho_n)_n$ converge, soit alors $l=\underset{n\rightarrow + \infty}{l i m}\rho_n$,\\
 donc 
$\underset{n\rightarrow + \infty}{l i m}\rho_{S_n}=\underset{n\rightarrow + \infty}{l i m}\rho_{S_{2,n}}=l$
\end{frame}
%..............................................38
\begin{frame}[t]{Deuxième méthode}
Or $$3_n^p+1_n^p=3_n^i+1_n^i+\frac{(-1)^n-1}{2}=\frac{1}{2}(n+\frac{(-1)^n-1}{2})=\frac{n}{2}+\frac{(-1)^n-1}{4}$$
donc $$(2\rho_S_{2,n}-1)S_{2,n}=n+\frac{(-1)^n-1}{2}+\frac{(-1)^n-1}{2}=n+(-1)^n-1.$$ 
ou bien, sachant que $S_{2,n}=(1+2\rho_{S_n})(1+2\rho_n),$ $$(2\rho_{S_{2,n}}-1)(1+2\rho_{S_n})(1+2\rho_n)=1+\frac{(-1)^n-1}{n}$$
 On suppose que $(\rho_n)_n$ converge, soit alors $l=\underset{n\rightarrow + \infty}{l i m}\rho_n$,\\
 donc 
$\underset{n\rightarrow + \infty}{l i m}\rho_{S_n}=\underset{n\rightarrow + \infty}{l i m}\rho_{S_{2,n}}=l$,
donc $$(2l-1)(1+2l)^2=1$$
\end{frame}
%..............................................38
\begin{frame}[t]{Deuxième méthode}
 Ceci est équivalent, à
$$
L^3+L^2-L-2=0 \quad \quad \quad (1)
$$ \\
où $L=2l$.
\end{frame}
%..............................................38
\begin{frame}[t]{Deuxième méthode}
 Ceci est équivalent, à
$$
L^3+L^2-L-2=0 \quad \quad \quad (1)
$$ \\
où $L=2l$. \\
Il y a deux racines complexes et une racine réelle.
\end{frame}

%..............................................38
\begin{frame}[t]{Deuxième méthode}
 Ceci est équivalent, à
 $$
L^3+L^2-L-2=0 \quad \quad \quad (1)
$$\\
où $L=2l$. \\
Il y a deux racines complexes et une racine réelle. \\ 
La densité est $\rho_3=\dfrac{L}{2}$, où L est la racine réelle de l'équation (1).
\end{frame}

%..............................................38
\begin{frame}[t]{Deuxième méthode}
 Ceci est équivalent, à
$$
L^3+L^2-L-2=0 \quad \quad \quad (1)
$$\\
où $L=2l$. \\
Il y a deux racines complexes et une racine réelle. \\ 
La densité est $\rho_3=\dfrac{L}{2}$, où L est la racine réelle de l'équation (1).\\
Finalement 
$$\rho_3=\dfrac{1}{2} 
\left[(\dfrac{43}{54}+\dfrac{\sqrt{177}}{18})^{\dfrac{1}{3}}+(\dfrac{43}{54}-\dfrac{\sqrt{177}}{18})^{\dfrac{1}{3}}-\dfrac{1}{3}\right]$$ \\
$$\rho_3\simeq0,6027847152$$
\end{frame}



%%===============================================
\section{Synthèse}
%..............................................39
\begin{frame}[t]{comparaison des formules obtenues}
  \begin{block}{Rappel}
    \begin{itemize}
    \item $\lambda_1=(\dfrac{43}{54}+\dfrac{\sqrt{177}}{18})^{\dfrac{1}{3}}+(\dfrac{43}{54}-\dfrac{\sqrt{177}}{18})^{\dfrac{1}{3}}+\dfrac{2}{3}$
    \end{itemize}
  \end{block}
  \begin{block}{Formule de la première méthode}
    \begin{itemize}
    $\item \rho_3=\dfrac{\lambda_1-1}{2}$
    \end{itemize}
  \end{block}
  \begin{block}{Formule de la deuxième méthode}
    \begin{itemize}
    \item $\rho_3=\dfrac{1}{2} 
\left[(\dfrac{43}{54}+\dfrac{\sqrt{177}}{18})^{\dfrac{1}{3}}+(\dfrac{43}{54}-\dfrac{\sqrt{177}}{18})^{\dfrac{1}{3}}-\dfrac{1}{3}\right]$
    \end{itemize}
  \end{block}
\end{frame}
%..............................................40
\begin{frame}[t]{Représentation graphique}
    \begin{itemize}
    \item $\rho_3\simeq0,6027847152$ 
    \end{itemize}
 \begin{center}
  \includegraphics[scale=1]{KOLAKO13.jpg}
 \end{center}
\end{frame}
%................................................
\begin{frame}[t]{comparaison des méthodes} 
       \begin{itemize}
  \item<alert@1-> La première méthode.
  \item 
  \item 
  \item 
       \end{itemize}
       \begin{itemize}
  \item<alert@1-> La deuxième méthode.
  \item 
  \item 
  \item 
       \end{itemize}
\end{frame}
\begin{frame}[t]{comparaison des méthodes} 
       \begin{itemize}
  \item<alert@1-> La première méthode.
  \item Méthode algébrique.
  \item 
  \item 
         \end{itemize}
       \begin{itemize}
  \item<alert@1-> La deuxième méthode.
  \item Méthode analytique.
  \item 
  \item 
       \end{itemize}
\end{frame}

%........................................
\begin{frame}[t]{comparaison des méthodes} 
       \begin{itemize}
  \item<alert@1-> La première méthode.
  \item Méthode algébrique.
  \item N'est pas précise.
  \item 
       \end{itemize}
       \begin{itemize}
  \item<alert@1-> La deuxième méthode.
  \item Méthode analytique.
  \item Exacte.
  \item 
       \end{itemize}
\end{frame}

%......................................
\begin{frame}[t]{comparaison des méthodes} 
       \begin{itemize}
  \item<alert@1-> La première méthode.
  \item Méthode algébrique.
  \item N'est pas précise.
  \item Généralisable sur $\Sigma=\{2p+1,2q+1\}$.
       \end{itemize}
       \begin{itemize}
  \item<alert@1-> La deuxième méthode.
  \item Méthode analytique.
  \item Exacte.
  \item Généralisable sur $\Sigma=\{2p+1,2q+1\}$.
       \end{itemize}
\end{frame}
%.........................................
\begin{frame}[t]{comparaison des méthodes} 
       \begin{itemize}
  \item<alert@1-> La première méthode.
  \item Méthode algébrique.
  \item N'est pas précise.
  \item Généralisable sur $\Sigma=\{2p+1,2q+1\}$.
       \end{itemize}
       \begin{itemize}
  \item<alert@1-> La deuxième méthode.
  \item Méthode analytique.
  \item Exacte.
  \item Généralisable sur $\Sigma=\{2p+1,2q+1\}$.
       \end{itemize}
\begin{center}
Le même résultat.
\end{center}
\end{frame}

%................................................
\begin{frame}[t]{C'est la fin de la présentation}
 \begin{Huge}
  \begin{center}
    \textcolor{green}{\emph{MERCI}} \\ 
    \textcolor{green}{\emph{pour}}  \\ 
    \textcolor{green}{\emph{l'attention !}}
  \end{center}
 \end{Huge}
\end{frame}
%::::::::::::::::::::::::::::::::::::::::::::::::
\end{document}  
  
