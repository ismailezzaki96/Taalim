
\documentclass[12pt]{beamer}
%====================================================================================
\usepackage[frenchb]{babel}
\usepackage[utf8]{inputenc}
\usepackage[T1]{fontenc}
%=====================================================================================
\usepackage{verbatim}
\usepackage{lmodern}
%\usetheme{JuanLesPins}
%\usetheme{Warsaw} %...
%\usetheme{CambridgeUS}
%\usetheme{Antibes} %.............
%\usetheme{JuanLesPins}
%\usetheme{Ilmenau}
%\usetheme{Berlin} % important
%\usecolortheme{whale}
%\usefonttheme[12pt]{serif}
%\usepackage{lmodern}
%\usetheme{Dresden}
%\usetheme{PaloAlto}
%\usetheme{Madrid}
\newcommand{\norm}[1]{\left\Vert #1\right\Vert}
\newcommand\restr[2]{{% we make the whole thing an ordinary symbol
		\left.\kern-\nulldelimiterspace % automatically resize the bar with \right
		#1 % the function
		\vphantom{|} % pretend it's a little taller at normal size
		\right|_{#2} % this is the delimiter
}}
\newcommand*\oldmacro{}%
\let\oldmacro\insertshorttitle%
\renewcommand*\insertshorttitle{%
	\oldmacro\hfill%
	\insertframenumber\,/\,\inserttotalframenumber}
\usetheme{Warsaw}


%\usetheme{Boadilla} % % % % % % % % % % % % % % % % % % % % % % %
%\usetheme[Goettingen]
%CHOIX DU THEME et/ou DE SA COULEUR
% => essayer diff�rents th�mes (en d�commantant une des trois lignes suivantes)
\linespread{1.0}


% => il est possible, pour un th�me donn�, de modifier seulement la couleur
%\usecolortheme{spruce}
%\usecolortheme{whale}
%\usecolortheme{albatross}
%\usecolortheme{crane}
%\usecolortheme{seahorse} % % % % % % % % % % % % % % % % % % % % % % % % % % % % % % % % % %
%\pagestyle{plain}
%\useoutertheme[left]{sidebar}
\setbeamersize{text margin left=0.3cm}
\setbeamersize{text margin right=0.3cm}
\setbeamersize{sidebar width left=0.1cm}
\setbeamersize{sidebar width right=0.1cm}
\begin{comment}
contenu...
%=================================================================================
\addtobeamertemplate{block begin}{%
\setlength{\textwidth}{1\textwidth}%
}{}

\addtobeamertemplate{block alerted begin}{%
\setlength{\textwidth}{0.9\textwidth}%
}{}

\addtobeamertemplate{block example begin}{%
\setlength{\textwidth}{1.2\textwidth}%
}{}\end{comment}

%{\begin{column}{0.5cm} %
%#2\end{column}\end{columns} %
%}


%========
\usepackage{amsmath, amssymb}
%===================================================

\usepackage{hyperref}
\usepackage{amsthm}

\usepackage{colortbl}
%===================================================================================================
\newtheorem{deff}{D\'efinition}
\newtheorem{prop}{Proposition}
\newtheorem{lem}{Lemme}
\newtheorem{cor}{Corollaire}
\newtheorem{dem}{Preuve}
\newtheorem{theo}{Th\'eor\`eme}
\newtheorem{rem}{Remarque}
\newtheorem{exo}{Exemple}
\usepackage{cancel}


%\usepackage[left=1cm,right=1cm]{geometry}

%\setbeamertemplate{frametitlecontinuation} {\insertcontinuationcount}
%================================================================================
%                       Pour le TITLEPAGE
%=================================================================================
\title[]{\bf  M\'{e}thode num\'{e}rique pour les \'{e}quations aux d\'{e}riv\'{e}e partielles issues du calcul des variations}
%\subtitle{Pour l'obtention du Dipl\^ome de Doctorat}
\author[COSD MATHS,FACT SCIENCES, UNIV SKIKDA]{  \it Pour l'obtention du Dipl\^ome de Master: \\ \it Encadr\'e par :Khenniche Ghania \\  Option COSD  } %(Informatique et Math\'ematiques)
\date{\today}



\begin{document}

	\begin{frame}
	\titlepage
\end{frame}

%===========================================================================

%=====================
\begin{frame}
\frametitle{Plan}
\tableofcontents
\end{frame}
\begin{frame}
\frametitle{Introduction}
	
	\end{frame}

\section{Probl\`{e}me de calcul des variations sans contrainte}
\begin{frame}{Chapitre 01}
	\textbf{\textcolor{blue}{\large {Probl\`{e}me de calcul des variations sans contrainte}}}
\begin{itemize}
	\item Un probl\`{e}me de calcul des variations sans contrainte (\'{e}quation d'Euler-Lagrange d'ordre deux)
	\item Un probl\`{e}me de calcul des variations sans contrainte (\'{e}quation d'Euler-Lagrange d'ordre quatre)
\end{itemize}
\end{frame}
\subsection*{Un probl\`{e}me de calcul des variations sans contrainte (\'{e}quation d'Euler-Lagrange d'ordre deux)}
\begin{frame}
\textcolor{red}{
\textbf{Un probl\`{e}me de calcul des variations sans contrainte (\'{e}quation d'Euler-Lagrange d'ordre deux)}}\\
	Le probl\`{e}me consiste \`{a} d\'{e}terminer l'extr\'{e}mum d'une fonctionnelle d\'{e}finie sur $\mathbb{E}=C^{1}(D,\mathbb{R}^{2})$ avec $D\subset \mathbb{R}^{2} $ et,
	\begin{equation}
	\mathit{J(u(x,y))=\iint_{D}f(u,u_{x},u_{y},x,y)dxdy} \label{1*} \end{equation} 
	o\`{u} D est la r\'{e}gion dans le plan (x,y) de fronti\`{e}re $ C $, avec:
	\begin{equation}
	u(x,y)=g(x,y)\quad\text{sur}\quad C \label{0*}
	\end{equation}
Le probl\`{e}me de calcul des variations est le suivant:
\begin{block}{}
   \begin{equation}
	\min J(u(x,y))\text{ telle que u $\in \mathbb{E}$}
	\label{012}
	\end{equation}
\end{block}
	qui v\'{e}rifie la condition au bord ($\ref{0*}$)\end{frame}\begin{frame}
\textcolor{purple}{
	\textbf{\large{Etape 1: Hypoth\`{e}se d'extremum}}}\\
Suposons que $u^{*}(x,y)$ est un minimum et que $u(x,y)$ est suffisamment proche de $u^{*}(x,y)$ donn\'{e}e par:
\begin{equation}
\textcolor{red}{	u(x,y)=u^{*}(x,y)+\delta u(x,y)\label{l}}
\end{equation} 
o\`{u} $\delta u(x,y)$ est une variation de $u^{*}(x,y)$  avec:
\begin{equation}
\delta u(x,y)=0 \quad sur \quad C \label{oo}
\end{equation}
\textcolor{purple}{
\textbf{\large{Etape 2: Accroissement}}}\\
Soit
\begin{align}
\Delta J( u^{*}(x,y),\delta u)&=J(u^{*}(x,y)+\delta u)-J(u^{*}(x,y)\nonumber\\\nonumber
&=\iint_{D}f(u^{*}(x,y)+\delta u,u^{*}_{x}+\delta u_{x},u^{*}_{y}+\delta u_{y})\\&\phantom{=}-f(u^{*},u^{*}_{x},u^{*}_{y},x,y)dxdy
\label{5*}\end{align}\end{frame}
\begin{frame}
\textcolor{purple}{
	\textbf{\large{ Etape 3:Premi\`{e}re variation}}}\\
En utilisant le d\'{e}veloppoment de Taylor autour de $u^{*}(x,y)+\delta u(x,y)$
\begin{align}
\Delta J=&\iint[\frac{\partial{f}}{\partial{u}}(u^{*}(x,y),u^{*}_{x},u^{*}_{y},x,y)\delta u\nonumber\\&+\frac{\partial{f}}{\partial{u_{x}}}(u^{*}(x,y),u^{*}_{x},u^{*}_{y},x,y)\delta u_{x}\nonumber\\&
+\frac{\partial{f}}{\partial{u_{y}}}(u^{*}(x,y),u^{*}_{x},u^{*}_{y},x,y)\delta u_{y}]dxdy \label{x}
\end{align}
On a \begin{equation}\delta u_{x}\dfrac{\partial{f}}{\partial{u_{x}}}=\dfrac{\partial}{\partial{x}}(\delta u\dfrac{\partial{f}}{\partial{u_{x}}})-\delta u\dfrac{\partial}{\partial{x}}(\dfrac{\partial{f}}{\partial{u_{x}}})\cdots 
\label{f}\end{equation}
et \begin{equation}\delta u_{y}\dfrac{\partial{f}}{\partial{u_{y}}}=\dfrac{\partial}{\partial{y}}(\delta u\dfrac{\partial{f}}{\partial{u_{y}}})-\delta u\dfrac{\partial}{\partial{y}}(\dfrac{\partial{f}}{\partial{u_{y}}})\cdots 
\label{d}\end{equation}
en substituant ($\ref{f}$) et ($\ref{d}$) dans ($\ref{x}$) on obtient:
\end{frame}
\begin{frame}
\begin{align}
\Delta J=&\int\int_{D}\left[  \delta u\left\lbrace \frac{\partial{f}}{\partial{u}}-\dfrac{\partial}{\partial{x}}(\dfrac{\partial{f}}{\partial{u}_{x}})-\frac{\partial}{\partial{y}}(\dfrac{\partial{f}}{\partial{u}_{y}})\right\rbrace \right. \nonumber \\&\left. +\left\lbrace \dfrac{\partial}{\partial{x}}(\delta u\dfrac{\partial{f}}{\partial{u}_{x}})+ \dfrac{\partial}{\partial{y}}(\delta u\dfrac{\partial{f}}{\partial{u}_{y}})\right\rbrace \right]dxdy 
\label{m}\end{align}
\begin{alertblock}{Th\'{e}or\`{e}m de Green}
	On consid\`{e}re une forme diff\'{e}rentielle continue d\'{e}finie par 
	\begin{equation}\forall (x,y)\in \mathbb{R}^{2}: \omega(x,y)=P(x,y)dx+Q(x,y)dy\end{equation} et un compact $D\subset \mathbb{R}^{2}$ dont la fronti\`{e}re $(C)$ soit de classe $\mathbb{C}^{1}$ par morceaux et orient\'{e}e, alors 
	\begin{equation}\int\int_{D}\left( \dfrac{\partial{Q}}{\partial{x}}-\dfrac{\partial{P}}{\partial{y}}\right) dxdy=\int_{C}(Pdx+Qdy)\label{c}\end{equation}
\end{alertblock}
\end{frame}
\begin{frame}
En appliquant le r\'{e}sultat ($\ref{c}$) sur le deuxi\`{e}me terme de l'intégrale ($\ref{m}$) on obtient:
\begin{align}
\iint_{D}&\left[ \dfrac{\partial}{\partial{x}}(\delta u\dfrac{\partial{f}}{\partial{u}_{x}})+\dfrac{\partial}{\partial{y}}(\delta u\dfrac{\partial{f}}{\partial{u}_{y}})\right]dxdy\nonumber\\&=\int_{C}\left[ \delta u\dfrac{\partial{f}}{\partial{u}_{x}}dy-\delta u\dfrac{\partial{f}}{\partial{u}_{y}}dx\right]  \label{o} 
\end{align}
La condition au bord donn\'{e}e par ($\ref{oo}$) implique que 
$$
\int_{C}\left[ \delta u\dfrac{\partial{f}}{\partial{u}_{x}}dy-\delta u\dfrac{\partial{f}}{\partial{u}_{y}}dx\right] =0$$
% Donc on obtient:
\begin{equation}
\delta J=\int\int_{D}\left[  \delta u\left\lbrace \frac{\partial{f}}{\partial{u}}-\dfrac{\partial}{\partial{x}}(\dfrac{\partial{f}}{\partial{u}_{x}})-\frac{\partial}{\partial{y}}(\dfrac{\partial{f}}{\partial{u}_{y}})\right\rbrace \right]\end{equation}
\end{frame}
\begin{frame}
\textcolor{purple}{
	\textbf{\large{ Etape 4:condition d'extremum}}}
\begin{alertblock}{Th\'{e}or\`{e}m Fondamental}
	Soit $y^{*}(t)$ un extremum, la premi\`{e}re variation de la fonctionnelle J doit \^{e}tre nulle i.e $\delta J(y^{*}(t),\delta y(t))=0$
	
\end{alertblock}
\begin{alertblock}{Lemme Fondamental}
	Si l'intégrale $\int f(y).\eta (y)dy$; o\`{u} $ f(y)$ est une fonction continue par morceaux dans l'intervalle $[a,b]$, s'annulant pour tout fonction $\eta\in C^{1}[a,b]$ telle que $\eta(a)=\eta(b)=0$, alors $f(y)$ est identiquement nul dans l'intervalle $[a,b] $.
	
\end{alertblock}
En appliquant le th\'{e}or\`{e}m fondamental c-\`{a}-d $$\delta J=0$$ puis le lemme fondamental on obtient:\end{frame}
\begin{frame} l'\'{e}quation d'\textbf{Euler-Lagrang}
\begin{block}
	
\textcolor{red}{	
	\begin{equation}
	\frac{\partial{f}}{\partial{u}}-\dfrac{\partial}{\partial{x}}(\dfrac{\partial{f}}{\partial{u}_{x}})-\frac{\partial}{\partial{y}}(\dfrac{\partial{f}}{\partial{u}_{y}})=0 \label{n}
	\end{equation}}
\end{block}
\textbf{\large{Application:}}\\
Soit le probl\`{e}me variationnelle suivant \begin{equation}J=\int_{0}^{1}\int_{0}^{1} [\frac{1}{2} k_{1}(x,y)(\dfrac{\partial{u}}{\partial{x}})^{2}+\frac{1}{2}k_{2}(x,y)(\dfrac{\partial{u}}{\partial{y}})^{2}+\frac{\partial{u}}{\partial{t}}u]dxdy \end{equation}
consid\'{e}rons $\frac{\partial{u}}{\partial{t}}$ comme constante puis
en appliquant les résultats obtenu ci-dessus c'est-\'{a}-dire l'\'{e}quation d'\textbf{Euler-Lagrange} ($\ref{n}$)  on obtient: \begin{equation}
\dfrac{\partial{u}}{\partial{t}}-\frac{\partial}{\partial{x}}(k_{1}(x,y)\frac{\partial{u}}{\partial{x}})-\frac{\partial}{\partial{y}}(k_{2}(x,y)\frac{\partial{u}}{\partial{y}})=0\label{**-}\end{equation}
\end{frame}
\subsection*{Un probl\`{e}me de calcul des variations sans contrainte \'{e}quation d'Euler-Lagrane d'ordre quatre}
\begin{frame}
\textcolor{red}{
\textbf{Un probl\`{e}me de calcul des variations sans contrainte (\'{e}quation d'Euler-Lagrane d'ordre quatre)}}\\
Le probl\`{e}me consiste \`{a} déterminer l'extremum d'une fonctionnelle d\'{e}finit sur $ \mathbb{E}=C^{1}(V,\mathbb{R})$ avec $V=[x_{0},x_{f}]\times [y_{0},y_{f}]\times [t_{0},t_{f}] $ et
\begin{equation}
J(u(x,y,t))=\iiint_{V}f(x,y,u,u_{t},u_{x},u_{y},u_{xx},u_{xy},u_{yy})
\end{equation} 
o\`{u} $f(x,y,u,u_{t},u_{x},u_{y},u_{xx},u_{xy},u_{yy})$ est de classe $ C^{4}$ par rapport \'{a} ses arguments avec
\begin{equation}
u(x,y,t)=h(x,y,t) \quad sur \quad S\label{-*-*}\end{equation}
Le problème de calcul des variations est le suivant :
\begin{block}{}
\begin{equation}
\min   J(u(x,y,t)),\quad \text{telle que u(x,y,t)$\in\mathbb{ E} $}\label{0123} \end{equation}
\end{block}
vérifie la condition au bord ($\ref{-*-*}$)\\

\end{frame}
\begin{frame}
\begin{block}
	
	L'\'{e}quation d'\textbf{ Euler-Lagrang} est donn\'{e}e par:
\textcolor{red}{\begin{equation}
f_{u}-\dfrac{\partial }{\partial t}(f_{u_{t}})-\dfrac{\partial }{\partial x}(f_{u_{x}})-\dfrac{\partial }{\partial y}(f_{u_{y}})+\dfrac{\partial^{2} }{\partial x ^{2}}(f_{u_{xx}})+\dfrac{\partial^{2} }{\partial y^{2}}(f_{u_{yy}})+\dfrac{\partial^{2}}{\partial{x}\partial{y}}(f_{xy})=0
\label{***} \end{equation}}
\end{block}
\textbf{\large {Application:}}\\
Consid\'{e}rons le probl\`{e}me de minimisation suivant :
\begin{equation}
I=\iint_{D}(\dfrac{\partial u}{\partial t})^{2}-(\dfrac{\partial ^{2}u}{\partial x^{2}})^{2}dt dx
\end{equation}
d'apr\'{e}s l'\'{e}quation d'\textbf{Euler-Lagrange} ($\ref{***}$) on obtient 
\begin{equation}  
\dfrac{\partial ^{2}u}{\partial t^{2}}=-\dfrac{\partial ^{4}u}{\partial x^{4}}
\label{2*} \end{equation}
\end{frame}
\section{Discr\'{e}tisation spatio-temporelle}
\subsection*{Discr\'{e}tisation du probl\`{e}me d'ordre deux}
\begin{frame}{Chapitre 02}
\textbf{\textcolor{blue}{\large {Discr\'{e}tisation spatio-temporelle}}}
\begin{itemize}
	\item Discr\'{e}tisation du probl\`{e}me d'ordre deux
	\item Discr\'{e}tisation du probl\`{e}me d'ordre quatre
\end{itemize}
\end{frame}
\begin{frame}
\textcolor{red}{\textbf{Discr\'{e}tisation du probl\`{e}me d'ordre deux}}
\begin{block}{}
On veut r\'esoudre le   probl\`eme suivant:
 \begin{equation}
\begin{cases}
\frac{\partial{u}}{\partial{t}}-\frac{\partial}{\partial{x}}[ K_{1}(x, y)\frac{\partial{u}(x,y)}{\partial{x}}]-\frac{\partial}{\partial{y}}[K_{2}(x, y)\frac{\partial{u}(x, y)}{\partial{y}}]=0 & \text{ dans $ \Omega\times[0,T]$}\\
u(x,y,t)=0 & \text{sur $\varGamma $ }\\
u(x,y,0)=0   
\end{cases}
\label{a}\end{equation}
o\`{u}:\\
$u:\Omega\times[0,T]\longrightarrow$ ${\mathbb{R}}$ est une fonction suffisamment r\'{e}guli\`{e}re et $ K_{1}(x,y),K_{2}(x,y)$ deux fonctions positive dans le rectangle $\Omega$, telle que $\Omega=[0,1]^{2}$.


\end{block}


 \end{frame}
 \begin{frame}
 	\textbf{Discr\'{e}tisation du domaine:}
 \begin{block}{}
 	 On divise d'une part\\
 	-l'intervalle $[0,1]^{2}$ en $ (N+1) $ intervalle de longueur $h=\frac{1}{N+1}$,\\ et d'autre part\\
 	-l'intervalle $[0,T] $ en M intervalle de temps k telle que :T=Mk.\\
 	On pose $x_{i}=ih$ ; $y_{j}=jh$ pour $ i,j=1...N$ ; $t_{m}=mk $ pour $m=1...M.	$
 \end{block}
\textbf{Discr\'{e}tisation temporelle: }
 	 \begin{block}{}
 		La d\'{e}riv\'{e}e temporelle \'{e}tant discr\'{e}tis\'{e}e par:
 		\begin{equation}
 		\frac{\partial{u}(x_{i},y_{j},t_{m})}{\partial{t}}=\frac{{u}(x_{i},y_{j},t_{m+1})-u(x_{i},y_{j},t_{m})}{k}+0(k)\end{equation}
 	\end{block}
 \end{frame}
  \begin{frame}
  \textbf{Discr\'{e}tisation spatiale:}
 \begin{block}{}
 Pour \'{e}tablir le sch\'{e}ma de discr\'{e}tisation des op\'{e}rateurs diffinissant le probl\`{e}me,  deux  sch\'{e}mas sont pris en compte,  le sch\'{e}ma \textbf{Avant-Arri\`{e}re} et le sch\'{e}ma \textbf{Arri\`{e}re-Avant} puis on proc\`{e}de \`{a} la moyenne de ces deux sch\'{e}mas.\\ On prend le pas de discr\'{e}tisation dans les deux directions OX et OY \'{e}gale \'{a} h.\end{block}
\textbf{Premier cas: $K_{1}(x,y)\neq K_{2}(x,y)$}\\
\textbf{\'{e}tape 01:} On fixe y et on pose : $k_{1}(x,y)\frac{\partial{u}}{\partial{x}}=v_{1}(x,y)$
 \end{frame}
 	\begin{frame}
 	\begin{block}{}
 		 	\textbf{\large{Le sch\'{e}ma Avant-Arri\`{e}re:}}
 	\begin{align} \frac{\partial{v}_{1}(x_{i},y_{j})}{\partial{x}}&\simeq \frac{1}{h^{2}}[k^{1}_{i+1,j}u_{i+1,j}-(k^{1}_{i,j}+k^{1}_{i+1,j})u_{i,j}+k^{1}_{i,j}u_{i-1,j}]\nonumber\\&
 	+\xi^{av}_{x}(x_{i},y_{j}) \label{37*}\end{align}
 	O\`{u}  $\xi^{av}_{x}(x_{i},y_{j})$ est l'erreur de troncature donn\'{e}e par:
 	\begin{align}
 	\xi^{av}_{x}(x_{i},y_{j})=&-\frac{h}{2}\left( \frac{\partial^{2}{k}_{1}}{\partial{x^{2}}}(x_{i},y_{j})\frac{\partial{u}}{\partial{x}}(x_{i},y_{j},t_{m})\right. \nonumber\\&+\left. \frac{\partial{k_{1}}}{\partial{x}}(x_{i},y_{j})\frac{\partial^{2}{u}}{\partial{x^{2}}}(x_{i},y_{j},t_{m})\right) +O(h^{2})	\end{align}
 	 \end{block}
 \end{frame}
\begin{frame}
\begin{block}{}
 	\textbf{\large{Le sch\'{e}ma Arri\`{e}re-Avant:}}
 	\begin{align} \frac{\partial{v}_{1}(x_{i},y_{j})}{\partial{x}}&\simeq \frac{1}{h^{2}}[k^{1}_{i,j}u_{i+1,j}-(k^{1}_{i,j}+k^{1}_{i-1,j})u_{i,j}+k^{1}_{i-1,j}u_{i-1,j}]\nonumber\\&+\xi^{ar}_{x}(x_{i},y_{j}) \label{312*}\end{align}
 	O\`{u}  $\xi^{ar}_{x}(x_{i},y_{j})$ est l'erreur de troncature donn\'{e}e par:
 	\begin{align}\xi^{ar}_{x}(x_{i},y_{j})=&\frac{h}{2}\left( \frac{\partial^{2}{k}_{1}}{\partial{x^{2}}}(x_{i},y_{j})\frac{\partial{u}}{\partial{x}}(x_{i},y_{j},t_{m})\right. \nonumber\\&+\left. \frac{\partial{k_{1}}}{\partial{x}}(x_{i},y_{j})\frac{\partial^{2}{u}}{\partial{x^{2}}}(x_{i},y_{j},t_{m})\right) +O(h^{2}) \end{align}
 \end{block}
 	 \end{frame}
  \begin{frame}
  	En prenant la moyenne des deux expressions pr\'{e}c\'{e}dentes ($\ref{37*}$) et ($\ref{312*}$) on obtient:
  \begin{align}
  \frac{\partial}{\partial{x}}(k_{1}(x,y)\frac{\partial{u}}{\partial{x}})\nonumber&=\frac{1}{2h^{2}}[ (k^{1}_{i,j}+k^{1}_{i+1,j})u_{i+1,j}\nonumber\\&-(2k^{1}_{i,j}+k^{1}_{i+1,j}+k^{1}_{i-1,j})u_{i,j}\nonumber\\&+(k^{1}_{i,j}+k^{1}_{i-1,j})u_{i-1,j}] \nonumber\\&+\frac{1}{2}(\xi^{av}_{x}(x_{i},y_{j})+\xi^{ar}_{x}(x_{i},y_{j}))\end{align}
  \textbf{\'{e}tape 02:} On fixe x et on pose :	$k_{2}(x,y)\frac{\partial{u}}{\partial{y}}=v_{2}(x,y)$\\
\end{frame}
 	 \begin{frame}
 	\begin{block}{}
\textbf{\large{Le sch\'{e}ma Avant-Arri\`{e}re:}}
\begin{align} \frac{\partial{v}_{2}(x_{i},y_{j})}{\partial{y}}&\simeq \frac{1}{h^{2}}[k^{2}_{i,j+1}u_{i,j+1}-(k^{2}_{i,j}+k^{2}_{i,j+1})u_{i,j}+k^{2}_{i,j}u_{i,j-1}]\nonumber\\&+\xi^{av}_{y}(x_{i},y_{j}) \label{318*}\end{align}
O\`{u}  $\xi^{av}_{y}(x_{i},y_{j})$ est l'erreur de troncature donn\'{e}e par:\\
\begin{align}
\xi^{av}_{y}(x_{i},y_{j})=&-\frac{h}{2}\left( \frac{\partial^{2}{k}_{2}}{\partial{y^{2}}}(x_{i},y_{j})\frac{\partial{u}}{\partial{y}}(x_{i},y_{j},t_{m})\right. \nonumber\\&+\left. \frac{\partial{k_{2}}}{\partial{y}}(x_{i},y_{j})\frac{\partial^{2}{u}}{\partial{y^{2}}}(x_{i},y_{j},t_{m})\right) +O(h^{2}) \end{align}
\end{block}
\end{frame}
\begin{frame}
\begin{block}{}
\textbf{\large{Le sch\'{e}ma Arri\`{e}re-Avant:}}
\begin{align}  \frac{\partial{v}_{2}(x_{i},y_{j})}{\partial{y}}&\simeq \frac{1}{h^{2}}[k^{2}_{i,j}u_{i,j+1}-(k^{2}_{i,j}+k^{2}_{i,j-1})u_{i,j}+k^{2}_{i,j-1}u_{i,j-1}]\nonumber\\&+\xi^{ar}_{y}(x_{i},y_{j})\label{323*} \end{align}
O\`{u}  $\xi^{ar}_{y}(x_{i},y_{j})$ est l'erreur de troncature donn\'{e}e par:
\begin{align}
\xi^{ar}_{y}(x_{i},y_{j})=&\frac{h}{2}\left( \frac{\partial^{2}{k}_{2}}{\partial{y^{2}}}(x_{i},y_{j})\frac{\partial{u}}{\partial{y}}(x_{i},y_{j},t_{m})\right. \nonumber\\&+\left. \frac{\partial{k_{2}}}{\partial{x}}(x_{i},y_{j})\frac{\partial^{2}{u}}{\partial{y^{2}}}(x_{i},y_{j},t_{m})\right) +O(h^{2}) 
\end{align}
\end{block}
  \end{frame}
\begin{frame}
	En prenant la moyenne des deux expressions ($\ref{318*}$) et ($\ref{323*}$) 
	\begin{align}
\frac{\partial}{\partial{y}}(k_{2}(x,y)\frac{\partial{u}}{\partial{y}})\nonumber&=\frac{1}{2h^{2}}[ (k^{2}_{i,j}+k^{2}_{i,j+1})u_{i,j+1}-(2k^{2}_{i,j}+k^{2}_{i,j+1}+k^{2}_{i,j-1})u_{i,j}\nonumber\\&+(k^{2}_{i,j}+k^{2}_{i,j-1})u_{i,j-1}] +\frac{1}{2}(\xi^{av}_{y}(x_{i},y_{j})+\xi^{ar}_{y}(x_{i},y_{j}))\end{align}
	 Le sch\'{e}ma final par rapport \'{a} x et y est donn\'{e}e par
	  \begin{block}{}
		\begin{align}
		u^{m}_{i,j}\nonumber=&- \gamma\left[  k^{1}_{i,j}+k^{1}_{i+1,j}\right]u^{m+1}_{i+1,j}- \gamma\left[k^{1}_{i,j}+k^{1}_{i-1,j}\right]u^{m+1}_{i-1,j}\\&+ \left[ \gamma(2k^{1}_{i,j}+k^{1}_{i+1,j}+k^{1}_{i-1,j}+2k^{2}_{i,j}+k^{2}_{i,j+1}+k^{2}_{i,j-1})+1\right]u^{m+1}_{i,j}\nonumber\\
		&- \gamma\left[  k^{2}_{i,j}+k^{2}_{i,j+1}\right]u^{m+1}_{i,j+1}- \gamma\left[ k^{2}_{i,j}+k^{2}_{i,j-1}\right]u^{m+1}_{i,j-1}\label{326*}\end{align}
		telle que  $ \gamma =\frac{k}{2h^{2}}$\\
	\end{block}
\end{frame}
\begin{frame}
\textbf{Discr\'{e}tisation des conditions aux limites}\\
 La discr\'{e}tisation des conditions aux limites  du probl\`{e}me ($\ref{a}$) est donn\'{e}e respectivement par \begin{equation}u^{m}_{i,j}=0\end{equation}  \begin{equation}u^{0}_{i,j}=0\end{equation}
La discr\'{e}tisation compl\`{e}te du probl\`{e}me ($\ref{a}$) est donn\'{e}e par:
\begin{block}{}
\begin{equation}
\begin{cases}
u^{m}_{i,j}=- \gamma[  k^{1}_{i,j}+k^{1}_{i+1,j}]u^{m+1}_{i+1,j}- \gamma[k^{1}_{i,j}+k^{1}_{i-1,j}]u^{m+1}_{i-1,j}\\+\left[ \gamma(2k^{1}_{i,j}+k^{1}_{i+1,j}+k^{1}_{i-1,j}+2k^{2}_{i,j}+k^{2}_{i,j+1}+k^{2}_{i,j-1})+1\right] u^{m+1}_{i,j}\\- \gamma[  k^{2}_{i,j}+k^{2}_{i,j+1}]u^{m+1}_{i,j+1}- \gamma[k^{2}_{i,j}+k^{2}_{i,j-1}]u^{m+1}_{i,j-1}, i,j=1\cdots N, m=1\cdots M\\
u^{m}_{i,j}=0\qquad \\
u^{0}_{i,j}=0\qquad i,j=0\cdots N+1
\end{cases}
\end{equation}
\end{block}
\end{frame}
\begin{frame}

		\textbf{Deuxième cas: $K_{1}= K_{2}=k(x,y)$}\\

	
		La discr\'{e}tisation compl\`{e}te du probl\`{e}me ($\ref{a}$) est donn\'{e}e par:
		\begin{block}{}
		\begin{equation}
	\begin{cases}
		u^{m}_{i,j}=-\mu[  k_{i,j}+k_{i+1,j}]u^{m+1}_{i+1,j}-\mu[k_{i,j}+k_{i-1,j}]u^{m+1}_{i-1,j}\\+[\mu(+4k_{i,j}+k_{i+1,j}+k_{i-1,j}+k_{i,j+1}+k_{i,j-1})+1]u^{m+1}_{i,j}
		\\-\mu[  k_{i,j}+k_{i,j+1}]u^{m+1}_{i,j+1}-\mu[ k_{i,j}+k_{i,j-1}]u^{m+1}_{i,j-1}, i,j=1\cdots N, m=1\cdots M\\
		u^{m}_{i,j}=0\qquad \\
		u^{0}_{i,j}=0\qquad i,j=0\cdots N+1
	\label{350**}	\end{cases}
\end{equation}
telle que $\mu=\frac{k}{2h^{2}}$
	\end{block}
\end{frame}
			\begin{frame}
		\textcolor{red}{\textbf{Etude de la consistance, stabilit\'{e} et convergence:}}
		\begin{itemize}
			\setbeamertemplate{itemize item}[triangle]
			\item Consistance
		\end{itemize}
			\begin{block}{}	
				\begin{equation}
			A(u)=\frac{\partial{u}}{\partial{t}}-\frac{\partial}{\partial{x}}\left[ k_{1}(x,y)\frac{\partial{u(x,y)}}{\partial{x}}\right]-\frac{\partial}{\partial{y}}\left[ k_{2}(x,y)\frac{\partial{u(x,y)}}{\partial{y}}\right]=0\label{350*}\end{equation}\\
			D\'{e}finissons l'op\'{e}rateur approcher par la relation suivante:\\
			\begin{equation}
			\tilde{A}(u)=0 
			\end{equation}
			On dit que le sch\'{e}ma de discr\'{e}tisation est consistant avec l'op\'{e}rateur ($\ref{350*}$) si:
			\begin{equation}
			\arrowvert{A(u)-\tilde{A}(u)}\rvert \longmapsto 0 \quad h\longmapsto 0 \quad et \quad k\longmapsto 0
			\end{equation}
			\end{block}
		
	\end{frame}
	\begin{frame}
	  \begin{align*}
	 \arrowvert A(u)-\tilde{A}(u)\rvert &\leq\frac{k}{2}\arrowvert\frac{\partial^{2}{u}}{\partial{t}^{2}}(x_{i},y_{j},t_{m})\rvert +\frac{h^{2}}{12}\arrowvert k(x_{i},y_{j})\frac{\partial^{4}{u}}{\partial{x^{4}}}(x_{i},y_{j},t_{m})\rvert\\&\phantom{=}+\frac{h^{2}}{6}\arrowvert\frac{\partial{k}}{\partial{x}}(x_{i},y_{j})\frac{\partial^{3}{u}}{\partial{x^{3}}}(x_{i},y_{j},t_{m})\rvert\\&\phantom{=}+\frac{h^{2}}{12}\arrowvert k(x_{i},y_{j})\frac{\partial^{4}{u}}{\partial{y^{4}}}(x_{i},y_{j},t_{m})\rvert\\&\phantom{=}+\frac{h^{2}}{6}\arrowvert\frac{\partial{k}}{\partial{y}}(x_{i},y_{j})\frac{\partial^{3}{u}}{\partial{y^{3}}}(x_{i},y_{j},t_{m})\rvert
	 \end{align*}
	 Alors : \begin{center}
	 	$	\arrowvert$${A(u)-\tilde{A}(u)}$$\rvert \longmapsto 0 \quad h\longmapsto 0 \quad et \quad k\longmapsto 0$
	 \end{center}
	\end{frame}
	\begin{frame}
	\begin{itemize}
		\setbeamertemplate{itemize item}[triangle]
		\item Stabilit\'{e}
	\end{itemize}
	\begin{block}{Proposition}
		Les sch\'{e}mas implicites ($\ref{326*}$) et ($\ref{350**}$) sont inconditionnellement stable.De plus ils sont consistants et l'erreur de troncature locale de ces deux sch\'{e}mas sont not\'{e}s $(\mathbb{E}^{i}_{i,j})$ v\'{e}rifient:\\
		$\arrowvert\mathbb{E}^{i}_{i,j}\lvert\leq C^{i}(k+h^{2})$ ou $C^{i}$ est une constante.
	\end{block}
	\textbf{Etude de la stabilit\'{e} dans le cas $k_{1}=k_{2}=c$ constante}
	\begin{block}{}
	      \begin{equation}
		\begin{cases}
		\frac{\partial{u}}{\partial{t}}-c\left( \frac{\partial^{2}{u}(x,y)}{\partial{x^{2}}}+\frac{\partial^{2}{u}(x, y)}{\partial{y^{2}}}\right) =0 & \text{ dans $ [0,1]^{2}\times[0,T]$}\\
		u(x,y,t)=0 & \text{sur $\varGamma $ }\\
		u(x,y,0)=0   
		\end{cases}\end{equation}
	\end{block}
\end{frame}
\begin{frame}
		Le sch\'{e}ma implicite obtenue par la m\'{e}thode des différences finies est:
		\begin{equation}u^{m}_{i,j}=-\lambda u^{m+1}_{i-1,j}-\lambda u^{m+1}_{i+1,j}-\lambda u^{m+1}_{i,j-1}+(1+4\lambda)u^{m+1}_{i,j}-\lambda u^{m+1}_{i,j+1}\label{355*}\end{equation}$avec: \lambda=\frac{ck}{h^{2}}$\\
		\begin{block}{M\'{e}thode de Von-Neuman}
		Le principe de Von-Neuman est de pos\'{e}e que $	u^{m}_{i,j}=\varphi(mk)e^{\mathbf{i}\alpha ih}.e^{\mathbf{i}\theta jh}$ cela permet de calculer 
		\begin{equation}
		\epsilon(m,i,j,\alpha,\theta)=\dfrac{\varphi((m+1)k)}{\varphi(mk)}
		\end{equation}  
		une condition suffisante de stabilit\'{e} est que \begin{equation}
		\mid\epsilon\mid\leq 1\quad\forall m,\forall i,\forall j,\forall\alpha ,\forall \theta
		\end{equation}
		\end{block}
	\end{frame}

\begin{frame}
	On appliquant la m\'{e}thode de Von-Neuman
	au sch\'{e}ma ($\ref{355*}$) on obtient:

	\begin{equation}\epsilon=\frac{1}{1-4\lambda(\sin \frac{\alpha h}{2}+\sin\frac{\theta h}{2})}\end{equation}
	Nous avons donc toujours $\arrowvert \epsilon\lvert\leq1$  le sch\'{e}ma ($\ref{355*}$) est donc inconditionnellement stable.
	\begin{itemize}
		\setbeamertemplate{itemize item}[triangle]
		\item Convergence
	\end{itemize}
	\begin{block}{}

Comme le sch\'{e}ma implicite ($\ref{355*}$) est consistant et stable alors d'apr\'{e}s le th\'{e}or\`{e}me de Lax il est convergent.

		
	\end{block}
	\end{frame}

\end{document} 