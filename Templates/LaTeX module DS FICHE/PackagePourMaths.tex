\documentclass[12pt,a4paper,oneside,twocolumn]{article} %type et options du document: autres options ;
%fleqn, formules mathématiques à gauche
%leqno, centrées et les numéros sont placés à gauche.

\usepackage[utf8]{inputenc}%codage des caractères
\usepackage[french]{babel}%pour la langue
\usepackage[T1]{fontenc}%taper directement les caractères accentués
\usepackage{amsmath,amsfonts,amssymb}%caractères mathématiques de beauté.
\usepackage{mathrsfs} 
\usepackage[np]{numprint}
\usepackage[left=2cm,right=2cm,top=2cm,bottom=2cm]{geometry} %marges
\usepackage{array} %pour les tableaux à contenu mathématique
\usepackage{cancel} %pour barrer un texte
\usepackage{color} %pour utiliser les couleurs
\usepackage{colortbl} %pour colorier les cellules d’un tableau
\usepackage{enumerate} %pour modifier le motif d’une énumération
\usepackage{fancybox} %pour les encadrements
\usepackage{fancyhdr} %pour le titre courant (les en-tête etc.)
\usepackage{graphicx} %pour l’insertion d’image
\usepackage{hyperref} %pour les liens hypertexte
\usepackage{lscape} %pour un format « paysage »
\usepackage{makeidx} %pour la création d’un index
\usepackage{multicol} %fusionner des colonnes d’un tableau et écrire un texte sur plusieurs colonnes
\usepackage{multido} %pour effectuer des boucles
\usepackage{multirow} %pour fusionner des lignes d’un tableau
\usepackage[thmmarks,amsmath]{ntheorem}%pour la personnalisation des théorèmes
\usepackage{pifont} %pour les symboles ding(12)
\usepackage{pstricks} %pour les dessins géométriques
\usepackage{rotating} %pour tourner du texte
\usepackage{tabularx} %pour les tableaux dont on fixe la longueur totale
\usepackage{titlesec} %pour changer la police des titres de niveaux de hiérarchie
\usepackage{url} %pour les adresses éponymes
\usepackage{setspace} %pour modifier les distance de separations
\usepackage[normalem]{ulem} %pour barrer ou hachurer un texte
\usepackage{frcursive} %pour écrit en cursive
\usepackage{pstricks} %pour retourner ou encedrer un texte 
\usepackage{draftwatermark} %pour le filgrame
%\usepackage[firstpage]{draftwatermark} %filegrame selement en premire page
\usepackage{calc} %pour effectuer les opérations + - * /
\usepackage{tablists} %pour 1) 2) 3) horisentales
\usepackage{arcs}
\usepackage{bm}
\usepackage{centernot}%La négation des relations au centre
\usepackage{colortbl}
\setlength{\parindent}{0cm } %espace avant chaque paragraphe
%\setlength{\parskip}{0.5cm}%espace entre paragraphes
\setlength{\arrayrulewidth}{2pt}%largeure trait des tableaux
\renewcommand{\arraystretch}{2.3} %inter ligne tableau
\newlength{\Nom} % crée une longueur nommée Nom.
\settowidth{\Nom}{Maths} %donner à \Nom la longueure du texte Maths
\newcommand{\barre}[1]{
\settowidth{\Nom}{#1} #1\hspace{-\Nom} \rule[0.8ex]{\Nom}{1pt}}
 \everymath{\displaystyle}
\numberwithin{equation}{section}
 %numération des formules suivre celle des sections (chapitres ...)
\DeclareMathOperator{\sh}{sh}
\DeclareMathOperator{\PGCD}{PGCD}
\DeclareMathOperator{\e}{e}
%\newcommand{\C}{\ensuremath{\mathbb{C}}}
%\renewcommand{\be}{\begin{enumerate}}
%\renewcommand{\ee}{\end{enumerate}}
\usepackage{pst-plot} %trassage des courbes

\begin{document}%\huge
\begin{verbatim}
\begin{spacing}{.5}
texte\\ texte 
\end{spacing}
\end{verbatim}
\begin{spacing}{.5}
texte\\ texte 
\end{spacing}

\begin{verbatim}
\begin{spacing}{1}
texte\\ texte 
\end{spacing}
\end{verbatim}
\begin{spacing}{1}
texte\\ texte 
\end{spacing}


\begin{verbatim}

\end{verbatim}



\begin{pspicture}(-2,-2)(4,4)
\psset{algebraic=true}
\psaxes{->}(0,0)(-2,-2)(4,4)
\psplot{-1}{2}{x^2-1}
\psplot{-1}{2}{x-1}
\end{pspicture}
\begin{verbatim}

\end{verbatim}

\begin{tabular}{|c|}
\hline Cell. A\\ \hline
\parbox[c][2cm][c]{3cm}{%
\centering Cell. B}\\
\hline Cell. C\\ \hline
\end{tabular}




\newcolumntype{K}[1]{>{\columncolor{#1}}c}
\begin{tabular}{|K{red}|c|}
\hline
BLA-BLA&Bla-bla\tabularnewline\hline
BLA-BLA&Bla-bla\tabularnewline\hline
\end{tabular}

\newcolumntype{L}[1]{>{\raggedleft}m{#1}}
\begin{tabular}{|L{2.5cm}|L{3.5cm}|}
\hline
BLA-BLA&Bla-bla
\tabularnewline\hline
\end{tabular}

%\newcolumntype{C}{>{\itshape\bfseries}c}
%\begin{tabular}{|l|C|C|}
%\hline
%Jean & 12,3 & Admis \\ \hline ...

\begin{tabular}{c>$c<$}
\hline
(1)&\sum_{k=0}^{n}\ k \\ \hline
(2)&\sum_{k=0}^{n}\ k^2\\ \hline
\end{tabular}


\begin{tabular}{>{\itshape}l r<{~DH}}
Article & Prix \\ 
\hline
Article 1 & 12 \\ 
Article 2 & 15 \\
\end{tabular}

\begin{tabular}{|p{5cm}|l|}
\hline
le petit ... bois\par pour ... -grand
& le loup \\ \hline
(les gentils) & (le méchant)\\
\hline \end{tabular}

\begin{tabular}{|c|p{4cm}|}
\hline
1&\multirow{2}{4cm}{texte texte beaucoup, beaucoup trop long }\\
\cline{1-1}
2&\\
\hline
\end{tabular}

\begin{tabular}{|c|c|}
\hline
\multirow{2}{*}{a} & b\\
\cline{2-2}
  & c\\
\hline
\end{tabular}

\begin{tabular}{|p{2cm}|r|}
\hline
\centering Elève & Note\\ \hline
Jean & 12,3 \\ \hline
\end{tabular}

\begin{tabular}{|>{\centering}p{2cm}|r|}
\hline
Elève& Note\tabularnewline
\hline
Jean& 12,3 \tabularnewline \hline
François & 9,7\tabularnewline \hline
Gilbert& 18,7\\
\hline
\end{tabular}

\begin{tabular}{|c|>{\raggedleft}m{2cm}|}
\hline
Début & 14 h 25 \tabularnewline
\hline
Fin & 15 h 32 \tabularnewline
\hline
\end{tabular}
%• \raggedright pour aligner à gauche (3) ;
%• \centering pour centrer ;
%• \raggedleft pour aligner à droite.
\begin{tabular}{|c|b{4cm}|c|}
\hline
centré & largeur de la colonne fixée à 4 cm ... & centré \\
\hline
\end{tabular}

\begin{tabular}{|c|m{4cm}|c|}
\hline
centré & largeur de la colonne fixée à 4 cm ... & centré \\
\hline
\end{tabular}

\begin{tabular}{|c|p{4cm}|c|}
\hline
centré & largeur de la colonne fixée à 4 cm ... & centré \\
\hline
\end{tabular}

\begin{tabular}{|r|r|r|}
\hline
texte & texte & texte \\
\cline{2-3}
\end{tabular}

%\begin{tabular}[t]{cc}
%\begin{tabular}[b]{cc}
\begin{tabular}{l*{3}{c}|*{2}{r|}}% l ccc r|r|
Rang & 1 & 2 & 3 & 4 & 5 \\
Candidat & 2 & 84 & 15 & 23 & 1 \\
\end{tabular}

$\centernot \Leftrightarrow$

$\mathit{\Gamma}$   $\Gamma$ 

$ \sh ~~ sh$
$\PGCD(x,y)$ 

\newcommand{\Degre}{\ensuremath{^\circ}}
la température vaut 30\Degre\ à 16~h
$\cos\alpha=0,5$ donc $\alpha=60\Degre$

\newcommand{\ER}{\ensuremath{\mathbb{R}}}
%affiche le symbole R dans un environnement mathématique même en mode texte
\ER{} est l’ensemble\dots\par
$\forall y \in \ER$

\begin{multline} %l ccc...ccc r
(2x-1)^8=256\,x^8-1\,024\,x^7\\
+1\,792\,x^6-1\,792\,x^5\\
+1\,120\,x^4-448\,x^3\\+112\,x^2-16\,x+1
\end{multline}

\begin{align*}
(3 + 2\,\mathrm{i})^2
&= 3^2 + 2 \times 3 \times 2\,\mathrm{i}
+ (2\,\mathrm{i})^2 \\
\intertext{Le terme ....}
&= 9 + 12\,\mathrm{i} -4 \\
&= 5 + 12\,\mathrm{i}
\end{align*}

\begin{equation} 
\begin{split} % 2 r l
(3 + 2\,\mathrm{i})^2 
& = 3^2 + ... + (2\,\mathrm{i})^2 \\
& = 9 + 12\,\mathrm{i} - 4 \\
& = 5 + 12\,\mathrm{i}
\end{split}
\end{equation}

\begin{align*}
a&=1 & b&=2 & c&>3\\
a’&=3 & b’&=0 & c’& <-2
\end{align*}

\begin{align}
(3 + 2\,\mathrm{i})^2 &= 3^2 + 2 ... \\
& = 9 + ...
\end{align}

{\setlength{\jot}{1cm}
\begin{eqnarray*}
x&=&1+2\\&=&3
\end{eqnarray*}}

\begin{eqnarray}% 3 r c l
(3 + 2\,\mathrm{i})^2 & = &
3^2 + 2 \times ... ^2\\
& = & 9 + 12\,\mathrm{i} - 4 \nonumber\\
& = & 5 + 12\,\mathrm{i}
\end{eqnarray}
\begin{eqnarray*}
(3 + 2\,\mathrm{i})^2 & = &
3^2 + 2 \times ... ^2 \\
& = & 5 + 12\,\mathrm{i}
\end{eqnarray*}

Soit l’équation
\begin{equation}
a\,x^2+b\,x+c=0 \label{eq:SD}
\end{equation}
L’équation~\eqref{eq:SD}

\begin{enumerate}[\roman{enumi}]
\item 
\item 
\end{enumerate}
\begin{enumerate}[\Roman{enumi}]
\item 
\item 
\end{enumerate}

b{\center d \endcenter}n

 \begin{equation*}
1+2=3 
\tag{A}
\end{equation*}

$\widetilde{f}$ $\tilde{f}$ $\bar{f}$ $ \dot{A} $ $ \vec{u} $ $ f^{(n)} $  $f^n$



$\overrightarrow{AC}=\overrightarrow{%
\underline{\phantom{A}}B}+\overrightarrow{%
\underline{\phantom{AB}}}$

$\sqrt{x} + \sqrt{X} + \sqrt{\vphantom{X}x}$

$f(x)=\left\{\begin{array}{%
l @{\qquad} r @{~\leqslant x <~} l}
x^2-24 & -5 & -2\\
x+2 & -2 & \phantom{-}3
\end{array}\right.$

$x^2=3x-2 \iff x^2-3x+2=0$\par
$\phantom{x^2=3x-2} \iff (x-1)(x-2)=0$\par
$\phantom{x^2=3x-2} \iff (x-1)(x-2)=0$

$\textrm{C}^{13}_{\phantom{1}7}$  $\textrm{C}^{13}_{7}$

$\cancel{A}$ $\bcancel{B}$ $\xcancel{C}$

$\mathbb{D}$  $\mathcal{D}$  $\mathscr{D}$  $\mathfrak{D}$

$\bm{f(x)=3\,x^2-1}$ $ f(x)=3\,x^2-1 $

\colorbox{yellow}{$f(x)=3\cos(2\,x)$}

{\setlength{\mathsurround}{20pt}
blabla \fbox{$ f(x)=\int_0^{\pi}
\cos x\,\mathrm{d}x=0 $} blabla}

$ \boxed{\mathrm{i}^2=-1}$

$\lim_{\substack{x \to 0 \\ x > 0}}$

$  \xrightarrow[texte bas]{texte haut} $

$\underbrace{\cos^2x+\sin^2x}_{=1}
+\overbrace{2\cos x\sin x}^{=\sin 2x}+...$

$\overarc{AB}$
$\widehat{ABC}$
$\overset{\frown}{AB}$

\big( \Big( \bigg (\Bigg(
\big\} \Big\} \bigg\} \Bigg\}
\big\| \Big\| \bigg\| \Bigg\|

$ \ldots \cdots \vdots \ddots $

$a \colon A \to B$

$\{x \in A \mid A \neq 0\}$

$a \equiv b \mod n$ ou $a \equiv b \ [n]$

$\binom{n}{p}$

$A \cup B$ $A \cap B$    $\varnothing$

$D \perp D’$

$\begin{cases}
-x & \text{si $x$ est négatif} \\
x & \text{si $x$ est positif (ou nul)}
\end{cases}$

$\begin{pmatrix}
\dfrac{1}{2} & \dfrac{1}{3} \\[3mm]
\dfrac{1}{4} & \dfrac{1}{5} \\
\end{pmatrix}$

$M=\bordermatrix{
& A & B & C \cr
1& a & b & c \cr
2& d & e & f \cr
3& g & h & i \cr
}$

$\begin{matrix} a&b\\ c&d \end{matrix}$
$\begin{pmatrix} a&b\\ c&d \end{pmatrix}$
$\begin{vmatrix} a&b\\ c&d \end{vmatrix}$
$\begin{Vmatrix} a&b\\ c&d \end{Vmatrix}$
$\begin{bmatrix} a&b\\ c&d \end{bmatrix}$
$\begin{Bmatrix} a&b\\ c&d \end{Bmatrix}$

$\arg z$

$\left\| \overrightarrow{AM} \right\| $

$ \overline{texte} $

$ \sum_a^b$ $ \sum\nolimits_a^b$

$<u\cdot v>$ 

$\frac12$

$\to$ $\mapsto$ $\longmapsto$
$\circlearrowleft$ $\circlearrowright$
$\curvearrowleft$ $\curvearrowright$
$\nearrow$ $\searrow$
$\nwarrow$ $\swarrow$
 $A \implies B$ $A \Rightarrow B$
$A \Longleftrightarrow B$ $A \iff B$
$\leftrightarrow$ $\Leftrightarrow$ $\longleftrightarrow$ 
$\Longleftrightarrow$ 
$\rightarrow$  $\longrightarrow$ $\Rightarrow$ $\Longrightarrow$

\np{6e-12} \np[kg]{91} \\
\np{3,4567} \qquad $ 3,4567 $

\mbox{$1+x+x^2+x^3+x^4$} %equation insecable

$x=1 \text{ et donc } y=2$
$\mathrm{2+5i}$ \quad $2+5i$

$2^{2^{2^2}}$
$2^{2^{%
{\scriptstyle {2^{\scriptstyle 2}}}}}$

%  \reversemarginpar
% \normalmarginpar 

texte\rule[9mm]{3mm}{7mm}texte

\setlength{\marginparsep}{0pt}
texte \dotfill
\marginpar{texte dans la marge texte dans la marge}
texte \\ texte

\begin{tabenum}
\tabenumitem Item 1
\tabenumitem Item 2 \\
\tabenumitem Item 3
\end{tabenum}

\begin{list}{$\square$}{}
\item C
\item D
\end{list}

texte
{\setlength\parindent{3mm}
\begin{itemize}
\item[$\bullet$] item 1 ;
\item[$\bullet$] item 2.
\end{itemize}
}
texte

\ding{51}

\begin{enumerate}[*]
\item texte
\item texte
\end{enumerate}

\S 

\centerline{texte}

--------------[-------------- \hspace*{-2cm} / / / / /
 / / / / /  \hspace*{-2cm} --------------[--------------

\begin{enumerate}
\item texte
\addtocounter{enumi}{1}
\item texte
\end{enumerate}

\begin{dingautolist}{192}
\item texte 
\item texte
\item texte
\end{dingautolist}

\begin{enumerate}[{[1]}]
\item AAA
\item BBB
\end{enumerate}

\begin{enumerate}[{Priorité} 1 :]
\item Parenthèses
\item Multiplication
\item Addition
\end{enumerate}

\no   \No

\rule{2cm}{1mm} Texte \rule{4cm}{1mm}

abc\rule[3mm]{5cm}{0.25cm}

\pagecolor{gray!15} 

\fcolorbox{red}{yellow}{Vrai}\\
\colorbox{red}{exemple} 

\fbox{\begin{minipage}{6cm}
blabla \\ blabla \\ blabla
 \end{minipage}}

\fbox{\parbox[][2.5cm][c]{2cm}{blabla \\ blabla \\ blabla}}

\raisebox{0.5ex}{3}\slash\raisebox{-0.5ex}{4}\\
texte \raisebox{3mm}{texte} texte \raisebox{-3mm}{texte}

\framebox[\width+2cm]{long cadre = long Texte + 2 cm}
\framebox[\linewidth]{De la largeur de la ligne} \\
\framebox[2\width]{2 fois longueur du texte}

{\fboxsep=5mm \fboxrule=1mm \fbox{Encadré}}\\
\fbox{Vrai} \fbox{\fbox{Vrai}} \fbox{\fbox{\fbox{Vrai}}}

\texttt{\makebox[0pt][l]{///}bon}
La \makebox[3cm]{} a pour\\
La \makebox[3cm]{\dotfill} a pour\\
\mbox{$\sin^2 x+\cos^2 x=1$} \\
\makebox[6cm]{Texte centré} \\
\makebox[6cm][l]{Texte à gauche} \\
\makebox[6cm][r]{Texte à droite} \\
\makebox[6cm][s]{Sur toute la largeur}

\ding{43} \\
b \dingfill{33}\\
\ding{114}\hspace{-.7em}\ding{51}

\SetWatermarkText{abc filgrame}
\SetWatermarkAngle{65}
\SetWatermarkColor{red}
\SetWatermarkScale{.8}

\url{https://www.facebook.com/groups/latex.fans/}

aaaa texte texte texte texte texte texte texte texte texte texte texte texte texte texte texte bbbb 
\begin{quotation}
ccccc texte texte texte texte texte texte texte texte texte texte texte texte texte dddd 
\end{quotation}

texte
\begin{quote}
texte\\
texte\\
texte
\end{quote}

\reflectbox{texte en miroir}

\fbox{texte encadré}

\parbox{1.5cm}{texte texte texte}

\rotatebox{45}{texte} texte  \rotatebox{-45}{texte}

\resizebox{!}{0.35cm}{Texte}

\resizebox{5cm}{0.35cm}{Texte}

\scalebox{5}[3]{texte}

\textcolor{red}{\underline{\textcolor{green}{texte}}}

texte \textcolor{red}{texte} texte {\color{blue} texte } texte

\primo \secundo \tertio  \quarto \FrenchEnumerate{7}

1\up{er}

conversation de 5 en chiffres Roman \MakeUppercase{\romannumeral 5} 

non cursive \mbox{\begin{cursive} cursive \end{cursive}} non cursive

\begin{cursive} Bouchiar Abdelhakim \end{cursive}

\uline{texte souligné}  \underline{texte souligné}

\uwave{Vague}

\sout{texte Barré}

\xout{texte Hachuré}

\psovalbox{Faux}

\shadowbox{Ombre}

\ovalbox{Entouré}

\Ovalbox{Entouré}

{\cornersize{3} \ovalbox{Encadré avec modification d'angle}}

\doublebox{Double}

\textcircled{b}

\barre{abcdefghijklm} %voir newcommand

%\setlength{\Nom}{2cm} %donner à \Nom la longueure 2cm

%\addtolength{\parskip}{10pt} % ajouter 10pt à \parskip

A\hspace{\stretch{2}}B\hspace{\stretch{5}}C

\begin{center} texte avant environnement \end{center} 
\smallskip text après environnement saut d’un quart de ligne
%\medskip text après environnement saut d’une demi-ligne
%\bigskip text après saut d’une ligne 

\rule{5cm}{1mm}

a \\ b \\[5mm] c \\[2cm] d

a - b -- c --- d

2~cm pour ne pas séparer la quantité de l'unité en cas de retour à la ligne

\ldots ou \dots

\og dermagogue  \fg

a \dotfill b

a \hfill b

À gauche\hfil au centre %une seule l !!!!!!!
\begin{verbatim}

\end{verbatim}
\centerline{au milieu}



%%not fond
%\usepackage{xlop} % réaliser des calculs arithmétiques.
%\opadd{356}{78}

\end{document}